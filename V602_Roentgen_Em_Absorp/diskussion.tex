
Die Abweichungen der einzelnen Abschirmkonstanten $\su{\sigma_k}$ sind in der
untenstehenden Tabelle \ref{tab:abw} zu sehen.
\begin{table}
  \centering
  \begin{tabular}{c c c c }
    \toprule
    Element & Literaturwert & Gemessener Wert & Relative Abweichung$\,/\;\%$ \\
    \midrule
    Zink       & 3.56 & 4.39 & 23.3 \\
    Germanium  & 3.66 & 3.56 &  3.0 \\
    Brom       & 3.85 & 3.74 &  3.0 \\
    Strontium  & 3.98 & 3.69 &  7.3 \\
    Zirkonium  & 4.08 & 5.70 & 39.7 \\
    Gold       & 3.58 & 1.70 & 52.5 \\
    \bottomrule
  \end{tabular}
  \caption{Relative Abweichungen der Abschirmkonstanten.}
  \label{tab:abw}
\end{table}
Ein Grund für die teilweise sehr hohen Abweichungen ist der Zustand der einzelnen
Absorber, da die Elementschicht teilweise beschädigt ist. Desweiteren lässt sich
der Winkel nicht exakt aus den Plots ablesen und muss noch mit dem Faktor $0.5$ multipliziert
werden, was zu weiteren Fehlern führt. Auch ist nicht
sicher, ob die Absorber richtig auf dem Zählrohr befestigt wurden, da diese möglicherweise
schief auf der Apparatur steckten.

Der gemessene Wert für die Maximalenergie liegt bei $24.61\keV$ und hat somit
eine Abweichung von $29\,\%$ zum eingestellten Wert $35\keV$. Hier muss es sich
um einen systematischen Fehler handeln, der sich nicht erklären lässt.
Die mittels Mosleyschem Gesetz berechnete Rydbergkonstante $R_\su{\infty}=1.7\,
\cdot10^7\mt^{-1}$ weicht um $54\dgr$ vom Literaturwert $R_\infty=1.1\,\cdot10^7
\mt^{-1}$\cite{Ryd} ab. Auch hier handelt es sich um einen systematischen Fehler,
der nicht erklärbar ist.
Die Abweichung vom Braggwinkel beträgt $0.6\dgr$ und stimmt im Rahmen der 
Messungenauigkeiten überein.
Aufgrund des guten Auflösungsvermögen lassen sich die Absorptionsspektren 
gut ablesen.
