Der Versuchsaufbau besteht im wesentlichen aus einer Kupfer-Röntgenröhre,
einem LiF-Kristall und einem Geiger-Müller-Zählrohr. Weitere Bestandteile des
Aufbaus sind in Abbildung \ref{fig:aufbau} zu sehen.
\begin{figure}[H]
  \centering
  \includegraphics[width=0.8\textwidth]{bilder/aufbau.pdf}
  \caption{Versuchsaufbau zur Messung verschiedener Emissions- und Absorptionsspektren
  \cite{V602}}
  \label{fig:aufbau}
\end{figure}
Die Elektronik ist im Röntgengerät integriert und kann per Rechner, oder
manuell bedient werden. Die Beschleunigerspannung wird für alle Messungen auf
$U_\su{b}=35\kVolt$ eingestellt. Der Emissionsstrom bleibt bei konstant
$I=1\mA$. Bevor die Messungen gestartet werden, wird überprüft, ob die $1\mm$
Blende und der LiF-Kristall in der Halterung stecken. Weiter zu beachten ist,
dass die Schlitzblende waagerecht auf dem Geiger-Müller-Zählrohr sitzt. Dies
ist wichtig, da die Röntgenstrahlen horizontal emittiert werden.
Für die Absorptionsmessungen können Blenden mit unterschiedlichen Absorbern
verwendet werden.
\subsection{Überprüfung der Bragg-Bedingung}
Die Bragg-Bedingung wird überprüft, indem der LiF-Kristall auf einen festen
Kristallwinkel $\theta=14\,\si{\degree}$ eingestellt wird. Das Zählrohr misst nun in einem
Winkelbereich von $\alpha_1=26\,\si{\degree}$ bis $\alpha_2=30\,\si{\degree}$ mit einem Winkelzuwachs
von $\Delta\alpha=0.1\,\si{\degree}$. Aus den gemessenen Daten wird das Maximum der Kurve
bestimmt und mit dem Sollwert verglichen.
\subsection{Emissionsspektrum einer Cu-Röntgenröhre}
Das Emissionsspektrum wird im Programm auf den 2:1 Koppelmodus gestellt. Das
Röntgenspektrum wird nun im Winkelbereich $4\,\si{\degree}\leq\theta\leq 26\,\si{\degree}$ in $0.2\,\si{\degree}$
Schritten gemessen. Das Röntgenspektrum besitzt hierbei die Beugungsordnung
$n=1$.
\subsection{Das Absorptionsspektrum}
Für die nächste Messreihe wird ein Germaniumabsorber vor das Zählrohr gesetzt.
Das Absorptionsspektrum wird dann in $0.1\,\si{\degree}$-Schritten gemessen. Die Messzeit pro
Winkel liegt bei $t=20\sek$. Die Messung wird für Absorber mit verschiedenen
Kernladungszahlen wiederholt.
