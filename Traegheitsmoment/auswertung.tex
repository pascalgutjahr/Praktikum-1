Um die Winkelrichtgröße nach Formel \eqref{eqn:D} berechnen zu können, werden
die Daten aus Tabelle \ref{MessD} benötigt.
\begin{table}[H]
  \begin{tabular}{c c c}
    \toprule \\
    $\varphi$ & F & r \\
    Grad & $\si{\milli\newton}$ & $\si{\meter}$ \\
    \midrule \\
    30  &    15   &   0.27  \\
    30  &    55   &   0.15  \\
    40  &    78   &   0.15  \\
    50  &   108   &   0.15  \\
    30  &    39   &   0.19  \\
    10  &    43   &   0.04  \\
    30  &    90   &   0.10  \\
    20  &    48   &   0.08  \\
    60  &    57   &   0.28  \\
    50  &    75   &   0.18  \\
    \bottomrule
  \end{tabular}
  \caption{Messdaten zur Bestimmung von D}
  \label{MessD}
\end{table}
Der Wert für den Winkel wird noch mit der Formel
\begin{equation*}
  \frac{\varphi}{360}\cdot2\pi
\end{equation*}
umgerechnet. 
