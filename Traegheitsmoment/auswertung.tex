Um die Winkelrichtgröße nach Formel \eqref{eqn:D} berechnen zu können, werden
die Daten aus Tabelle \ref{MessD} benötigt.
\begin{table}[H]
  \centering
  \begin{tabular}{c c c}
    \toprule
    $\varphi$ & F & r \\
    Grad & $\si{\milli\newton}$ & $\si{\meter}$ \\
    \midrule
    30  &    15   &   0.27  \\
    30  &    55   &   0.15  \\
    40  &    78   &   0.15  \\
    50  &   108   &   0.15  \\
    30  &    39   &   0.19  \\
    10  &    43   &   0.04  \\
    30  &    90   &   0.10  \\
    20  &    48   &   0.08  \\
    60  &    57   &   0.28  \\
    50  &    75   &   0.18  \\
    \bottomrule
  \end{tabular}
  \caption{Messdaten zur Bestimmung von D}
  \label{MessD}
\end{table}
Der Wert für den Winkel wird noch mit der Formel
\begin{equation*}
  \varphi_\su{rad}\frac{\varphi}{360}\cdot2\pi
\end{equation*}
umgerechnet. Nachdem $D$ für die verschiedenen Werte berechnet wurde, wird der
Mittelwert berechnet und der Fehler bestimmt. Somit erhält man für die
Winkelrichtgrße einen Wert von
\begin{equation*}
  D= (0.014 \pm 0.003) \,\si{\newton\meter}.
\end{equation*}
\begin{table}
  \centering
  \begin{tabular}{c c c}
    \toprule
    $\varphi$ & a & T \\
    Grad & \cm & \sek \\
    \midrule
    50    &    8  &   2.96    \\
    50    &   10  &   3.44    \\
    50    &   12  &   3.77    \\
    50    &   14  &   4.29    \\
    50    &   16  &   4.77    \\
    50    &   18  &   5.11    \\
    50    &   20  &   5.55    \\
    50    &   22  &   6.05    \\
    50    &   24  &   6.71    \\
    50    &   26  &   7.22    \\
    50    &   28  &   7.71    \\
    \bottomrule
  \end{tabular}
  \caption{Messwerte zur Bestimmung von $I_\su{D}$}
  \label{messI}
\end{table}
Anschließend wird das Eigenträgheitsmoment der Feder der Apparatur bestimmt,
indem mit den Werten aus Tabelle \ref{messI} ein Graph erstellt wird, bei dem
$T^2$ gegen $a^2$ aufgetragen wird. Abbildung \ref{fig:trag} zeigt diesen Graph.
\begin{figure}[H]
  \centering
  \includegraphics[width=0.8\textwidth]{bilder/tragheit.pdf}
  \caption{Graph zur Bestimmung des Eigenträgheitsmoments}
  \label{fig:trag}
\end{figure}
Mit Hilfe der linearen Regression lässt sich das $I_\su{D}$ nun einfach bestimmen.
Die mit Python 3.5.2 berechneten Werte lauten
\begin{align*}
  a &= (1.42 \pm 0.02)\,\cdot10^{-3} \\
  b &= (6.0 \pm 0.8)\,\cdot10^{-3}.
\end{align*}
Der Parameter a ist hierbei der gesuchte Wert für die Eigenträgheit.
Als nächstes werden die Trägheitsmomente von einem aufrechten und einem liegenden
Zylinder gemessen und mit den theoretischen Werten verglichen. Die Abmessungen
der beiden Zylinder sind mit einer Höhe von $3\cm$ und einem Durchmesser von $7.5\cm$
gleich. Somit unterscheiden sie sich lediglich im Gewicht. Der Aufrechte Zylinder
wiegt $1119.3\gr$ und das Gewicht des liegenden Zylinders liegt bei $1117.3\gr$.
Um das Trägheitsmoment bestimmen zu können, werden beide Körper jeweils 5 Mal um
50° ausgelenkt. Die gemittelten Periodenschwingdauern betragen
\begin{align*}
  T_\su{a} &= (0.97 \pm 0.02)\sek \\
  T_\su{l} &= (0.82 \pm 0.04)\sek.
\end{align*}
Das Trägheitsmoment lässt sich nun mittels einer Umformung von Formel \eqref{eqn:T}
berechnen. Die Formel für den Fehler ergibt sich aus der Gaußschen Fehlerfortpflanzung
und lautet:
\begin{equation*}
  \Delta I = \sqrt{\biggl(\frac{2TD}{4\pi^2}\biggr)^2\cdot\sigma_\su{T}^2 +
  \biggl(\frac{T^2}{4\pi^2}\biggr)^2\cdot\sigma_\su{D}^2}
\end{equation*}
Somit betragen die experimentell festgestellten Werte für die Trägheitsmomente
der Zylinder
\begin{align*}
  T_\su{a} &= (3.4 \pm 0.8)\,\cdot10^{-4}\,\si{\kilo\gram\square\meter} \\
  T_\su{l} &= (2.4 \pm 0.9)\,\cdot10^{-4}\,\si{\kilo\gram\square\meter}.
\end{align*}
Die theoretischen Werte lassen sich mit Formel \eqref{eqn:ZylI} berechnen.
Da der Durchmesser und die Höhe der Zylinder eindeutig bestimmt werden können,
bleibt eine Fehlerrechnung aus.
Die theoretischen Werte lauten:
\begin{align}
  T_\su{a} &= 7.9\,\cdot10^{-4}\,\si{\kilo\gram\square\meter} \\
  T_\su{l} &= 4.8\,\cdot10^{-4}\,\si{\kilo\gram\square\meter}.
\end{align}
