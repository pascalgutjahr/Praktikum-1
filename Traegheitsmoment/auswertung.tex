Um die Winkelrichtgröße nach Formel \eqref{eqn:D} berechnen zu können, werden
die Daten aus Tabelle \ref{MessD} benötigt.
\begin{table}[H]
  \begin{tabular}{c c c}
    \toprule \\
    $\varphi$ & F & r \\
    Grad & $\si{\milli\newton}$ & $\si{\meter}$ \\
    \midrule \\
    30  &    15   &   0.27  \\
    30  &    55   &   0.15  \\
    40  &    78   &   0.15  \\
    50  &   108   &   0.15  \\
    30  &    39   &   0.19  \\
    10  &    43   &   0.04  \\
    30  &    90   &   0.10  \\
    20  &    48   &   0.08  \\
    60  &    57   &   0.28  \\
    50  &    75   &   0.18  \\
    \bottomrule
  \end{tabular}
  \caption{Messdaten zur Bestimmung von D}
  \label{MessD}
\end{table}
Der Wert für den Winkel wird noch mit der Formel
\begin{equation*}
  \frac{\varphi}{360}\cdot2\pi
\end{equation*}
umgerechnet. Nachdem D für die verschiedenen Werte berechnet wurde, wird der
Mittelwert berechnet und der Fehler bestimmt. Somit erhält man für die
Winkelrichtgrße einen Wert von
\begin{equation*}
  D= (0.014 \pm 0.003) \,\si{\newton\meter}.
\end{equation*}
\begin{table}
  \begin{tabular}{c c c}
    \toprule \\
    $\varphi$ & a & T \\
    Grad & \cm & \sek \\
    \midrule
    50    &    8  &   2.96    \\
    50    &   10  &   3.44    \\
    50    &   12  &   3.77    \\
    50    &   14  &   4.29    \\
    50    &   16  &   4.77    \\
    50    &   18  &   5.11    \\
    50    &   20  &   5.55    \\
    50    &   22  &   6.05    \\
    50    &   24  &   6.71    \\
    50    &   26  &   7.22    \\
    50    &   28  &   7.71    \\
  \end{tabular}
  \caption{messwerte zur Bestimmung von $I_\su{D}$
  \label{messI}
\end{table}
Anschließend wird das Eigenträgheitsmoment der Feder der Apparatur bestimmt,
indem mit den Werten aus Tabelle \ref{messI} ein Graph erstellt wird, bei dem
$T^2$ gegen $a^2$ aufgetragen wird. Abbildung \ref{fig:trag} zeigt diesen Graph.
\begin{figure}[H]
  \centering
  \includegraphics[width=0.8\textwidth]{bilder/tragheit.pdf}
  \caption{Graph zur Bestimmung des Eigenträgheitsmoment}
  \label{fig:trag}
\end{figure}
Mit Hilfe der linearen Regression lässt sich das $I_\su{D}$ nun einfach bestimmen.
Die mit Python 3.5.2 berechneten Werte lauten
\begin{align*}
  a &= (1.42 \pm 0.02)\,\cdot10^{-3} \\
  b &= (6.0 \pm 0.8)\,\cdot10^{-3}
\end{align*}
