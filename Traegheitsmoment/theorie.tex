\subsection{Grundlagen}
In der Dynamik von Rotationsbewegungen gibt es drei wichtige Größen:
Das Drehmoment $M$, das Trägheitsmoment $I$ und die Winkelbeschleunigung $\dot
\omega$.

Wird dabei ein ausgedehnter Körper betrachtet, der sich um eine feste Achse bewegt,
ist die Winkelgeschwindigkeit $\omega$ für alle Massenelemente gleich groß.
Für das Trägheitsmoment $I$ gilt dann
\begin{equation}
  I = \sum_i r_\su{i}^2 \cdot m_\su{i}
\end{equation}
mit $r_\su{i}$ als Abstand der einzelnen Massenelemte $m_\su{i}$ zur Drehachse.
Werden diese Massenelemente als infinitisimal klein angesehen, so gilt:
\begin{equation}
  I = \int r^2dm.
\end{equation}
Liegt die Drehachse nicht in einer Schwerpunktachse des Körpers, sondern parallel dazu,
so kann das Trägheitsmoment mithilfe des Steinerschen Satzes bestimmt werden.
Daraus folgt
\begin{equation}
  I = I_\su{s} + m \cdot a^2
\end{equation}
mit $I_\su{s}$ als Trägheitsmoment der Drehachse durch den Schwerpunkt und $a$ als
Abstand der Drehachse zur Schwerpunktachse.

Das Drehmoment $M$ lässt sich über
\begin{equation}
  \vec{M} = \vec{F} \times \vec{r} \label{eqn:M}
\end{equation}
ermitteln. Dabei wirkt die Kraft $\vec{F}$ an einem Punkt mit Abstand $\vec{r}$
zur Drehachse.
Ist das System Schwingungsfähig, so liegt ein Auslenkungswinkel $\varphi$ und eine
Periodendauer $T$ vor. Für eine harmonische Schwingung mit kleiner Auslenkung $\varphi$
folgt
\begin{equation}
  T = 2\pi\sqrt{\frac{I}{D}}. \label{eqn:T}
\end{equation}
$D$ ist dabei die Winkelrichtgröße, welche sich über
\begin{equation}
  D = \frac{M}{\varphi}
\end{equation}
bestimmen lässt.
Für die statische Messmethode wird das Objekt um den Winkel $\varphi$ ausgelenkt
und die Kraft $F$ wird senkrecht zum Bahnradius $r$ gemessen. Für $\varphi = 0 \,\si{\degree}$
gilt
\begin{equation}
  D = \frac{F \cdot r}{\varphi}. \label{eqn:D}
\end{equation}
Bei der dynamischen Messmethode wird auf die Schwingungsdauer $T$ und die Formel
\eqref{eqn:T} zurückgegriffen.
Die Trägheitsmomente der verwendeten Körper lassen sich über
\begin{align}
  \begin{split}
    I_\su{Zylinder, aufrecht} &= \frac{mR^2}{2} \\
    I_\su{Zylinder, liegend}  &= m\biggl(\frac{R^2}{4}+\frac{h^2}{12}\biggr)
  \end{split}
  \label{eqn:ZylI}
\end{align}
\newpage
\subsection{Vorbereitung}
Als Vorbereitung soll das Drehmoment $M$ einer Stange als Funktion des Abstandes $r$
ermittelt werden. Die Kraft beträgt $F = 0.1 \,\si{\newton}$ und die Auslenkung
$\varphi = 45 \,\si{\degree}$.
Aus der vektoriellen Größe $M$ aus \eqref{eqn:M} ergibt sich für die skalare Größe
\begin{equation}
  M = F \cdot r \cdot \sin{(\varphi)}.
\end{equation}
Damit werden die Drehmomente für 10 Abstände im Bereich von $(5-25)\cm$ bestimmt:
\begin{table}
  \centering
  \begin{tabular}{c c}
    \toprule
    $r\,/\cm$ & $M\,/\,\si{\newton\meter}\cdot 10^{-3}$ \\
    \midrule
    5     &  3.54   \\
    7     &  4.95   \\
    9     &  6.36   \\
    11    &  7.78   \\
    13    &  9.19   \\
    15    & 10.61   \\
    17    & 12.02    \\
    19    & 13.44    \\
    21    & 14.45    \\
    23    & 16.26    \\
    25    & 17.68    \\
    \bottomrule
    \end{tabular}
    \caption{Drehmomente einer Stange}
\end{table}
