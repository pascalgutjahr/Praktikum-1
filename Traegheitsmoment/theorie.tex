In der Dynamik von Rotationsbewegungen gibt es drei wichtige Größen:
Das Drehmoment $M$, das Trägheitsmoment $I$ und die Winkelbeschleunigung $\dot
\omega$.

Wird dabei ein ausgedehnter Körper betrachtet, der sich um eine feste Achse bewegt,
ist die Winkelgeschwindigkeit $\omega$ für alle Massenelemente gleich groß.
Für das Trägheitsmoment $I$ gilt dann
\begin{equation}
  I = \sum_i r_\su{i}^2 \cdot m_\su{i}
\end{equation}
mit $r_\su{i}$ als Abstand der einzelnen Massenelemte $m_\su{i}$ zur Drehachse.
Werden diese Massenelemente als infinitisimal klein angesehen, so gilt:
\begin{equation}
  I = \int r^2dm.
\end{equation}
Liegt die Drehachse nicht in einer Schwerpunktachse des Körpers, sondern parallel dazu,
so kann das Trägheitsmoment mithilfe des Steinerschen Satzes bestimmt werden.
Daraus folgt
\begin{equation}
  I = I_\su{s} + m \cdot a^2
\end{equation}
mit $I_\su{s}$ als Trägheitsmoment der Drehachse durch den Schwerpunkt und $a$ als
Abstand der Drehachse zur Schwerpunktachse.

Das Drehmoment $M$ lässt sich über
\begin{equation}
  \vec{M} = \vec{F} \times \vec{r}
\end{equation}
ermitteln. Dabei wirkt die Kraft $\vec{F}$ an einem Punkt mit Abstand $\vec{r}$
zur Drehachse.
Ist das System Schwingungsfähig, so liegt ein Auslenkungswinkel $\varphi$ und eine
Periodendauer $T$ vor. Für eine harmonische Schwingung mit kleiner Auslenkung $\varphi$
folgt
\begin{equation}
  T = 2\pi\sqrt{\frac{I}{D}}.
\end{equation}
$D$ ist dabei die Winkelrichtgröße, welche sich über
\begin{equation}
  D = \frac{M}{\varphi}
\end{equation}
bestimmen lässt.
Für die statische Messmethode wird das Objekt um den Winkel $\varphi$ ausgelenkt
und die Kraft $F$ wird senkrecht zum Bahnradius $r$ gemessen.
Bei der dynamischen Messmethode wird auf die Schwingungsdauer $T$ zurückgegriffen.
