Bei diesem Experiment lassen sich Abweichungen keineswegs ohne große Mühen vermeiden.
Zum einen haben wir die Zeiten der Perioden per Hand bestimmt und da wir Zeiten unter einer
Sekunde gemessen haben ist dies natürlich mit einem Fehler verbunden.
Zudem werden sämtliche Reibungseffekte wie die Luftreibung bzw. die Reibung der Bauteile
vernachlässigt.

In der ersten Messung ergeben sich damit Abweichungen von  $57.7\,\%$ für den ersten Zylinder
und $49.7\,\%$ für den zweiten Zylinder. Dies ist im Rahmen der gesamten Messungenauigkeiten
somit in Ordnung.

Für die Bestimmung des Trägheitsmoments der Puppe ergibt sich eine Abweichung von $24.4\,\%$ bei der
ersten Stellung und eine Abweichung von $33.4\,\%$ bei der zweiten Stellung.
Hier muss jedoch berücksichtig werden, dass bei der Puppe sehr starke Näherungen vorgenommen
wurden, da die Puppe nicht wirklich aus Zylindern besteht. Dennoch hält sich die Abweichung
in Grenzen und liefert eine sehr gute Näherung für das Trägheitsmoment der Puppe.
