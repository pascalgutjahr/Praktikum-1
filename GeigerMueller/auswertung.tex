\subsection{Impulsmessung}
Um die Charakteristik des Geiger-Müller-Zählrohrs aufzunehmen wird die
$\beta$-Strahlungsquelle vor das Zählrohr gespannt.
\begin{table}[H]
  \centering
  \caption{Impulsraten für verschiedene Spannungen.}
  \begin{tabular}{ccc||ccc}
    \toprule
    \mc{1}{c}{Spannung}&\mc{1}{c}{Impulsrate}&\mc{1}{c||}{Strom}&\mc{1}{c}{Spannung}&\mc{1}{c}{Impulsrate}&\mc{1}{c}{Strom} \\
    $\su{U}\,/\Volt$&$\su{N}\,/\sek$&$\su{I}\,/\mcA$&$\su{U}\,/\Volt$&$\su{N}\,/\sek$&$\su{I}\,/\mcA$ \\
    \midrule
    300 & \hrulefill & \hrulefill & 510 & 323.8 \pm 18.0 & 0.8 \\
    310 & 298.6 \pm 17.3 & 0.1 & 520 & 323.3 \pm 18.0 & 1.0 \\
    320 & 308.1 \pm 17.6 & 0.1 & 530 & 324.9 \pm 18.0 & 1.0 \\
    330 & 313.6 \pm 17.7 & 0.2 & 540 & 324.6 \pm 18.0 & 1.0 \\
    340 & 321.1 \pm 17.9 & 0.2 & 550 & 326.4 \pm 18.1 & 1.1 \\
    350 & 318.0 \pm 17.8 & 0.2 & 560 & 323.7 \pm 18.0 & 1.1 \\
    360 & 312.6 \pm 17.7 & 0.3 & 570 & 327.3 \pm 18.1 & 1.2 \\
    370 & 317.2 \pm 17.8 & 0.3 & 580 & 328.1 \pm 18.1 & 1.2 \\
    380 & 321.4 \pm 17.9 & 0.4 & 590 & 328.6 \pm 18.1 & 1.3 \\
    390 & 321.8 \pm 17.9 & 0.4 & 600 & 330.3 \pm 18.2 & 1.3 \\
    400 & 320.8 \pm 17.9 & 0.4 & 610 & 335.6 \pm 18.3 & 1.3 \\
    410 & 322.2 \pm 17.9 & 0.5 & 620 & 332.8 \pm 18.2 & 1.4 \\
    420 & 326.5 \pm 18.1 & 0.6 & 630 & 336.9 \pm 18.4 & 1.4 \\
    430 & 327.5 \pm 18.1 & 0.6 & 640 & 335.3 \pm 18.3 & 1.5 \\
    440 & 324.7 \pm 18.0 & 0.6 & 650 & 335.1 \pm 18.3 & 1.5 \\
    450 & 326.6 \pm 18.1 & 0.6 & 660 & 343.3 \pm 18.5 & 1.6 \\
    460 & 324.5 \pm 18.0 & 0.6 & 670 & 343.3 \pm 18.5 & 1.6 \\
    470 & 328.1 \pm 18.1 & 0.8 & 680 & 344.3 \pm 18.6 & 1.6 \\
    480 & 327.1 \pm 18.1 & 0.8 & 690 & 343.5 \pm 18.5 & 1.7 \\
    490 & 324.0 \pm 18.0 & 0.8 & 700 & 352.3 \pm 18.8 & 1.8 \\
    500 & 326.0 \pm 18.1 & 0.8 & \hrulefill&\hrulefill&\hrulefill \\
    \bottomrule
  \end{tabular}
  \label{.}
\end{table}
Mit diesen Werten lässt sich die Charakteristik des Zählrohrs wie in Abbildung
\ref{fig:char} darstellen, indem die Impulsrate gegen die Spannung aufgetragen
wird.
\begin{figure}[H]
  \includegraphics[width=0.8\textwidth]{bilder/charakteristik.pdf}
  \caption{Charakteristik des Zählrohrs}
  \label{fig:char}
\end{figure}
Die Steigung des Plateus lässt sich mit der linearen Regression mit den Parametern
\begin{align*}
  m &= 0.0123 \pm 0.009 \\
  n &= 318.8 \pm 4.1
\end{align*}
darstellen. Hierbei gibt $m$ die Steigung des Plateaus an. Da die Steigung
üblicherweise in $\%$ pro $100\Volt$ angegeben wird, erhält man für die
Plateau-Steigung
\begin{equation*}
  m= (12.3\pm0.9)\%
\end{equation*}
auf $100\Volt$

\subsection{Primär- und Nachladeimpuls}
\begin{table}[H]
  \centering
  \caption{}
  \begin{tabular}{ccc}
    \toprule
    \mc{1}{c}{Spannung}&\mc{1}{c}{Erholzeit}&\mc{1}{c}{Nachentladungsimpuls}\\
    $U\,/\Volt$&$t\,/\ms$&$t\,/\ms$\\
    \midrule
    350 &  70 & 50 \\
    400 &  90 & 60 \\
    450 & 100 & 70 \\
    500 & 120 & 80 \\
    550 & 150 & 70 \\
    600 & 160 & 70 \\
    650 & 160 & 65 \\
    700 & 160 & 65 \\
    \bottomrule
  \end{tabular}
  \label{}
\end{table}
\subsection{Messung der Totzeit}
\subsubsection{Bestimmung durch Oszilloskop}
\begin{table}[H]
  \centering
  \caption{Werte zur Bestimmung der Totzeit.}
  \begin{tabular}{cc}
    \toprule
    \mc{1}{c}{Spannung}&\mc{1}{c}{Totzeit}\\
    $U\,/\Volt$&$t\,/\ms$\\
    \midrule
    300 & \hrulefill \\
    350 & 100 \\
    400 & 110 \\
    450 & 130 \\
    500 & 135 \\
    550 & 145 \\
    600 & 140 \\
    650 & 150 \\
    700 & 150 \\
    \bottomrule
  \end{tabular}
  \label{tab:tot}

\end{table}
\subsubsection{Zwei-Quellen-Methode}
\begin{align*}
  N_\text{1} &= (621.32\pm\,24.93)\,\si[per-mode=fraction]{\per\second} \\
  N_\text{2} &= (15.58\pm\,3.95)\,\si[per-mode=fraction]{\per\second} \\
  N_\text{1+2} &= (634\pm\,25.18)\,\si[per-mode=fraction]{\per\second}
\end{align*}
Die Totzeit errechnet sich nach Gleichung \eqref{eqn:tot} mit einem
Fehler von
\begin{equation}
  \Delta T = \sqrt{\left(\frac{N_\text{1+2} - N_\text{2}}{2\,N_\text{1}^2
N_\text{2}}\right)^2\cdot \sigma_\symup{N_1}^2 +
\left(\frac{N_\text{1+2} -
   N_\text{1}}{2\,N_\text{1} N_\text{2}^2}\right)^2\cdot
\sigma_\symup{N_2}^2 +
\left(-\frac{1}{2\,N_\text{1}N_\text{2}}\right)^2\cdot
\sigma_\symup{N_\text{1+2}}^2}
\end{equation}
zu
\begin{equation*}
  T \approx (0.14 \pm\,1.83)\,\si{\milli\second}.
\end{equation*}

\begin{table}[H]
  \centering
  \caption{}
  \begin{tabular}{cccc}
    \toprule
    \mc{1}{c}{Spannung}&\mc{1}{c}{Strom}&\mc{1}{c}{Impulsrate}&\mc{1}{c}{Ladung} \\
    $U\,/\Volt$&$I\,/\mcA$&$N\,/60\sek$&$Q\,/e\cdot10^{16}\si{\coulomb}\cdot10^{16}$\\
    \midrule
    350 & 0.2 & 19079 & 0.39 \\
    400 & 0.4 & 19245 & 0.78 \\
    450 & 0.6 & 19597 & 1.14 \\
    500 & 0.8 & 19559 & 1.53 \\
    550 & 1.1 & 19583 & 2.10 \\
    600 & 1.3 & 19816 & 2.46 \\
    650 & 1.5 & 20105 & 2.79 \\
    700 & 1.8 & 21140 & 3.19 \\
    \bottomrule
  \end{tabular}
  \label{}
\end{table}
