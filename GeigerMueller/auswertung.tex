\subsection{Charakteristik de Zählrohrs}
Um die Charakteristik des Geiger-Müller-Zählrohrs aufzunehmen wird die
$\beta$-Strahlungsquelle vor das Zählrohr gespannt.
\begin{table}[H]
  \centering
  \begin{tabular}{ccc||ccc}
    \toprule
    \mc{1}{c}{Spannung}&\mc{1}{c}{Impulsrate}&\mc{1}{c||}{Strom}&\mc{1}{c}{Spannung}&\mc{1}{c}{Impulsrate}&\mc{1}{c}{Strom} \\
    $\su{U}\,/\Volt$&$\su{N}\,/\si[per-mode=fraction]{\per\second}
    $&$\su{I}\,/\mcA$&$\su{U}\,/\Volt$&$\su{N}\,/\si[per-mode=fraction]{\per\second}$&$\su{I}\,/\mcA$ \\
    \midrule
    300 & \hrulefill & \hrulefill & 510 & 323.8 \pm 18.0 & 0.8 \\
    310 & 298.6 \pm 17.3 & 0.1 & 520 & 323.3 \pm 18.0 & 1.0 \\
    320 & 308.1 \pm 17.6 & 0.1 & 530 & 324.9 \pm 18.0 & 1.0 \\
    330 & 313.6 \pm 17.7 & 0.2 & 540 & 324.6 \pm 18.0 & 1.0 \\
    340 & 321.1 \pm 17.9 & 0.2 & 550 & 326.4 \pm 18.1 & 1.1 \\
    350 & 318.0 \pm 17.8 & 0.2 & 560 & 323.7 \pm 18.0 & 1.1 \\
    360 & 312.6 \pm 17.7 & 0.3 & 570 & 327.3 \pm 18.1 & 1.2 \\
    370 & 317.2 \pm 17.8 & 0.3 & 580 & 328.1 \pm 18.1 & 1.2 \\
    380 & 321.4 \pm 17.9 & 0.4 & 590 & 328.6 \pm 18.1 & 1.3 \\
    390 & 321.8 \pm 17.9 & 0.4 & 600 & 330.3 \pm 18.2 & 1.3 \\
    400 & 320.8 \pm 17.9 & 0.4 & 610 & 335.6 \pm 18.3 & 1.3 \\
    410 & 322.2 \pm 17.9 & 0.5 & 620 & 332.8 \pm 18.2 & 1.4 \\
    420 & 326.5 \pm 18.1 & 0.6 & 630 & 336.9 \pm 18.4 & 1.4 \\
    430 & 327.5 \pm 18.1 & 0.6 & 640 & 335.3 \pm 18.3 & 1.5 \\
    440 & 324.7 \pm 18.0 & 0.6 & 650 & 335.1 \pm 18.3 & 1.5 \\
    450 & 326.6 \pm 18.1 & 0.6 & 660 & 343.3 \pm 18.5 & 1.6 \\
    460 & 324.5 \pm 18.0 & 0.6 & 670 & 343.3 \pm 18.5 & 1.6 \\
    470 & 328.1 \pm 18.1 & 0.8 & 680 & 344.3 \pm 18.6 & 1.6 \\
    480 & 327.1 \pm 18.1 & 0.8 & 690 & 343.5 \pm 18.5 & 1.7 \\
    490 & 324.0 \pm 18.0 & 0.8 & 700 & 352.3 \pm 18.8 & 1.8 \\
    500 & 326.0 \pm 18.1 & 0.8 & \hrulefill&\hrulefill&\hrulefill \\
    \bottomrule
  \end{tabular}
  \caption{Impulsraten für verschiedene Spannungen.}
  \label{tab:data}
\end{table}
Mit diesen Werten lässt sich die Charakteristik des Zählrohrs wie in Abbildung
\ref{fig:char} darstellen, indem die Impulsrate gegen die Spannung aufgetragen
wird.
\begin{figure}[H]
  \includegraphics[width=0.8\textwidth]{bilder/charakteristik.pdf}
  \caption{Charakteristik des Zählrohrs.}
  \label{fig:char}
\end{figure}
Die Steigung des Plateaus lässt sich mit der linearen Regression mit den Parametern
\begin{align*}
  m &= 0.0123 \pm 0.009 \\
  n &= 318.8 \pm 4.1
\end{align*}
darstellen. Hierbei gibt $m$ die Steigung des Plateaus an. Da die Steigung
üblicherweise in $\%$ pro $100\Volt$ angegeben wird, erhält man für die
Plateau-Steigung
\begin{equation*}
  m= (1.23\pm0.09)\%.
\end{equation*}
Das Plateau beginnt bei $380\Volt$ und endet bei $560\Volt$. Für die Breite
$b_\su{Plateau}$ ergibt sich somit ein Wert von
\begin{equation*}
  b_\su{Plateau} = 180\Volt.
\end{equation*}

\subsection{Primär- und Nachentladeimpuls}
Die Nachentladungsimpulse und die Erholzeit werden auf dem Oszilloskop abgelesen
und in Tabelle \ref{tab:nach} aufgeführt.
\begin{table}[H]
  \centering
  \begin{tabular}{ccc}
    \toprule
    \mc{1}{c}{Spannung}&\mc{1}{c}{Erholzeit}&\mc{1}{c}{Nachentladungsimpuls}\\
    $U\,/\Volt$&$t\,/\ms$&$t\,/\ms$\\
    \midrule
    350 &  70 & 50 \\
    400 &  90 & 60 \\
    450 & 100 & 70 \\
    500 & 120 & 80 \\
    550 & 150 & 70 \\
    600 & 160 & 70 \\
    650 & 160 & 65 \\
    700 & 160 & 65 \\
    \bottomrule
  \end{tabular}
  \caption{Erholzeit und Nachentladungsimpuls für ausgewählte Spannungen.}
  \label{tab:nach}
\end{table}
Für die Nachentladung wird die Laufzeit vom Beginn des Sekundärpeaks bis zum Ende
des Sekundärpeaks gemessen.
Bei der Erholzeit wird das Ende des Primär-Peaks bis zum Ende des Sekundär-Peaks
gemessen.
\subsection{Messung der Totzeit}
Um die Totzeit zu bestimmen, stehen 2 Methoden zur Verfügung. Zum einen die
Bestimmung mittels Oszilloskop und zum anderen mit der Zwei-Quellen-Methode.
\subsubsection{Bestimmung durch Oszilloskop}
Bei der Bestimmung mit dem Oszilloskop wird die Breite des Primär-Peaks abgelesen.
Die jeweiligen Werte für die Totzeit finden sich in Tabelle \ref{tab:tot} wieder.
\begin{table}[H]
  \centering
  \begin{tabular}{cc}
    \toprule
    \mc{1}{c}{Spannung}&\mc{1}{c}{Totzeit}\\
    $U\,/\Volt$&$t\,/\ms$\\
    \midrule
    300 & \hrulefill \\
    350 & 100 \\
    400 & 110 \\
    450 & 130 \\
    500 & 135 \\
    550 & 145 \\
    600 & 140 \\
    650 & 150 \\
    700 & 150 \\
    \bottomrule
  \end{tabular}
  \caption{Werte zur Bestimmung der Totzeit.}
  \label{tab:tot}
\end{table}
\subsubsection{Zwei-Quellen-Methode}
Bei der Messung mit der Zwei-Quellen-Methode wird die Strahlintensität von
Quelle 1, Quelle 2 und beiden Quellen zusammen gemessen. Hieraus ergibt sich eine
Strahlungsintensität von
\begin{align*}
  N_\text{1} &= (621.32\pm\,24.93)\,\si[per-mode=fraction]{\per\second} \\
  N_\text{2} &= (15.58\pm\,3.95)\,\si[per-mode=fraction]{\per\second} \\
  N_\text{1+2} &= (634\pm\,25.18)\,\si[per-mode=fraction]{\per\second}
\end{align*}
Die Totzeit errechnet dann sich nach Gleichung \eqref{eqn:tot} mit einem
Fehler von
\begin{equation}
  \Delta T = \sqrt{\left(\frac{N_\text{1+2} - N_\text{2}}{2\,N_\text{1}^2 N_\text{2}}\right)^2\cdot \sigma_\symup{N_1}^2 +
  \left(\frac{N_\text{1+2} -
   N_\text{1}}{2\,N_\text{1} N_\text{2}^2}\right)^2\cdot
  \sigma_\symup{N_2}^2 +
  \left(-\frac{1}{2\,N_\text{1}N_\text{2}}\right)^2\cdot
  \sigma_\symup{N_\text{1+2}}^2}
\end{equation}
zu
\begin{equation*}
  T \approx (0.15 \pm\,1.83)\,\si{\milli\second}.
\end{equation*}
\subsection{Bestimmung der freigesetzten Ladung}
Abschließend wird die freigesetzte Ladung $Q$ mit Formel \eqref{eqn:lad}
für ausgewählte Spannungen $U$ bestimmt.Diese befinden sich in Tabelle \ref{tab:lad}.
Die Werte für den Strom $I$ und
die Impulsrate $N$ finden sich in Tabelle \ref{tab:data}. Die Teilchenzahl $Z$
entspricht hier der Impulsrate, da für die gesamte Messung $\Delta t=60\sek$ gilt.
\begin{table}[H]
  \centering
  \begin{tabular}{cccc}
    \toprule
    \mc{1}{c}{Spannung}&\mc{1}{c}{Strom}&\mc{1}{c}{Teilchenzahl}&\mc{1}{c}{Ladung} \\
    $U\,/\Volt$&$I\,/\mcA$&Z&$Q\,/e\cdot10^{16}\si{\coulomb}$\\
    \midrule
    350 & 0.2 & 19079 & 0.39 \pm 0.003 \\
    400 & 0.4 & 19245 & 0.78 \pm 0.006 \\
    450 & 0.6 & 19597 & 1.14 \pm 0.008 \\
    500 & 0.8 & 19559 & 1.53 \pm 0.010 \\
    550 & 1.1 & 19583 & 2.10 \pm 0.020 \\
    600 & 1.3 & 19816 & 2.46 \pm 0.020 \\
    650 & 1.5 & 20105 & 2.79 \pm 0.020 \\
    700 & 1.8 & 21140 & 3.19 \pm 0.020 \\
    \bottomrule
  \end{tabular}
  \caption{Ladungen für verschiedene Spannungen.}
  \label{tab:lad}
\end{table}
Der Fehler Der Ladung $Q$ ergibt sich mit derGauß'schen Fehlerfortpflanzung
\begin{equation*}
  \Delta Q = \sqrt{\left(\frac{\su{It}}{\su{Z}^2})^2\cdot\sigma_\su{Z}^2},
\end{equation*}'
ist jedoch vernachlässigbar klein.
