Es ist relativ gut gelungen die Charakteristik des Zählrohrs zu ermitteln.
Die Plateausteigung von $1.23\%$ ist nur sehr gering und spricht somit für
ein gut arbeitendes Zählrohr. Die Plateaubreite von $180\Volt$ entspricht
ungefähr der Breite aus der Theoriekurve \ref{fig:bereiche.png}. Hierbei
ist jedoch auffällig, dass das Plateau nicht geradlinig ansteigt, sondern
nach einem kleinen Peak erst einmal abfällt. Dies liegt daran, dass die
Messwerte die ganze Zeit ein wenig hin- und herschwanken.

Bei der Bestimmung der Totzeit mithilfe des Oszilloskops wird die Breite des Primärpeaks gemessen, jedoch beginnt der Graph nicht auf Höhe der x-Achse, sondern erst ein wenig
später. Wir haben ab Beginn des Graphen gemessen, in der Realität führt
dies jedoch dazu, dass die Totzeit etwas größer ist.
