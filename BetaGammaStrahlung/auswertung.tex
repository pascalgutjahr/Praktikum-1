% Nulleffekt 1.01 1100s
% gamma-Strahler 131,62 100s
\subsection{\texorpdfstring{Absorption von $\gamma$}{(Absorption von gamma)}-Strahlung}
\begin{wraptable}{r}{5cm}
  \centering
  \caption{Daten zu Eisen und Blei.}
  \begin{tabular}{cccc}
    \toprule
    & $M\,/\,\si{\gram\per\mol}$ & $\rho\,/\DichteSI$ & z \\
    \midrule
    Blei  & 3,44 \cite{Mpb} & 11,340 \cite{dichte} & 82
    Eisen & 0,93 \cite{Mfe }&  7,860 \cite{dichte} & 26
  \end{tabular}
  \label{tab:werte}
\end{wraptable}
Zunächst wird mit den Formeln \eqref{eqn:sigma} und \eqref{eqn:mu} der
Wirkungsquerschnitt, beziehungsweise der Absorptionskoeffizient von den
verwendeten Absorbern berechnet. Als Absorber werden Eisen und Blei verwendet.
Zudem werden die molekulare Masse $M$, die Ordnungszahl $z$ und die Dichte $\rho$
der Absorber benötigt.
Diese Werte befinden sich in Tabelle \ref{tab:werte}.
Somit ergeben sich die Parameter
\begin{equation*}
  \sigma = 2,77\,\cdot10^{-29}\qm
\end{equation*}
mit $\epsilon=1,295$ und daraus folgend
\begin{align*}
  \mu_\su{Pb} &= 2,01\,\cdot 10^{27}, \\
  \mu_\su{Fe} &= 1,64\,\cdot 10^{27}.
\end{align*}
Der Fehler auf den Absorptionskoeffizienten berechnet sich mit
\begin{equation*}
  \Delta\mu = \sqrt{\lf(\frac{z\su{N_L}\rho}{M^2}\rt)^2\cdot\sigma_\su{M}^2},
\end{equation*}
ist jedoch vernachlässigbar klein.
Anschließend werden diese Werte im Experiment überprüft. Tabelle \ref{tab:pb}
zeigt die verschiedenen Zählraten für die beiden Absorber bei verschiedenen Dicken.
\begin{table}[H]
  \centering
  \caption{Zählrate von Blei- und Eisen-Absorbern bei verschiedenen Dicken.}
  \begin{tabular}{cccc}
    \toprule
    \mc{2}{c}{Blei}&\mc{2}{c}{Eisen} \\
    $\su{D}\,/\cm$ & $\su{N}\,/\sek$ & $\su{D}\,/\cm$ & $\su{N}\,/\sek$ \\
    \midrule
    0,10 & 113,38\,\pm\,0,200 & 0,5 & 101,80\,\pm\,0,200 \\
    0,40 &  92,90\,\pm\,0,090 & 1,0 &  75,34\,\pm\,0,100 \\
    1,03 &  43,87\,\pm\,0,040 & 1,5 &  59,16\,\pm\,0,060 \\
    1,30 &  34,30\,\pm\,0,030 & 2,0 &  44,76\,\pm\,0,040 \\
    2,01 &  15,10\,\pm\,0,010 & 2,5 &  35,29\,\pm\,0,020 \\
    3,06 &   5,61\,\pm\,0,004 & 3,0 &  28,24\,\pm\,0,020 \\
    3,45 &   4,62\,\pm\,0,003 & 3,5 &  21,56\,\pm\,0,010 \\
    4,06 &   2,58\,\pm\,0,002 & 4,0 &  17,93\,\pm\,0,010 \\
    4,56 &   2,12\,\pm\,0,002 & 4,5 &  13,71\,\pm\,0,008 \\
    5,10 &   1,68\,\pm\,0,001 & 5,0 &  10,68\,\pm\,0,006 \\
    \bottomrule
  \end{tabular}
  \label{tab:pb}
\end{table}
Bei steigender Dicke wird die Messzeit $t=(60-1000)\sek$ erhöht. Die gemessenen
Counts sind bereits auf eine Sekunde normiert.
Um den Absorptionskoeffizient graphisch darzustellen, wird der Nulleffekt $N_0=1,01\,\si{\per\second}$
von den gemessenen Zählraten subtrahiert und durch $N_\su{ohne}=131,62\,\si{\per\second}$
dividiert. Hierbei entspricht $N_\su{ohne}$ der Messung ohne Absorber.
Diese Daten werden dann logarithmisch gegen die Absorberdicke aufgetragen.
Dies ist in den Abbildungen \ref{fig:pb} und \ref{fig:fe} dargestellt.
\begin{figure}[H]
  \centering
  \begin{subfigure}{0.48\textwidth}
    \centering
    \includegraphics[width=5cm]{bilder/gammaPB.pdf}
    \caption{Absorptionskurve von Blei.}
    \label{}
  \end{subfigure}
  \begin{subfigure}{0.48\textwidth}
    \centering
    \includegraphics[width=5cm]{bilder/gammaFE.pdf}
    \caption{Absorptionskurve von Eisen}
    \label{fig:fe}
  \end{subfigure}
  \caption{Absorptionskurven der beiden Absorbermaterialien}
  \label{fig:abs}
\end{figure}
Die Parameter der Regression betragen
\begin{align*}
  a &= (-0,89\,\pm\,0,04)\,\si{\per\centi\meter}, \\
  b &= (-0,2\,\pm\,0,1)
\end{align*}
für Blei und für Eisen
\begin{align*}
  a &= (-0,492\,\pm\,0,006)\,\si{\per\centi\meter}, \\
  b &= (-0,06\,\pm\,0,02).
\end{align*}
Der Absorptionskoeffizient ergibt sich aus der Steigung $a$ nach Formel \eqref{eqn:steig}
und beträgt
\begin{align*}
  \mu_\su{Pb} &= (89\,\pm\,4)\,\si{\per\meter}, \\
  \mu_\su{Fe} &= (49,2\,\pm\,0,6)\,\si{\per\meter}.
\end{align*}
\subsection{Absorption \texorpdfstring{$\beta$}{beta}-Strahlung}

Die zur Auswertung verwendeten Daten sind in Tabelle \ref{tab:mess}
aufgeführt.

\begin{table}[H]
 \centering
 \begin{tabular}{cccc}
   \toprule
   \multicolumn{1}{c}{Laufzeit} & \multicolumn{1}{c}{Absorberdicke} &
\multicolumn{1}{c}{Impulse} & \multicolumn{1}{c}{Impulsrate} \\
   {$t\,/\,\si{\second}$} & {$d\,/\,\si{\micro\meter}$} & {$N$} &
{$\frac{N}{t}\,/\,\si[per-mode=fraction]{\per\second}$} \\
   \midrule
     100  & 102\,\pm\,1   & 4286 & 42,86\,\pm\,0,07 \\
     200  & 126\,\pm\,1   & 4505 & 22,53\,\pm\,0,02 \\
     300  & 153\,\pm\,0.5 & 3003 & 10,01\,\pm\,0,01 \\
     400  & 160\,\pm\,1   & 2195 & 5,49\,\pm\,0,01 \\
     500  & 200\,\pm\,1   & 1017 & 2,03\,\pm\,0,001 \\
     600  & 253\,\pm\,1   & 478  & 0,79\,\pm\,0,001 \\
     700  & 302\,\pm\,1   & 395  & 0,56\,\pm\,0,001 \\
     800  & 338\,\pm\,5   & 484  & 0,61\,\pm\,0,001 \\
     900  & 400\,\pm\,1   & 494  & 0,55\,\pm\,0,001 \\
     1000 & 444\,\pm\,2   & 533  & 0,53\,\pm\,0,001 \\
 \bottomrule
 \end{tabular}
 \caption{Messdaten zur Auswertung.}
 \label{tab:mess}
\end{table}

Zur Auswertung der Absorptionskurve wird die Dicke des Absorbers gegen die
logarithmische Zählrate dargestellt. Dazu wird von den Zählraten aus
Tabelle \ref{tab:mess} die Nullmessung subtrahiert und durch die Messung
ohne Absorber dividiert. Zur Bestimmung der maximalen Reichweite
$R_\text{max}$ wird der Bereich oberhalb und unterhalb des Graphen linear
nach

\begin{equation}
 y_\text{1, 2} = m_\text{1, 2} \cdot d + b_\text{1, 2}
\end{equation}
dargestellt.

Die Koeffizienten ergeben mit Python 3.5.2
\begin{align*}
 m_\text{1} &=
(-0,03\,\pm\,0,001)\,\si[per-mode=fraction]{\per\micro\meter}\\
 m_\text{2} &=
(-0,01\,\pm\,0,005)\,\si[per-mode=fraction]{\per\micro\meter} \\
 b_\text{1} &= 0,94\,\pm\,0,23\\
 b_\text{2} &= -5,86\,\pm\,1,86.
\end{align*}

Danach lässt sich nach $R_\text{max} = \frac{b_\text{2} -
b_\text{1}}{m_\text{1} - m_\text{2}}$ die Reichweite zu
\begin{equation}
 R_\text{max} = (287,87\,\pm\,100.96)\,\si{\micro\meter}
\end{equation}

berechnen. Mit der maximalen Reichweite folgt nach Gleichung
\eqref{eqn:emax} für die maximale Energie mit einem Fehler nach

\begin{equation}
 \Delta E_\text{max} = \sqrt{\left(\frac{\partial E_\text{max}}{\partial
R_\text{max}}\right)^2 \cdot \sigma_\symup{R_\symup{max}}^2} =
\frac{1,92 R_\text{max} + 0,2112}{\sqrt{R_\text{max}^2 + 0,22
R_\text{max}}}
\end{equation}

der Wert zu

\begin{equation}
 E_\text{max} = (0.015\,\pm\,0.003)\,\si{\mega\electronvolt}.
\end{equation}
\par\bigskip

Die Nullmessung zu Beginn mit $t_\text{0} = 1100\,\si{\second}$ beträgt
auf $1\,\si{\second}$ normiert $N_\text{0} =
0.52\,\si[per-mode=fraction]{\per\second}$. Die Impulsrate ohne Absorber
beträgt $N_\text{o. Absorp} =
554.29\,\si[per-mode=fraction]{\per\second}$.

Der Graph ist in Abbildung \ref{fig:absorp} dargestellt.

\begin{figure}[H]
 \centering
 \includegraphics[width = 0.8\textwidth]{bilder/betaAl.pdf}
 \caption{Absorptionskurve des $\beta$ - Strahlers \ce{^{99}Tc}.}
 \label{fig:absorp}
\end{figure}
