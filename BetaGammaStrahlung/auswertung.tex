% Nulleffekt 1.01 1100s
% gamma-Strahler 131,62 100s
\subsection{\texorpdfstring{Absorption von $\gamma$}{(Absorption von gamma)}-Strahlung}
Zunächst wird mit den Formeln \eqref{eqn:sigma} und \eqref{eqn:mu} der
Wirkungsquerschnitt, beziehungsweise der Absorptionskoeffizient von den
verwendeten Absorbern berechnet. Als Absorber werden Eisen und Blei verwendet.
Somit ergeben sich die Parameter
\begin{equation*}
  \sigma = 2,77\,\cdot10^{-29}\qm
\end{equation*}
und daraus folgend
\begin{align*}
  \mu_\su{Pb} &= 2,01\,\cdot 10^{27} \\
  \mu_\su{Fe} &= 1,64\,\cdot 10^{27}.
\end{align*}

Der Fehler auf den Absorptionskoeffizienten berechnet sich mit
\begin{equation*}
  \Delta\mu = \sqrt{\lf(\frac{z\su{N_L}\rho}{M^2}\rt)^2\cdot\sigma_\su{M}^2},
\end{equation*}
ist jedoch vernachlässigbar klein.
\begin{table}[H]
  \centering
  \caption{Zählrate von Blei- und Eisen-Absorbern bei verschiedenen Dicken.}
  \begin{tabular}{cccc}
    \toprule
    \mc{2}{c}{Blei}&\mc{2}{c}{Eisen} \\
    $\su{D}\,/\cm$ & $\su{N}\,/\sek$ & $\su{D}\,/\cm$ & $\su{N}\,/\sek$ \\
    \midrule
    0,10 & 113,38 & 0,5 & 101,80 \\
    0,40 &  92,90 & 1,0 &  75,34 \\
    1,03 &  43,87 & 1,5 &  59,16 \\
    1,30 &  34,30 & 2,0 &  44,76 \\
    2,01 &  15,10 & 2,5 &  35,29 \\
    3,06 &   5,61 & 3,0 &  28,24 \\
    3,45 &   4,62 & 3,5 &  21,56 \\
    4,06 &   2,58 & 4,0 &  17,93 \\
    4,56 &   2,12 & 4,5 &  13,71 \\
    5,10 &   1,68 & 5,0 &  10,68 \\
    \bottomrule
  \end{tabular}
  \label{tab:pb}
\end{table}
