Bei diesem Experiment spielt energiereiche Strahlung eine Rolle, die mit Materie
wechselwirkt. Es wird sowohl $\gamma$-Strahlung in einem Energiebereich von
$60\keV$ bis $1300\keV$, also auch $\beta^{-}$-Strahlung betrachtet. Beide
Strahlungsarten stammen von instabilen Kernen.

Wechselwirkung zwischen den emmitierten Photonen beziehungsweise Elektronen mit
Materie treten nur auf, wenn die Teilchen aufeinander treffen. Dies führt zu einer
Abnahme der Intensität der Strahlung.
Der Wirkungsquerschnitt $\sigma$ wird verwendet, da Materie zu einem großen Teil
aus Freiraum besteht. Hierbei gilt: je größer die Wahrscheinlichkeit ist, das
Teilchen miteinander Wechselwirken, desto größer ist der Wirkungsquerschnitt.

Ausschlaggebend für die Anzahl von Wechselwirkungen ist die Dicke $\su{D}$ des
verwendeten Absorbers. Für $\gamma$-Strahlung gilt ein exponentieller Abfall der
Intensität in Abhängigkeit der Dicke:
\begin{equation}
  N(D) = N_0 \cdot \exp^{-n\sigma D}.
\end{equation}
Hierbei ist $N_0$ die Ausgangsaktivität und $n$ die Anzahl der Teilchen im Absorber.
