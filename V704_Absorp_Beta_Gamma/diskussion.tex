Bei der Bestimmung der Absorptionskoeffizienten von Blei und Eisen ergeben sich die relativen Fehler aus Tabelle \ref{tab:fehler}.
\begin{table}
  \centering
  \caption{Relative Abweichungen der Absorptionskoeffizienten.}
  \begin{tabular}{cccc}
    \toprule
    \mc{1}{c}{Material} & \mc{1}{c}{Literaturwert} & \mc{1}{c}{Messwert} & \mc{1}{c}{Fehler} \\
     & $\mu\,/\,\si[per-mode=fraction]{\per\meter}$ & $\mu\,/\,\si[per-mode=fraction]{\per\meter}$ & \% \\
     \midrule
     Blei  & 69,2 &   89\,\pm\,4   & 28,6 \\
     Eisen & 56,4 & 49,2\,\pm\,0,6 & 12,8 \\
     \bottomrule
  \end{tabular}
  \label{tab:fehler}
\end{table}
Bei dieser Messung wurden die Dicken der Absorber mit einer Schieblehre
bestimmt. Um verschiedene Dicken zu erhalten, wurden dann einzelne Platten
einfach hintereinander gesteckt. Dabei haben wir jedoch die Dicke des
gesamten Blocks erneut gemessen, und nicht die Dicken der einzelnen Platten
addiert. Zwischen den Platten befindet sich jedoch noch ein wenig Luft,
welche dazu führt, dass die wahre Dicke der Platten etwas geringer ist.
Hierbei ist auffällig, dass bei Eisen eine deutlich geringere Abweichung
auftritt. Dies ist im allgemeinen dadurch zu erklären, dass wir möglicherweise
bei Eisen weniger Zwischenraum zwischen den einzelnen Platten hatten, wodurch
die gemessene Dicke der wahren Dicke etwas besser entsprach.

Dennoch ist ganz klar zu erkennen, dass Blei, wie erwartet, den deutlich
höheren Absorptionskoeffizieten besitzt, was mit der massiven Dichte von
Blei zu erklären ist.

Bei der Bestimmung der Reichweite und der Energie der $\beta$ - Strahlung ergeben
sich Werte im Bereich von  $E =15 \,\si{\kilo\electronvolt}$. Dies liegt genau 
im erwarteten Bereich. 
Die Effekte, wie Compton-Effekt, Photo-Effekt oder Paarbildung, sind 
materialabhängig. 
Bei der gegebenen Energie der $\beta$-Strahlung und der gegebenen Materialien tritt häufig 
der Compton-Effekt auf. Jedoch ist auch der Photo-Effekt möglich. 
Bei der $\gamma$-Strahlung ist der Photo-Effekt unwahrscheinlich. Primär
tritt der Compton-Effekt auf und es erfolgen die ersten Paarbildungen.
 
