\subsection{Fadenlänge 50 cm}

Anschließend werden die Messungen für eine Fadenlänge von $50\cm$ wiederholt.
Die aufgenommenen Werte für die Schwingdauern sind in der untenstehenden Tabelle
\ref{tab:50cm} zu sehen.
\begin{table}[H]
  \centering
  \begin{tabular}{c | c c c c c c}
    \toprule
    &Pendel 1 &Pendel 2 & gleichsinnig & gegensinnig &gekoppelt&
    gekoppelt, einzeln \\
    \midrule
    $t_1\,/\sek$ & 6.96 & 6.89 & 6.72 & 6.23 & 18.50 & 6.50 \\
    $t_2\,/\sek$ & 7.03 & 6.83 & 6.90 & 6.40 & 17.06 & 6.55 \\
    $t_3\,/\sek$ & 7.04 & 6.96 & 6.96 & 6.15 & 17.41 & 6.64 \\
    $t_4\,/\sek$ & 7.04 & 6.92 & 6.84 & 6.47 & 17.92 & 6.50 \\
    $t_5\,/\sek$ & 6.93 & 6.96 & 7.00 & 6.30 & 17.78 & 6.36 \\
    $t_6\,/\sek$ & 6.93 & 6.90 & 6.93 & 6.23 & 18.96 & 6.50 \\
    $t_7\,/\sek$ & 7.07 & 6.90 & 6.90 & 6.16 & 17.16 & 6.41 \\
    $t_8\,/\sek$ & 6.93 & 6.93 & 6.93 & 6.36 & 17.10 & 6.53 \\
    $t_9\,/\sek$ & 6.93 & 6.93 & 7.01 & 6.06 & 17.21 & 6.50 \\
    \bottomrule
  \end{tabular}
  \caption{Schwingdauer für 5 Perioden bei einer Fadenlänge von $50\cm$}
  \label{tab:50cm}
\end{table}
Die Werte für die Schwingungsdauern und -frequenzen werden analog berechnet. Somit
erhält man für die Schwingungsdauer die Werte die in Tabelle \ref{tab:aus50} zu sehen
sind.
\begin{table}
  \centering
  \begin{tabular}{c c | c c | c c}
    \toprule
    $T_\su{+, exp}$\,/\sek & $T_\su{+, theo}$\,/\sek & $T_\su{-, exp}$\,/\sek &
    $T_\su{-, theo}$\,/\sek & $T_\su{S, exp}$\,/\sek & $T_\su{S, theo}$\,/\sek \\
    \midrule
    $1.38\pm0.01$ & 1.42 & $1.25\pm0.03$ & 1.40 & $17.68\pm0.64$ & 99.4 \\
    \bottomrule
  \end{tabular}
  \caption{Schwingungsdauern für eine Fadenlänge von $50\cm$}
  \label{tab:aus50}
\end{table}
Die Kopplungskonstante $K$ welche für die Zeiten $T_-$ und $T_+$ benötigt wird,
wurde erneut über die experimentell festgestellten Werte berechnet. Sie beträgt
\begin{equation*}
  K = 0.018
\end{equation*}
Die Ergebnisse für die verschiedenen Frequenzen finden sich in Tabelle \ref{tab:aus50w}
wieder.
\begin{table}
  \centering
  \begin{tabular}{c c | c c | c c}
    \toprule
    $\omega_\su{+, exp}$\,/\sek & $\omega_\su{+, theo}$\,/\sek & $\omega_\su{-, exp}$\,/\sek &
    $\omega_\su{-, theo}$\,/\sek & $\omega_\su{S, exp}$\,/\sek & $\omega_\su{S, theo}$\,/\sek \\
    \midrule
    $4.54\pm0.01$ & 4.43 & $5.02\pm0.01$ & 4.47 & $0.36\pm0.01$ & -0.04 \\
    \bottomrule
  \end{tabular}
  \caption{Schwingungsfrequenzen für eine Fadenlänge von $50\cm$}
  \label{tab:aus50w}
\end{table}
