Zunächst wird eine Fadenlänge von $70\cm$ eingestellt. Um sicher zu gehen, dass
beide Pendel die selbe Fadenlänge haben, wird die Schwingdauer beider Pendel
einzeln gemessen. Hierfür werden 9 mal 5 Schwingungen gemessen und gemittelt.
Der jeweilige Fehler der Messwerte ergibt sich über die Standardabweichung.

Die Werte der Messung sind in Tabelle \ref{tab:70cm} zu sehen. Die Auslenkung
aller Messungen betrug hierbei $5\cm$
\begin{table}[H]
  \centering
  \begin{tabular}{c c c c c c c}
    \toprule
    links, einzeln & rechts, einzeln & gleichsinnig & gegensinnig &gekoppelt&
    gekoppelt, einzeln \\
    \midrule
    $t_1\,/\sek$ & 7.84 & 7.98 & 8.18 & 6.36 & 7.92 & 7.98 \\
    $t_2\,/\sek$ & 7.86 & 8.17 & 8.21 & 6.44 & 7.55 & 8.17 \\
    $t_3\,/\sek$ & 8.12 & 8.18 & 8.07 & 6.44 & 7.75 & 8.18 \\
    $t_4\,/\sek$ & 8.01 & 8.16 & 7.96 & 6.65 & 7.69 & 8.16 \\
    $t_5\,/\sek$ & 7.82 & 8.21 & 8.03 & 6.58 & 8.23 & 8.21 \\
    $t_6\,/\sek$ & 8.00 & 8.15 & 8.06 & 6.55 & 8.43 & 8.15 \\
    $t_7\,/\sek$ & 7.75 & 8.22 & 8.15 & 6.56 & 8.04 & 8.22 \\
    $t_8\,/\sek$ & 7.96 & 8.07 & 7.93 & 6.50 & 8.32 & 8.07 \\
    $t_9\,/\sek$ & 8.06 & 8.07 & 8.06 & 6.73 & 8.03 & 8.17 \\
    \bottomrule
  \end{tabular}
  \caption{Schwingdauer für 5 Perioden bei einer Fadenlänge von $70\cm$}
  \label{tab:70cm}
\end{table}
Um den Schwingungsdauer einer einzigen Perioden zu erhalten, werden die Werte
in der Tabelle mit dem Faktor $\frac{1}{5}$ multipliziert und der Mittelwert
berechnet. Somit erhält man
\begin{equation*}
T_\su{links}=(1.59\pm0.02)\sek
T_\su{rechts}=(1.63\pm0.01)\sek
% T_\su{koppel}=(7.9\pm0.3)\sek
% T_\su{einzel, koppel}=(1.27\pm0.02)\sek
\end{equation*}
für die Schwingdauer der einzelnen Pendel. Nun werden die Pendel mit einer Feder
verbunden und in eine gleichsinnige Schwingung versetzt. Die gemessenen Werte
sind ebenfalls in Tabelle \ref{tab:70cm} zu sehen. Der experimentell bestimmte
Wert beträgt
\begin{equation*}
  T_\su{+, exp}=(1.61\pm0.02)\sek.
\end{equation*}
Dieser wird mit dem theoretischen Wert, welcher sich nach Formel \eqref{eqn:T+}
berechnen lässt verglichen. Der theoretische Wert für die gleichsinnige Schwinung
liegt bei
\begin{equation*}
  T_\su{+, theo}=1.68\sek.
\end{equation*}
Zudem soll die Schwingungsfrequenz miteinander verglichen werden. Während sich
der theoretische Wert nach Formel \eqref{eqn:w+}
berechnen lässt, muss für den experimentellen Wert die Formel
\begin{equation}
  \su{\omega}=\frac{2\pi}{T_\su{+, exp}}
  \label{eqn:wexp}
\end{equation}
angewendet werden. $K$ wird hierbei mit den experimentell ermittelten Werten nach
Formel \eqref{eqn:K} berechnet. Der Fehler auf die Frequenz berehnet sich nach
Gauß'scher Fehlerfortpflanzung mit der Formel
\begin{equation}
  \Delta\omega=\sqrt{\biggl(-\frac{2\pi}{T^2}\biggr)^2\cdot\sigma_\su{T}^2}.
\end{equation}
Die Werte die sich dann berechnen lassen liegen bei
\begin{align*}
  \omega_\su{+,theo} &= 3.74\Hz \\
  \omega_\su{+,exp}  &= (3.90 \pm 0.08)\Hz
\end{align*}
Als nächstes werden die Pendel gegensinnig ausgelenkt. Die Werte für die
Schwingungsdauer und -frequenz werden analog berechnet. Somit erhält man für die
Schwingungsdauer
\begin{align*}
  T_\su{-, exp}=(1.31\pm0.02)\sek \\
  T_\su{-, theo}=1.65\sek
\end{align*}
und für die Schwingungsfrequenz
\begin{algin*}
  \omega_\su{-, exp}  &= (4.80 \pm 0.04)\Hz \\
  \omega_\su{-, theo} &= 3.82\Hz.
\end{align*}

Abschließend wird eine gekoppelte Schwingung erzeugt. Nun wird die
Schwebungsdauer und die Schwebungsfrequenz ermittelt und verglichen. Um die
Schwebungsdauer zu ermitteln, wird die Zeit eines Pendels von Ruhelage zu
Ruhelage gemessen. Auch dieser Wert wird gemittelt, jedoch nicht faktorisiert.
Der theoretische Wert wird mit Formel \eqref{eqn:Ts} berechnet. Somit ergeben
sich die Werte
\begin{align*}
  T_\su{S, exp} =& (7.9\pm0.3)\sek \\
  T_\su{S,theo} =& 83.16\sek
  % 90\% Abweichung
\end{align*}
für die Schwebungsdauer und
\begin{align*}
  \omega_\su{S, exp} &= (0.79 \pm 0.03)\Hz \\
  \omega_\su{S, theo}&= -0.08\Hz
  % negative Frequenz???
\end{align*}
für die Schwebungsfrequenz.
\newpage
