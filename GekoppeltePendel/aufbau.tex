\subsection{Vorbereitung}
"Eine harmonische Schwingung ist eine reine Sinusschwingung.
Diese periodische Bewegung kann als Projektion einer Kreisbewegung gedacht werden."
\,\cite{frankfurt}
Die kleine Winkelnäherung gilt im allgemeinen für Winkel bis ca. $10 \,\si{\degree}$ \,\cite{darmstadt}.
Ein Pendel mit Fadenlänge $l=0.7 \,\si{\meter}$ kann somit um
\begin{equation}
  \sin{10} \cdot 0.7 \,\si{\meter} = 0.12 \,\si{\meter}
\end{equation}
ausgelenkt werden, damit die kleine Winkelnäherung noch gilt.

\subsection{Durchführung}
Zu Beginn sollen die Schwingungsdauern $T_1$ und $T_2$ für zwei frei schwingende
Pendel ermittelt werden. Die Massen der Pendel betragen jeweils $1 \kg$.
Danach werden die Schwingungsdauern $T_+$ und $T_-$ für die gleichphasige und die
gegenphasige Schwingung für mindestens zwei Pendellängen ermittelt. Dazu muss die Feder
zwischen die beiden Pendel eingespannt werden.
Die Länge der Pendel lässt sich dabei durch das hoch bzw. runterschieben der Masse
am jeweiligen Stab einstellen. Dabei misst man die Länge vom Aufhängpunkt bis zum
Mittelpunkt der Masse. Anschließend werden 10 Messungen für eine gekoppelte Schwingung
durchgeführt. Die Schwingungsdauer $T$ und die Schwebungsdauer $T_\su{S}$ werden dabei
für zwei verschiedene Pendellängen bestimmt. Abschließend wird der Kopplungsgrad $K$
bestimmt und die Thoeriewerte werden mit den Praxiswerten verglichen. Alle Zeiten werden
dabei mit einer Stoppuhr aufgenommen.
