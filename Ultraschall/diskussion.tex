Bei der Bestimmung der Durchmesser der Bohrungen ergeben sich die folgenden
Abweichungen, welche in Tabelle \ref{tab:abweichung} zu finden sind.
Dabei werden jeweils die Durchmesser, welche mittels A- und B-Scan bestimmt werden, mit den Durchmessern, welche mit der Schieblehre bestimmt werden, verglichen. Zusätzlich ist zu beachten, dass die Durchmesser des ersten, zweiten und zehnten Lochs mittels Scanverfahren nicht bestimmt werden können, da sich die Laufzeiten gegenseitig überlagen, wodurch sich ein negative Durchmesser ergeben würde.
\begin{table}
  \centering
  \begin{tabular}{c|cc}
    \toprule
    \multicolumn{1}{c|}{Loch} & \multicolumn{2}{c}{relative Abweichung zur Schieblehre$\,/\,\%$}\\
    & {A-Scan} & {B-Scan} \\
  \midrule
  1  &  \hrulefill  & \hrulefill  \\
  2  &  \hrulefill  & \hrulefill  \\
  3  &  40.6        & 29.5  \\
  4  &  47.5        & 29.1  \\
  5  &  65.0        & 39.7  \\
  6  &  87.6        & 58.6  \\
  7  &  83.5        & 54.5  \\
  8  &  71.6        & 43.1  \\
  9  &  78.7        & 65.9  \\
  10 &  \hrulefill  & \hrulefill  \\
  11 &  24.6        & 19.0  \\
  \bottomrule
  \end{tabular}
  \caption{Abweichungen der Löcherdurchmesser von A-Scan und B-Scan zu den Durchmessern, die mit der Schieblehre bestimmt werden.}
  \label{tab:abweichung}
\end{table}
Bei den Scans ist zu beachten, dass der A-Scan mit einer $1\MHz$ und der B-Scan
mit einer $2\MHz$ Sonde durchgeführt wurden. Dadurch lässt sich direkt erklären, warum die Abweichungen der Bohrungen bei dem B-Scan deutlich geringer ausfallen, denn je höher die Frequenz, desto besser ist das Auflösungsvermögen. Dabei nimmt die Eindringtiefe jedoch ab. Dies bestätigt sich auch bei der Überprüfung des Auflösungsvermögens aus Tabelle \ref{tab:auf}. Mit der $1\MHz$ Sonde war es nicht möglich, die Durchmesser der Bohrung zu bestimmen, da sich auch hier die Laufzeiten gegenseitig überlagert haben. Mit der $2\MHz$ Sonde konnten die Maße dagegen relativ gut bestimmt werden - mit einer Abweichung von $26.9\%$ zur ersten Bohrung und $29.4\%$ zur zweiten Bohrung.

Bei der Bestimmung des $HZV$ lassen sich die Peaks im TM-Scan schwierig ablesen, da die Grafik sehr pixelig ist. Zudem ist es nahe zu unmöglich, im gleichen Abstand mit gleicher Intensität per Hand zu pumpen, um einen gleichmäßigen Herzschlag zu simulieren. Zudem wird angenommen, dass $ESV = 0$

da sich die Membran bei der Systole total kontrahiert.
Das Herzzeitvolumen beträgt somit
