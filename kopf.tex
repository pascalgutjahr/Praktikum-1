\documentclass[captions=tableheading,
  bibliography=totoc,
  titlepage=firstiscover
  ]{scrartcl}
\usepackage{scrhack}

% \usepackage[a4paper,top=2.5cm,left=2.5cm,right=2cm,bottom=3cm,bindingoffset=5mm]{geometry}

\usepackage[aux]{rerunfilecheck}

\usepackage{polyglossia}
\setmainlanguage{german}

\usepackage{amsmath}
\usepackage{amssymb}
\usepackage{mathtools}

\usepackage{fontspec}

\usepackage{biblatex}
\addbibresource{lit.bib}  %nach polyglossia

\usepackage[
  math-style=ISO,
  bold-style=ISO,
  sans-style=italic,
  nabla=upright,
  partial=upright,
]{unicode-math}

\usepackage[
  locale=DE,
  separate-uncertainty=true,
  per-mode=symbol-or-fraction,
]{siunitx}

\usepackage[section, below]{placeins}
\usepackage[
  labelfont=bf,        % Tabelle x: Abbildung y: ist jetzt fett
  font=small,          % Schrift etwas kleiner als Dokument
%  width=0.9\textwidth, % maximale Breite einer Caption schmaler
  format=plain,
  indention=1em, % Abbildung sticht links etwas hervor
]{caption}
\usepackage{graphicx}
\usepackage{grffile}
\usepackage{subcaption}

\usepackage{booktabs}
\usepackage{float}
\floatplacement{figure}{htbp}
\floatplacement{table}{htbp}


\usepackage[unicode]{hyperref}
\usepackage{bookmark}
\usepackage{microtype}
