Zunächst wird die mittlere Weglänge der Elektronen mit Formel \eqref{eqn:wbar}
und \eqref{eqn:psat} berechnet. Die Ergebnisse werden dann mit dem Abstand $a$
zwischen Kathode und Beschleunigungselektrode verglichen. Der an der Apparatur
gemessene Abstand beträgt $a=1\,\cdot10^{-2}\mt$
\begin{table}
  \centering
  \begin{tabular}{ccc}
    \toprule
    $T/\Kel$ & $\bar w/\mt$ & $\frac{a}{\bar w}$ \\
    \midrule
     296.15 & $6.38\,\cdot 10^{-3}$ &    1.6  \\
     383.15 & $3.28\,\cdot 10^{-5}$ &  304.9  \\
     433.15 & $4.13\,\cdot 10^{-6}$ & 2421.3  \\
     443.15 & $2.89\,\cdot 10^{-6}$ & 3460.2  \\
    \bottomrule
  \end{tabular}
  \caption{mittlere Weglänge}
  \label{tab:weg}
\end{table}
