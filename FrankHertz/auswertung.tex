Zunächst wird die mittlere Weglänge der Elektronen mit Formel \eqref{eqn:wbar}
und \eqref{eqn:psat} berechnet. Die Ergebnisse werden dann mit dem Abstand $a$
zwischen Kathode und Beschleunigungselektrode verglichen. Der Abstand der
verwendeten Apparatur beträgt $a=1\,\cdot10^{-2}\mt$ \cite{601}.
\begin{table}
  \centering
  \begin{tabular}{ccc}
    \toprule
    $T/\Kel$ & $\bar w/\mt$ & $\frac{a}{\bar w}$ \\
    \midrule
     296.15 & $6.38\,\cdot 10^{-3}$ &    1.6  \\
     383.15 & $3.28\,\cdot 10^{-5}$ &  304.9  \\
     433.15 & $4.13\,\cdot 10^{-6}$ & 2421.3  \\
     443.15 & $2.89\,\cdot 10^{-6}$ & 3460.2  \\
    \bottomrule
  \end{tabular}
  \caption{Die mittlere Weglänge der Elektronen.}
  \label{tab:weg}
\end{table}

Die Frank-Hertz-Kurve des Hg-Dampfes ist in Abb. \ref{fig:kurve} zu sehen.
Dabei wird $U_\su{B}$ gegen $I_\su{A}$ geplottet.
\begin{figure}
  \centering
  \includegraphics[width=0.8\textwidth]{bilder/kurve.jpg}
  \caption{Frank-Hertz-Kurve zu Hg-Dampf.}
  \label{fig:kurve}
\end{figure}
Die Temperatur wird auf $T=170\,\si{\degree}$ geregelt. Um die Anregungsenergien
$\Delta E$
des ersten Zustands von Hg zu bestimmen werden die Abstände der Maxima $\zeta$ in
$\cm$ gemessen. Anhand der Abbildung \ref{fig:kurve} lässt sich ermitteln,
dass $5\Volt = (1.8\pm0.2)\cm$ entsprechen. Die jeweiligen Werte für $\Delta U$
lassen sich dann bestimmen. Mit
\begin{equation}
  \Delta E = e\cdot \Delta U
\end{equation}
ergeben sich die entsprechenden Energien. Alle benötigten Werte sind in Tabelle
\ref{tab:werte} zu finden.
\begin{table}
  \centering
  \begin{tabular}{c c}
    \toprule
    $\zeta \,/\cm$   & $\Delta E \,/\eV$  \\
    \midrule
    1.6 & 4.44 \\
    1.8 & 5.00 \\
    1.6 & 4.44 \\
    2.0 & 5.56 \\
    1.9 & 5.28 \\
    \bottomrule
  \end{tabular}
  \caption{Abstände $\zeta$ und Anregungsenergien $\Delta E$ des ersten Anregungszustands.}
  \label{tab:werte}
\end{table}
Für $\zeta$ ergibt sich dann ein Mittelwert von
\begin{equation*}
  \zeta = (1.8\pm0.2)\cm
\end{equation*}
Für den Mittelwert folgt:
\begin{equation*}
  \Delta E = (4.9 \pm 0.4) \eV.
\end{equation*}
Mit
\begin{equation}
  \lambda = \frac{hc}{\Delta E} \label{eqn:lambda}
\end{equation}
lässt sich die Wellenlänge
\begin{equation*}
  \lambda = (250 \pm 20)\nm
\end{equation*}
mit der Gauß'schen Fehlerfortpflanzung
\begin{equation*}
  \Delta\lambda=\sqrt{\left(\frac{hc}{\Delta E^2}\right)^2\cdot\sigma_\su{\Delta E^2}}
\end{equation*}
berechnen.

Das Kontaktpotential $K$ ergibt sich aus der Differenz der Spannungsdifferenz $\Delta U$
der Anregungsenergien für Quecksilber und der Spannung $U_\su{1.Max}$ bei dem das
erste Maximum auftritt. Es gilt $U_\su{1.Max} = 12.2\Volt$. Mit $\Delta U = 4.9\Volt$
folgt:
\begin{equation}
  K_1 = (7.3\pm0.2)\Volt.
\end{equation}
Der Fehler ergibt sich aus der Fehlerfortpflanzung
\begin{equation}
  \Delta K = \sqrt{\sigma_\su{U_B}}.
\end{equation}

Eine Weitere Möglichkeit zur Bestimmung des Kontaktpotentials ist mithilfe
des Plots \ref{fig:plot} gegeben.
\begin{figure}
  \centering
  \includegraphics[width=0.6\textwidth]{bilder/plt.pdf}
  \caption{Bestimmung des Kontaktpotentials über die Steigung $\sfrac{dI_\su{A}}{dU_\su{A}}$.}
  \label{fig:plot}
\end{figure}
Das Kontaktpotential $K$ ergibt sich dann aus der Beschleunigungsspannung minus
des Maximus des Graphen:
\begin{equation}
  K_2 = U_\su{B} - U_\su{max} = 13\Volt - 9.8\Volt = (3.2\pm0.1)\Volt.
\end{equation}
Der Fehler ergibt sich hier aus der Skalierung der x-Achse.
Damit ergibt sich für den Mittelwert des Kontaktpotentials
\begin{equation}
  \bar{K} = (5 \pm 2) \Volt.
\end{equation}

Zur Bestimmung der Ionisierungsenergie wird der Graph aus Abb. \ref{fig:ion}
verwendet.
\begin{figure}
  \centering
  \includegraphics[width=0.8\textwidth]{bilder/ion.pdf}
  \caption{$I_\su{A}$ gegen $U_\su{B}$ zur Bestimmung der Ionisierungsenergie.}
  \label{fig:ion}
\end{figure}
Dazu wird versucht, eine Ausgleichsgerade eingefügt, welche im Unendlichen starten
soll. Im Idealfall wäre dies eine Asymptote, das ist jedoch bei diesem Graphen nicht
der Fall.
Für die Ausgleichsgerade ergeben sich die Parameter
\begin{align*}
  m&= (0.1\pm 0.005)\frac{\nA}{\Volt} \\
  n&= (-5.0\pm 0.3)\nA
\end{align*}
Der Schnittpunkt mit der x-Achse liegt bei $U_\su{x} = 36.0 \Volt$.
Die Ionisierungsenergie ergibt sich dann, indem das Kontaktpotential $K$ von
dem Schnittpunkt abgezogen wird.
\begin{equation}
  E_\su{ion} = (U_\su{x}-\bar{K}) \cdot e = (31.0\pm2)\eV
\end{equation}
