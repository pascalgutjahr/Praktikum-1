Der Literaturwert für die Anregungsenergie von Hg liegt bei $\Delta E=4.9\eV$
\cite{hg}. Hieraus ergibt sich mit Formel \eqref{eqn:lambda} eine Wellenlänge
von $\lambda=253\nm$. Der gemessene Wert stimmt mit dem Literaturwert überein.
Es wird somit ein Lichtquant emittiert, welcher sich im nicht sichtbaren
UV-Bereich befindet.
In diesem Fall müssen elastische Stöße nicht berücksichtigt werden, da die
Energieübertragung des $e^-$ auf das Atom ausbleibt.

Bei der Bestimmung des Kontaktpotentials $K_2$ ist zu berücksichtigen, dass der
geplottete Graph sehr hohe Ablesefehler beinhaltet, da die stark abfallende Steigung
nur einen Bereich von $1\cm$ eingenommen hat. Innerhalb dieses kleinen Bereichs
ist es somit nahezu unmöglich vernünftige Steigungsdreicke einzuzeichnen. Hinzu
kommt, dass der Schreiber die ganze Zeit kleine Ausschläge nach oben und unten
durchgeführt hat - er hat gezittert.

Die untenstehende Tabelle \ref{tab:rel} zeigt die möglichen Ionisierungsenergien für Quecksilber
und die Abweichung vom gemessenen Wert von $E_\su{ion}$.
\begin{table}[H]
  \centering
  \begin{tabular}{ccc}
    \toprule
    \mc{2}{c}{Ionisierungsenergien}&\mc{1}{c}{relative Abweichung} \\
    Experimentell$\,/\eV$ & Theoretisch$\,/\eV$&\% \\
    \midrule
    mr{3}{*}{31.0} & 10.44 &196.9 \\
                   & 18.76 & 65.2 \\
                   & 34.20 &  9.4 \\
    \bottomrule
  \end{tabular}
  \caption{Ionisierungsenergien von Quecksilber \cite{hg} und relative Abweichung
  des experimentell festgestellten Wert.}
  \label{tab:rel}
\end{table}
Der Gemessene Wert weicht somit extrem von der minimalen Ionisierungsenergie ab.
Hierbei muss es sich um einen systematischen Fehler handeln, der sich nicht
erklären lässt.
