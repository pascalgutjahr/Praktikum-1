Der Literaturwert für die Anregungsenergie von Hg liegt bei $\Delta E=4.9\eV$
\cite{hg}. Hieraus ergibt sich mit Formel \eqref{eqn:lambda} eine Wellenlänge
von $\lambda=253\nm$. Der gemessene Wert stimmt mit dem Literaturwert überein.
Es wird somit ein Lichtquant emittiert, welcher sich im nicht sichtbaren
UV-Bereich befindet.
In diesem Fall müssen elastische Stöße nicht berücksichtigt werden, da die
Energieübertragung des $e^-$ auf das Atom ausbleibt.

Bei der Bestimmung des Kontaktpotentials $K_2$ ist zu berücksichtigen, dass der
geplottete Graph sehr hohe Ablesefehler beinhaltet, da die stark abfallende Steigung
nur einen Bereich von $1\cm$ eingenommen hat. Innerhalb dieses kleinen Bereichs
ist es somit nahezu unmöglich vernünftige Steigungsdreicke einzuzeichnen. Hinzu
kommt, dass der Schreiber die ganze Zeit kleine Ausschläge nach oben und unten
durchgeführt hat - er hat gezittert.

Die maximale Ionisierungsenergie beträgt laut Literatur $E_\su{ion, theo} = 34.2\eV$ \cite{theo}.
Unsere Abweichung zu  $E_\su{ion} = 33.8\eV$ beträgt somit nur $1\,\%$, wodurch
die Überprüfung der Ionisierungsenergie sehr gut gelungen ist.
