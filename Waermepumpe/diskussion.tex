Bei diesem Versuch kommen viele systematsiche Fehler vor, da wir laut Theorie
von einem reversiblen Kreisprozess ausgehen, welcher jedoch nur im Idealfall
möglich ist. Im vorliegenden Versuch war die Wärmeisolierung der Wasserbehälter
jedoch nicht sehr gut und die Behälter waren nicht vollständig verschlossen, wodurch
die Wärme nach oben entweichen konnte, bzw. sich das Wasser dadurch wieder erwärmt hat.
Dadurch sind auch die Drücke niedriger, als eigentlich erwartet.
Des Weiteren sind die Skalen der Barometer sehr ungenau abzulesen, da die Skala vom Hersteller
sehr grob gewählt wurde. Hinzu kommt, dass die Leistung zu Beginn des Versuchs stetig anstieg,
bis sie sich bei einem Wert von ca. 212 Watt eingependelt hat. Durch diesen Verlauf
ergibt sich der Extrem hohe Fehler des Mittelwerts der Leistung von $\sigma_\su{N}
=\pm 50 \,\si{\watt}$.
