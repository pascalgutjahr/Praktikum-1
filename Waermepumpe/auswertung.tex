Tabelle \ref{tab:messwerte} zeigt die im Versuch gemessenen Werte.
$T_\su{1}$ entspricht hierbei der Temperatur des wärmeren Reservoirs, während $T_\su{2}$
die Temperatur des kälteren Reservoires beschreibt. Der Druck $p_\su{a}$ wird dem warmen Reservoir
zugeordnet und $p_\su{b}$ dem kälteren.
\begin{table}[h]
  \centering
  \begin{tabular}{c c c c c c}
    \toprule
    $t\,/\sek$ &$T_\su{1}$\,/$\Kel$ & $T_\su{2}\,/\Kel$ & $p_\su{a}\cdot 10^5\,/\,\pas $ &
    $p_\su{b}\cdot 10^5\,/ \pas$ & N\,/$\,\si{\watt}$  \\
    \midrule
       0   &   294.65   &   294.55   &   5.1    &     5.25   &      0 \\
      60   &   295.55   &   294.55   &   2.4    &     6.90   &    165 \\
     120   &   296.35   &   294.45   &   2.6    &     7.00   &    175 \\
     180   &   297.65   &   293.55   &   2.9    &     7.50   &    185 \\
     240   &   299.25   &   292.05   &   3.0    &     7.75   &    195 \\
     300   &   301.15   &   290.25   &   3.1    &     8.25   &    200 \\
     360   &   303.15   &   288.35   &   3.2    &     8.50   &    203 \\
     420   &   305.15   &   286.55   &   3.2    &     9.00   &    205 \\
     480   &   307.05   &   284.65   &   3.2    &     9.50   &    206 \\
     540   &   308.95   &   282.75   &   3.2    &     9.80   &    208 \\
     600   &   310.85   &   281.05   &   3.2    &    10.25   &    209 \\
     660   &   312.55   &   279.45   &   3.2    &    10.50   &    211 \\
     720   &   314.35   &   277.85   &   3.2    &    11.00   &    212 \\
     780   &   316.05   &   276.25   &   3.2    &    11.25   &    212 \\
     840   &   317.55   &   274.95   &   3.2    &    11.75   &    212 \\
     900   &   319.05   &   273.95   &   3.2    &    12.00   &    213 \\
     960   &   320.55   &   273.35   &   3.2    &    12.50   &    213 \\
    1020   &   321.95   &   272.85   &   3.2    &    13.00   &    210 \\
    1080   &   323.15   &   272.45   &   3.2    &    13.25   &    207 \\
    \bottomrule
  \end{tabular}
  \caption{Messwerte}
  \label{tab:messwerte}
\end{table}
\\
Der Verlauf der Temperaturen $T_1$ und $T_2$ wird graphisch in Abbildung
\ref{fig:T1T2} dargestellt.
\begin{figure}[h]
  \centering
  \includegraphics[width=0.7\textwidth]{Bilder/Plot1.pdf}
  \caption{Verlauf der Temperaturen in Abhängigkeit der Zeit}
  \label{fig:T1T2}
\end{figure}
\\
Die Ausgleichsrechnung ergibt sich über die Formel
\begin{equation*}
  T(t) = At^2+Bt+C.
\end{equation*}
Hieraus ergeben sich für $T_1$ die Parameter
\begin{align*}
  A &= (-3.0 \pm 1.0) \cdot10^{-6}\,\si{\kelvin\per\square\second} \\
  B &= (0.032 \pm 0.002) \,\si{\kelvin\per\second} \\
  C &= (292.5 \pm 0.4) \Kel
\end{align*}
und für $T_2$
\begin{align*}
  A &= (7.0 \pm 2.0) \cdot 10^{-6} \,\si{\kelvin\per\square\second}\\
  B &= (-0.033 \pm 0.003)  \,\si{\kelvin\per\second} \\
  C &= (298.5 \pm 0.7) \Kel.
\end{align*}

\noindent Aus diesen Parametern wird der Differenzenquotient
\begin{equation}
  \frac{dT}{dt} = 2At+B
\end{equation}
für 4 Zeitpunkte bestimmt.
\begin{table}[!h]
  \centering
  \begin{tabular}{c c c c c}
    \toprule
    $t\,/\sek$ & $T_1 \,/\Kel$ & $\frac{dT_1}{dt}\cdot 10^{-3}\,/\,\frac{\si{\kelvin}}{\si{\second}}$ &
    $T_2\,/\Kel$ & $\frac{dT_2}{dt}\cdot 10^{-3}\,/\,\frac{\si{\kelvin}}{\si{\second}}$ \\
    \midrule
    180 &  297.65 &  31\pm2 &  293.55 &  -30\pm3  \\
    540 &  308.95 &  29\pm2 &  282.75 &  -25\pm4  \\
    720 &  314.35 &  28\pm2 &  277.85 &  -23\pm4  \\
    900 &  319.05 &  27\pm3 &  273.95 &  -20\pm5  \\
    \bottomrule
  \end{tabular}
  \caption{Differenzenquotienten zu $T_1$ und $T_2$}
  \label{tab:diff}
\end{table}
\\
Im Folgenden wird die Güteziffer $\nu$ der Wärmepumpe mit Formel \eqref{eqn:güte}
ermittelt. Dabei wird nur $T_1$ betrachtet, da im Idealfall davon ausgegangen wird,
dass $T_1$ und $T_2$ gleich sind (reversibler Kreisprozess).
Die Wärmekapazität der Eimer und Kupferschlangen kann an der Apparatur abgelesen werden,
sie beträgt $660\,\si{\joule\per\kelvin}$. In einem Eimer befinden sich
$3\,\si{\kilo\gram}$ Wasser mit der Wärmekapazität
$c_\su{w}= 4.183 \,\si{\kilo\joule\per\kilo\gram\kelvin}$\,\cite{chemie}.
Für die Leistung $N$ wird der Mittelwert mit $\overline{N}=(190\pm50)\,\si{\watt}$ verwendet.
Der Fehler der Güteziffer berechnet sich nach der Gauß'schen Fehlerfortpflanzung
mit:
\begin{equation*}
\Delta\nu = \sqrt{\biggl(\frac{(m_\su{w}c_\su{w}+m_\su{w}c_\su{w})}{N}\biggr)^2
\cdot\sigma_\su{\frac{dT}{dt}}^2 + \biggl(\frac{(m_\su{w}c_\su{w}+m_\su{w}c_\su{w})\frac{dT_1}{dt}}{N^2}\biggr)^2
\cdot\sigma_\su{N}^2}
\end{equation*}
\begin{table}[!h]
  \centering
  \begin{tabular}{c c}
    \toprule
    $t\,/\sek$ & $\nu_\su{exp}$ \\
    \midrule
    180 &  2.2\pm 0.6 \\
    540 &  2.0\pm 0.6 \\
    720 &  2.0\pm 0.6 \\
    960 &  1.9\pm 0.6 \\
    \bottomrule
  \end{tabular}
  \caption{Experimentelle Güteziffer}
  \label{tab:expgüte}
\end{table}\\

\noindent Der Massendurchsatz $\frac{dm}{dt}$ lässt sich nach Formel \eqref{eqn:dm/dt}
berechnen. Die benötigte Verdampfungswärme lässt sich aus dem Graphen
aus Abbildung \ref{fig:verdampfung} berechnen, da
\begin{align*}
  L &= -m\cdot \su{R} \\
  \text{mit}\quad \su{R}&= 8,314472 \,\si{\joule\per\mol\kelvin} \\
  \text{und}\quad \su{m}&= (-2.4 \pm 0.1)\,\cdot 10^3
\end{align*}
gilt. R ist dabei die Gaskonstante\,\cite{chem} und m die Steigung des Graphen.
Hierbei wird $1\,/\,T$ gegen $\log{(p_\su{b}\,/\,p_0)}$ aufgetragen.
\begin{figure}[H]
  \centering
  \includegraphics[width=0.8\textwidth]{Bilder/verdampfung.pdf}
  \caption{Verdampfungswärme von $T_1$}
  \label{fig:verdampfung}
\end{figure}
\noindent Für die vier verschiedenen Zeiten ergeben sich die in Tabelle
\ref{tab:dm/dt} gelisteten Massendurchsätze.
\begin{table}
  \centering
  \begin{tabular}{c c c}
    \toprule
    $t\,/\sek$ & $\frac{dm}{dt}\,/\,\si{\mol\per\second}$ &$\frac{dm}{dt}\,/\,\si{\kilo\gram\per\second}$ \\
    \midrule
    180  & (19.9\pm 0.2)$\cdot\,10^{-3}$ &(2.34 \pm 0.02)$\cdot\,10^{-3}$  \\
    540  & (16.5\pm 0.2)$\cdot\,10^{-3}$ &(1.98 \pm 0.02)$\cdot\,10^{-3}$  \\
    720  & (15.2\pm 0.2)$\cdot\,10^{-3}$ &(1.82 \pm 0.02)$\cdot\,10^{-3}$  \\
    960  & (13.2\pm 0.2)$\cdot\,10^{-3}$ &(1.58 \pm 0.02)$\cdot\,10^{-3}$  \\
    \bottomrule
  \end{tabular}
  \caption{Massendurchsatz zu verschiedenen Zeiten}
  \label{tab:dm/dt}
\end{table} \\

\noindent Um $\si{\mol}$ in $\kg$ umzurechnen wird folgende Formel benötigt:
\begin{align*}
  m &= n\cdot M \\
  \text{mit}\quad M &= 120.9\,\cdot 10^{-3} \si{\kilo\gram\per\mol}
\end{align*}

\noindent Auch hier wird der Fehler mittels der Fehlerfortpflanzung berechnet. Der Fehler
lässt sich dann durch
\begin{equation*}
  \Delta\frac{dm}{dt}=\sqrt{\biggl(\frac{(m_\su{w}c_\su{w}+m_\su{w}c_\su{w})}{L}\biggr)^2
  \cdot\sigma_\su{\frac{dT_2}{dt}}^2 + \biggl(\frac{(m_\su{w}c_\su{w}+m_\su{w}c_\su{w})\frac{dT_2}{dt}}{L^2}\biggr)^2
  \cdot\sigma_\su{L}^2}
\end{equation*}
berechnen.

\noindent Um die mechanische Leistung des Kompressors zu berechnen, wird Formel \eqref{eqn:Nmech}
verwendet. Das darin enthaltene $\rho$ berechnet sich nach Formel \eqref{eqn:rho}.
\\
\begin{table}[!h]
  \centering
  \begin{tabular}{c c}
    \toprule
    $t\,/\sek$ & $\rho\,/\,\Dichte$ \\
    \midrule
    180 &  14.87 \\
    540 &  17.03 \\
    720 &  17.33 \\
    960 &  17.58 \\
    \bottomrule
  \end{tabular}
  \caption{Dichte}
\end{table} \\
\newpage
\noindent Für die verschiedenen Zeiten ergeben sich die in Tabelle \ref{tab:Nmech} aufgelisteten
Werte.
\begin{table}[!h]
  \centering
  \begin{tabular}{c c}
    \toprule
    $t\,/\sek$ & $N_\su{mech}\,/\si{\watt}$ \\
    \midrule
    180 &  40.4  \pm   0.3 \\
    540 &  39.2  \pm   0.4 \\
    720 &  39.3  \pm   0.4 \\
    960 &  36.2  \pm   0.5 \\
    \bottomrule
  \end{tabular}
  \caption{Mechanische Leistung}
  \label{tab:Nmech}
\end{table} \\

\noindent Auch hier wird der Fehler mittels Fehlerfortpflanzung berechnet, sodass:
\begin{equation*}
  \Delta N = \sqrt{\biggl(\frac{1}{\kappa-1}\,\biggl(p_\su{b}\sqrt[\kappa]{\frac{p_\su{a}}{p_\su{b}}}-p_\su{a}\biggr)
  \frac{1}{\rho}\biggr)^2\cdot\sigma_\frac{dm}{dt}^2}
\end{equation*}
gilt.
