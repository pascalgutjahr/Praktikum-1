Tabelle \ref{tab:messwerte} zeigt die im Versuch gemessenen Werte.
$T_1$ entspricht hierbei der Temperatur des wärmeren Reservoirs, während $T_2$
die Temperatur des Kälteren. Der Druck $P_\su{a}$ wird dem warmen Reservoir
zugeordnet und $p_\su{b}$ dem Kälteren.
\begin{table}[h]
  \centering
  \begin{tabular}{c c c c c c}
    \toprule
    t\,/$\sek$ &$T_\su{1}$\,/$\Kel$ & $T_\su{2}\,/\Kel$ & $p_\su{a}\,/\,\pas $ &
    $p_\su{b}\,/ \pas$ & N\,/$\,\si{\watt}$  \\
    \midrule
       0   &   294.65   &   294.55   &   5100000    &     5250000   &      0 \\
      60   &   295.55   &   294.55   &   2400000    &     6900000   &    165 \\
     120   &   296.35   &   294.45   &   2600000    &     7000000   &    175 \\
     180   &   297.65   &   293.55   &   2900000    &     7500000   &    185 \\
     240   &   299.25   &   292.05   &   3000000    &     7750000   &    195 \\
     300   &   301.15   &   290.25   &   3100000    &     8250000   &    200 \\
     360   &   303.15   &   288.35   &   3200000    &     8500000   &    203 \\
     420   &   305.15   &   286.55   &   3200000    &     9000000   &    205 \\
     480   &   307.05   &   284.65   &   3200000    &     9500000   &    206 \\
     540   &   308.95   &   282.75   &   3200000    &     9800000   &    208 \\
     600   &   310.85   &   281.05   &   3200000    &    10250000   &    209 \\
     660   &   312.55   &   279.45   &   3200000    &    10500000   &    211 \\
     720   &   314.35   &   277.85   &   3200000    &    11000000   &    212 \\
     780   &   316.05   &   276.25   &   3200000    &    11250000   &    212 \\
     840   &   317.55   &   274.95   &   3200000    &    11750000   &    212 \\
     900   &   319.05   &   273.95   &   3200000    &    12000000   &    213 \\
     960   &   320.55   &   273.35   &   3200000    &    12500000   &    213 \\
    1020   &   321.95   &   272.85   &   3200000    &    13000000   &    210 \\
    1080   &   323.15   &   272.45   &   3200000    &    13250000   &    207 \\
    \bottomrule
  \end{tabular}
  \caption{Messwerte}
  \label{tab:messwerte}
\end{table}
\\
Der Verlauf der Temperaturen $T_1$ und $T_2$ wird graphisch in Abbildung
\ref{fig:T1T2} dargestellt.
\begin{figure}[h]
  \centering
  \includegraphics[width=0.7\textwidth]{Bilder/Plot1.pdf}
  \caption{Verlauf der Temperaturen in Abhängigkeit der Zeit}
  \label{fig:T1T2}
\end{figure}
\\
Die die Ausgleichsrechnung ergibt sich über die Formel
\begin{equation*}
  T(t) = At^2+Bt+C
\end{equation*}
hieraus ergeben sich für $T_1$ die Parameter
\begin{align*}
  A &= -3.3\cot10^{-7} \pm
\end{align*}
