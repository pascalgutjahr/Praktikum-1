Tabelle \ref{tab:messwerte} zeigt die im Versuch gemessenen Werte.
$T_\su{1}$ entspricht hierbei der Temperatur des wärmeren Reservoirs, während $T_\su{2}$
die Temperatur des kälteren Reservoires beschreibt. Der Druck $P_\su{a}$ wird dem warmen Reservoir
zugeordnet und $p_\su{b}$ dem kälteren.
\begin{table}[h]
  \centering
  \begin{tabular}{c c c c c c}
    \toprule
    $t\,/\sek$ &$T_\su{1}$\,/$\Kel$ & $T_\su{2}\,/\Kel$ & $p_\su{a}\,/\,\pas $ &
    $p_\su{b}\,/ \pas$ & N\,/$\,\si{\watt}$  \\
    \midrule
       0   &   294.65   &   294.55   &   510000    &     525000   &      0 \\
      60   &   295.55   &   294.55   &   240000    &     690000   &    165 \\
     120   &   296.35   &   294.45   &   260000    &     700000   &    175 \\
     180   &   297.65   &   293.55   &   290000    &     750000   &    185 \\
     240   &   299.25   &   292.05   &   300000    &     775000   &    195 \\
     300   &   301.15   &   290.25   &   310000    &     825000   &    200 \\
     360   &   303.15   &   288.35   &   320000    &     850000   &    203 \\
     420   &   305.15   &   286.55   &   320000    &     900000   &    205 \\
     480   &   307.05   &   284.65   &   320000    &     950000   &    206 \\
     540   &   308.95   &   282.75   &   320000    &     980000   &    208 \\
     600   &   310.85   &   281.05   &   320000    &    1025000   &    209 \\
     660   &   312.55   &   279.45   &   320000    &    1050000   &    211 \\
     720   &   314.35   &   277.85   &   320000    &    1100000   &    212 \\
     780   &   316.05   &   276.25   &   320000    &    1125000   &    212 \\
     840   &   317.55   &   274.95   &   320000    &    1175000   &    212 \\
     900   &   319.05   &   273.95   &   320000    &    1200000   &    213 \\
     960   &   320.55   &   273.35   &   320000    &    1250000   &    213 \\
    1020   &   321.95   &   272.85   &   320000    &    1300000   &    210 \\
    1080   &   323.15   &   272.45   &   320000    &    1325000   &    207 \\
    \bottomrule
  \end{tabular}
  \caption{Messwerte}
  \label{tab:messwerte}
\end{table}
\\
Der Verlauf der Temperaturen $T_1$ und $T_2$ wird graphisch in Abbildung
\ref{fig:T1T2} dargestellt.
\begin{figure}[h]
  \centering
  \includegraphics[width=0.7\textwidth]{Bilder/Plot1.pdf}
  \caption{Verlauf der Temperaturen in Abhängigkeit der Zeit}
  \label{fig:T1T2}
\end{figure}
\\
Die Ausgleichsrechnung ergibt sich über die Formel
\begin{equation*}
  T(t) = At^2+Bt+C.
\end{equation*}
Hieraus ergeben sich für $T_1$ die Parameter
\begin{align*}
  A &= (-3 \pm 1) \cdot10^{-6}\,\si{\kelvin\per\square\second} \\
  B &= (0.032 \pm 0.002) \,\si{\kelvin\per\second} \\
  C &= (292.5 \pm 0.4) \Kel
\end{align*}
und für $T_2$
\begin{align*}
  A &= (7.0 \pm 2) \cdot 10^{-6} \,\si{\kelvin\per\square\second}\\
  B &= (-0.033 \pm 0.003)  \,\si{\kelvin\per\second} \\
  C &= (298.5 \pm 0.7) \Kel.
\end{align*}

\noindent Aus diesen Parametern wird der Differenzenquotient
\begin{equation}
  \frac{dT}{dt} = 2At+B
\end{equation}
für 4 Zeitpunkte bestimmt.
\begin{table}[!h]
  \centering
  \begin{tabular}{c c c c c}
    \toprule
    $t\,/\sek$ & $T_1 \,/\Kel$ & $\frac{dT_1}{dt}\,/\si{\milli\kelvin\per\second}$ &
    $T_2\,/\Kel$ & $\frac{dT_2}{dt}\,/\si{\milli\kelvin\per\second}$ \\
    \midrule
    180 &  297.65 &  31\pm2 &  293.55 &  -29\pm4  \\
    540 &  308.95 &  29\pm3 &  282.75 &  -24\pm6  \\
    720 &  314.35 &  28\pm4 &  277.85 &  -21\pm7  \\
    900 &  319.05 &  27\pm4 &  273.95 &  -18\pm8  \\
    \bottomrule
  \end{tabular}
  \caption{Differenzenquotienten zu $T_1$ und $T_2$}
  \label{tab:diff}
\end{table}
\\
Im Folgenden wird die Güteziffer $\nu$ der Wärmepumpe mit Formel \eqref{eqn:güte}
ermittelt. Dabei wird nur $T_1$ betrachtet, da im Idealfall davon ausgegangen wird,
dass $T_1$ und $T_2$ gleich sind (reversibler Kreisprozess).
Die Wärmekapazität der Eimer und Kupferschlangen kann an der Apparatur abgelesen werden,
sie beträgt $660\,\si{\joule\per\kelvin}$. In einem Eimer befinden sich
$3\,\si{\kilo\gram}$ Wasser mit der Wärmekapazität
$c_\su{w}= 4.183 \,\si{\kilo\joule\per\kilo\gram\kelvin}$\,\cite{chemie}.
Für die Leistung $N$ wird der Mittelwert mit $\overline{N}=(190\pm50)\,\si{\watt}$ verwendet.
Der Fehler der Güteziffer berechnet sich nach der Gauß'schen Fehlerfortpflanzung
mit:
\begin{equation*}
\Delta\nu = \sqrt{\biggl(\frac{(m_\su{w}c_\su{w}+m_\su{w}c_\su{w})}{N}\biggr)^2
\cdot\sigma_\su{\frac{dT}{dt}}^2 + \biggl(\frac{(m_\su{w}c_\su{w}+m_\su{w}c_\su{w})\frac{dT_1}{dt}}{N^2}\biggr)^2
\cdot\sigma_\su{N}^2}
\end{equation*}
\begin{table}[!h]
  \centering
  \begin{tabular}{c c}
    \toprule
    $t\,/\sek$ & $\nu_\su{exp}$ \\
    \midrule
    180 &  2.2\pm 0.6 \\
    540 &  2.0\pm 0.6 \\
    720 &  2.0\pm 0.6 \\
    960 &  1.9\pm 0.6 \\
    \bottomrule
  \end{tabular}
  \caption{Experimentelle Güteziffer}
  \label{tab:expgüte}
\end{table}\\

Der Massendurchsatz $\frac{dm}{dt}$ lässt sich nach Formel \eqref{eqn:dm/dt}
berechnen. Für die vier verschiedenen Zeiten erhält man die in Tabelle
\ref{tab:dm/dt} gelisteten Massendurchsätze
\begin{table}
  \centering
  \begin{tabular}{c c}
    \toprule
    $t\,/\sek$ & $\frac{dm}{dt}\,/\,\frac{\si{\mol}\cdot\Kel}{\sek}$ \\
    \midrule
    180  &  (-19.1 \pm 3)$\cdot\,10^{-3}$  \\
    540  &  (-15.8 \pm 4)$\cdot\,10^{-3}$  \\
    720  &  (-13.8 \pm 5)$\cdot\,10^{-3}$  \\
    960  &  (-11.8 \pm 5)$\cdot\,10^{-3}$  \\
  \end{tabular}
  \caption{Massendurchsatz zu verschiedenen Zeiten}
  \label{tab:dm/dt}
\end{table}
Auch hier wird der Fehler mittels der Fehlerfortpflanzung brechnet. Der Fehler
lässt sich dann durch
\begin{equation*}
  \Delta\frac{dm}{dt}=\sqrt{\biggl(\frac{(m_\su{w}c_\su{w}+m_\su{w}c_\su{w})}{L}\biggr)^2
  \cdot\sigma_\su{\frac{dm}{dt}}^2 + \biggl(\frac{(m_\su{w}c_\su{w}+m_\su{w}c_\su{w})\frac{dT_2}{dt}}{L^2}\biggr)^2
  \cdot\sigma_\su{L}^2}
\end{equation*}
berechnen.

Um die mechanische Leistung des Kompressors zu berechnen, wird Formel \eqref{eqn:Nmech}
verwendet. Das darin enthaltene $\rho$ berechnet sich nach Formel \eqref{eqn:rho}.
Für die verschiedenen Zeiten erhält man die in Tabelle \ref{tab:Nmech} aufgelisteten
Werte.
\begin{table}[!h]
  \centering
  \begin{tabular}{c c}
    \toprule
    $t\,/\sek$ & $N_\su{mech}\,/\si{\watt}$ \\
    \midrule
    180 &  32.9 \pm 3.5  \\  %Werte sind eigentlich negativ, habe also den Betrag genommen
    540 &  43.2 \pm 4.7  \\
    720 &  48.0 \pm 6.3  \\
    960 &  53.4 \pm 6.7  \\
    \bottomrule
  \end{tabular}
  \caption{Mechanische Leistung}
  \label{tab:Nmech}
\end{table}
Auch hier wird der Fehler mittels Fehlerfortpflanzung berechnet, sodass:
\begin{equation*}
  \Delta N = \sqrt{\biggl(\frac{1}{\kappa-1}\,p_\su{b}\sqrt[\kappa]{\frac{p_\su{a}}{p_\su{b}}}
  \frac{1}{\rho}\biggr)^2\cdot\sigma_\frac{dm}{dt}}
\end{equation*}
gilt.
