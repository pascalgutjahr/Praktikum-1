Wärme fließt grundsätzlich vom wärmeren Medium zum kälteren Medium - laut dem
zweiten Hauptsatz der Thermodynamik. Für die Umkehrung dieses Prozesses muss
Energie von außen zugeführt werden, z.B. durch eine Wärmepumpe, welche mechanische
Arbeit in das System steckt. Es gilt:
\begin{equation}
  Q_\su{1} = Q_\su{2} + A
\end{equation}
mit $Q_\su{1}$ als wärmere Wärmemenge, $Q_\su{2}$ als kältere Wärmemenge und $A$ als
aufzuwendende Arbeit.
\subsection{Merkmale der Wärmepumpe}
Eine Wärmepumpe besitzt im Wesentlichen drei entscheidende Faktoren: die Güteziffer,
den Massendurchsatz und den Wirkunsgrad.
\subsubsection{Güteziffer}
Die Güteziffer $\nu$ bezeichnet das Verhältnis zwischen der Arbeit $A$, welche aufzuwenden
ist, um die Wärme vom kälteren zum wärmeren Reservoir zu transportieren, und der
transportierten Wärmemenge $Q_\su{trans}$. Unter idealen Voraussetzungen gilt
\begin{equation}
      \nu_\su{ideal} = \frac{Q_\su{trans}}{A} = \frac{T_\su{1}}{T_\su{1}-T_\su{2}}.
\end{equation}
Hierfür muss die Wärmeübertragung jedoch reversibel sein. Für die Realität folgt
somit
\begin{equation}
  \nu_\su{real} < \frac{T_\su{1}}{T_\su{1}-T_\su{2}}.
\end{equation}
Je geringer die Temperaturunterschiede zwischen den Reservoiren sind, desto höher ist
die Güteziffer und desto weniger Arbeit muss die Pumpe verrichten.

Wird der Differenzenquotient durch die Differentialquotienten $\frac{dT_\su{1}}{dt}$
und $\frac{dQ_\su{1}}{dt}$ ersetzt, so folgt:
\begin{equation}
  \frac{dQ_\su{1}}{dt}= (m_\su{1}c_\su{w}+m_\su{k}c_\su{k})\frac{dT_\su{1}}{dt}
\end{equation}
$m_\su{1}c_\su{w}$ beschreibt die Wärmekapazität des Wassers im ersten Reservoir und
$m_\su{k}c_\su{k}$ die Wärmekapazität der Kupferschlange und des Eimers. Mit $N$
als gemittelte Leisungsaufnahme des Kompressors ergibt sich damit für die Güteziffer
$\nu$:
\begin{equation}
  \nu = (m_\su{1}c_\su{w}+m_\su{k}c_\su{k})\frac{dT_\su{1}}{dt} \cdot \frac{1}{N}
\end{equation}
\subsubsection{Massendurchsatz}
Der Massendurchsatz berechnet sich aus dem Quotienten $\frac{dT_\su{2}}{dt}$ als
\begin{equation}
  \frac{dQ_\su{2}}{dt}= (m_\su{2}c_\su{w}+m_\su{k}c_\su{k})\frac{dT_\su{2}}{dt}.
\end{equation}
Der Wärmeentzug geschieht durch die Verdampfung des Transportmediums. Dabei wird die
Verdampfungswärme $L$ pro Zeit- und Masseneinheit verbraucht. Aus diesem Zusammenhang folgt
\begin{align}
  \frac{dm}{dt} &= \frac{dQ_\su{2}}{dt} \cdot \frac{1}{L} \\
      &= (m_\su{2}c_\su{w}+m_\su{k}c_\su{k})\frac{dT_\su{2}}{dt} \cdot \frac{1}{L}.
\end{align}
\subsubsection{Wirkungsgrad}
\subsection{Schematischer Aufbau der Wärmepumpe}
