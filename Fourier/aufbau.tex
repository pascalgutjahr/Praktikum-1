\subsection{Vorbereitung}
Bevor der eigentliche Versuch beginnt, werden für drei periodische Funktionen
die jeweiligen Fourier-Koeffizienten berechnet.
\subsubsection{Rechteckspannung}
Um die Koeffizienten der Rechteck-Schwingung zu berechnen, wird diese zunächst
als ungerade Funktion definiert, sodass $f(t)=-f(-t)$ gilt. Daraus wird direkt
gefolgert, dass der Koeffizient $a_\su{n}=0$ ist.
Die Rechteckschwingung wird über die Funktion
% \begin{itemize}
  % \item Rechteckspannung
  \begin{align}
  f(t)&=
  \begin{cases}
    1 & \, 0<t<\frac{\pi}{2} \\
    -1& \, \frac{\pi}{2}<t<\pi \\
  \end{cases}
\end{align}
% \end{itemize}
definiert. Nach Formel \eqref{eq:analyse} berechnet sich der Koeffizient $b_\su{n}$
wie folgt:
\begin{align*}
  b_\su{n} &= \frac{2}{\pi}\int^\pi_0{f(t)\sin{\left(\frac{2\pi n t}{\pi}\right)} dt} \\
  &= \frac{2}{\pi}\left(\int_0^{\frac{\pi}{2}}{\sin{(2nt)} dt} -
  \int_{\frac{\pi}{2}}^\pi{\sin{(2nt)} dt}\right) \\
  &= \frac{2}{\pi}\left(\left[-\cos{(2nt)}\frac{1}{2n}\right]_0^{\frac{\pi}{2}} +
  \left[\cos{(2nt)}\frac{1}{2n}\right]_{\frac{\pi}{2}}^\pi\right) \\
  &= \frac{2}{\pi}\left(\frac{1}{2n}+\frac{1}{2n}+\frac{1}{2n}+\frac{1}{2n}\right) \\
  &= \frac{2}{\pi}\left(\frac{2}{n}\right) \\
  &= \frac{4}{\pi n}.
\end{align*}
Da $\cos{(2\pi)}=0$ ist, ist der Cosinus auch für jedes ganzzahlige Vielfache von
$2\pi$ gleich 0.
Zudem muss die Startbedingung $a_0$ berechnet werden. Dies geschieht mit der
Formel
\begin{equation*}
  a_0 = \frac{1}{\pi}\int_{-\pi}^\pi{f(x) dx}.
\end{equation*}
Somit ist
\begin{align*}
  a_0 &= \frac{1}{\pi}\int_{-\pi}^\pi{f(x) dx} \\
  &= \frac{2}{\pi}\int_0^{\pi}{f(x) dx} \\
  &= \frac{2}{\pi}\int_0^{\frac{\pi}{2}}{1 dx} -
  \int^\pi_{\frac{\pi}{2}}{1 dx} \\
  &= \frac{2}{\pi}\left(\biggl[x\biggr]_0^{\frac{\pi}{2}} -
  \biggl[x\biggr]_{\frac{\pi}{2}}^\pi\right) \\
  &= -2
\end{align*}
Mit den berechneten Koeffizienten lautet die Fourier-Reihe für die Rechteckspannung
\begin{equation}
  f(t) = -1 + \sum_{n=1}^\infty{\frac{4}{n\pi}\sin{\left(n\frac{2\pi}{\su{T}}t\right)}}
\end{equation}
\subsubsection{Sägezahnspannung}

\par
\noindent Für die Rechteckspannung und die Sägezahnspannung ergeben sich damit die Zusammenhänge
von $\sfrac{1}{n}$, für die Dreieckspannung ergibt sich $\sfrac{1}{n^2}$.
\subsection{Fourier-Synthese}
Für die Fourier-Synthese werden zunächst zwei Ausgänge des Oberwellengenerators
an das Oszilloskop angeschlossen. Damit alle Ausgänge in Phase geschalten werden
können, wird das Oszilloskop zunächst in den X-Y-Betrieb geschaltet. Um eine
optimale Genauigkeit zu erhalten, werden die einzelnen Amplituden der Oberwellen
auf maximal eingestellt. Nun werden die Phasen so lange variiert, bis sich
Lissajous-Figuren, wie beispielsweise in Abbildung \ref{lissa}, bilden.
\begin{figure}[H]
  \centering
  \includegraphics{bilder/lissajous.jpg}
  \caption{Beispiel für eine Lissajous-Figur}
  \label{lissa}
\end{figure}
% Wie es sich in der Vorbereitung
% herausgestellt hat, fallen die Koeffizienten der Funktionen der
% Sägezahn- und Recht-
% eckspannung mit dem Proportionalitätsfaktor X ab.
Die Ausgänge des Oszilloskops werden nun nacheinander an ein Voltmeter
angeschlossen. Die Amplituden der Oberwellen werden entsprechend eingestellt.
Das Oszilloskop wird in den normalen Betrieb zurückgeschaltet und der Generator
an das Oszilloskop angeschlossen. Die einzelnen Oberwellen werden aufaddiert und
gegebenenfalls in der Phase verschoben, bis möglichst genau eine Sägezahnspannung
approximiert wird.
Für die Rechteckspannung wird ähnlich vorgegangen, nur dass jetzt nur die
ungeraden Oberwellen benutzt werden.

Abschließend wird die Durchführung für die Dreieckspannung wiederholt. Hier müssen
zunächst die Amplituden mit dem Proportionalitätsfaktor X eingestellt werden.
Auch hier werden, wie bei der Rechteckspannung, nur die ungeraden Oberwellen
genutzt. Die Phase wird wieder so verschoben, dass die Dreieckspannung möglichst
genau approximiert wird.
\subsection{Fourier-Analyse}
Für die Fourier-Analyse von periodischen Schwingungen wird eine Schaltung wie in
Abbildung \ref{fig:ana} verwendet.
\begin{figure}[h]
  \centering
  \includegraphics[width=0.8\textwidth]{bilder/analyse.jpg}
  \caption{Schaltung zur Fourier-Analyse\,\cite{351}}
  \label{fig:ana}
\end{figure}
Als Signalquelle dient hier ein durchstimmbarer Funktionsgenerator. Die
Transformation wird direkt vom Oszilloskop durchgeführt.

Damit genügend Peaks des Linienspektrums sichtbar werden, müssen ausreichend
Perioden angezeigt werden. Die angezeigten Peaks werden nun mit einer
Sinusspannung kalibriert und die Frequenz und die Amplitude werden notiert.
Nebenmaxima die sich bilden können, werden nicht beachtet. Die Messung wird für
alle genannten Funktionen durchgeführt.
