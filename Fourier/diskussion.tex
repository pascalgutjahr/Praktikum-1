Die Abweichungen der Einzelmessungen für die Fourier-Analyse ergeben sich aus der Formel
\begin{equation}
  \sigma = \frac{a_\su{theo}-a_\su{exp}}{a_\su{theo}}.
\end{equation}
Die jeweiligen Werte wurden dabei schon in der dazugehörigen Tabelle eingetragen.


\noindent Bei der Rechteckspannung ergab sich dabei eine sehr hohe Abweichung, da die Peaks nur
sehr niedrig waren, wodurch wir die Werte nicht genau messen konnten. Die Abweichungen
bei der Sägezahn- und Dreieckspannung sind im Rahmen der Messungenauigkeiten in Ordnung,
da wir auch hier den Cursor nicht immer exakt auf den Peak legen konnten.


\noindent Für die Abweichung der gesamten Kurve wird Formel \eqref{eqn:fehler} verwendet.
Hier wird die Abweichung der Steigung zum Wert $2$ bzw. $1$ betrachtet.
% für welche Funktion wird 1 bzw 2 betrachtet, also wann hat man a_n und wann b_n?????
% habe jetzt für saegzahn die 1 genommen und für recht und dreieck die 2
% ---> dies kann man an 1/n bzw. 1/n^2 erkennen!
Der Fit der Dreieckspannung weicht somit um $10\,\%$ ab, der Fit der Rechteckspannung
um $9\,\%$ und der Fit der Sägezahnspannung zeigt keine Abweichung.
Im Rahmen der Messungenauigkeiten sind diese Messungen somit sehr gut gelungen.
% Die hohe Abweichung
% der Rechteckspannung folgt aus den wenigen Messwerten, die dazu aufgenommen wurden.
% Für die Rechteckspannung werden nämlich mehrere Messwerte benötigt, da die Rechteckspannung stärker von
% der gewöhnlichen Sinunsspannung abweicht, als die anderen zwei verwendeten Spannungen.
% Zudem ergeben sich Messungenauigkeiten beim Ablesen der Amplituden.

\noindent Bei der Fourier-Synthese haben wir nahezu alle Werte exakt einstellen können,
nur bei der Dreieckspannung war dies etwas schwierig, da wir für den vierten Wert $38 \mV$
einstellen sollten, jedoch ließ sich die Spannung nur bis zu einem Minimalwert von $48 \mV$
einstellen. Diesen Wert haben wir aber trotzdem noch in die Messung mit einbezogen.
