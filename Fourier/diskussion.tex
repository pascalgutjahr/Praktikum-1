Die Abweichungen der Einzelmesungen für die Fourier-Analyse ergeben sich aus der Formel
\begin{equation}
  \sigma = \frac{a_\su{theo}-a_\su{exp}}{a_\su{theo}}.
\end{equation}
Die jeweiligen Werte wurden dabei schon in der dazugehörigen Tabelle eingetragen.

Für die Abweichung der gesamten Kurve wird Formel \eqref{eqn:fehler} verwendet.
Der Fit der Dreieckspannung weicht somit um $50\,\%$ ab, der Fit der Rechteckspannung
um $19\,\%$ und der Fit der Sägezahnspannung zeigt keine Abweichung. Die hohe Abweichung
der Dreieckspannung folgt aus den wenigen Messwerten, die dazu aufgenommen wurden.
Zudem ergeben sich Messungenauigkeiten beim Ablesen der Amplituden.

\noindent Bei der Fourier-Synthese haben wir nahezu alle Werte exakt einstellen können,
nur bei der Dreieckspannung war dies etwas schwierig, da wir für den vierten Wert $38 \mV$
einstellen sollten, jedoch ließ sich die Spannung nur bis zu einem Minimalwert von $48 \mV$
einstellen. Diesen Wert haben wir aber trotzdem noch in die Messung mit einbezogen.
