Die Abweichungen der Einzelmesungen für die Fourier-Analyse ergeben sich aus der Formel
\begin{equation}
  \sigma = \frac{a_\su{theo}-a_\su{exp}}{a_\su{theo}}.
\end{equation}
Die jeweiligen Werte wurden dabei schon in der dazugehörigen Tabelle eingetragen.

Für die Abweichung der gesamten Kurve wird Formel \eqref{eqn:fehler} verwendet.
Der Fit der Dreieckspannung weicht somit um $50\,\%$ ab, der Fit der Rechteckspannung
um $19\,\%$ und der Fit der Sägezahnspannung zeigt keine Abweichung.

\noindent Die hohe Abweichung bei der Dreieckspannung ergibt sich daraus, dass wir nur drei Werte genau
einstellen konnten. Bei dem vierten Wert sollte wir $38 \mV$ einstellen, jedoch
ließ sich die Spannung nur bis zu einem Minimalwert von $48 \mV$ einstellen. Diesen
Wert haben wir aber trotzdem noch in die Messung mit einbezogen.
