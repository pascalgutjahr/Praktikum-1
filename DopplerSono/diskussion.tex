In Tabelle \ref{tab:mittelwerte} ist schön zu sehen, dass die Strömungsgeschwindigkeiten bei einem kleineren Rohrdurchmesser steigen und bei einem größeren Rohrdurchmesser sinken. Dies wurde auch genau so vorhergesagt, damit der Volumenstrom erhalten bleibt. Des Weiteren ist
an den geringen Fehlern zu erkennen, dass die Bestimmung der Strömungsgeschwindigkeit unabhängig vom Einstrahlwinkel $\theta$ ist.
Auch dies ist logisch, da sich die Strömungsgeschwindikgeit natürlich nicht für verschiedene Messarten ändert.

In den Plots \ref{fig:dünn}, \ref{fig:mittel} und \ref{fig:dick} ist klar zu erkennen, dass ein linearer Zusammenhang zwischen der Frequenzverschiebung und den Geschwindigkeiten besteht.
Unabhängig vom Rohrdurchmesser gilt somit, dass die Frequenzverschiebung proportional zur Geschwindigkeit ist. Dieser Zusammenhang findet sich in Formel \eqref{eqn:v} wieder.

Bei der Bestimmung des Strömungsprofils ist ein parabelförmiges Profil zu erkennen, welches sich in Ausbreitungsrichtung erstreckt. Hierbei muss jedoch beachtet werden, dass die Angabe von $6\mm = 4\ms$ innerhalb der
Dopplerflüssigkeit nicht korrekt sein kann, da sich nach dieser Angabe das Maximum schon bei $\sfrac{1}{4}$ der Gesamtstrecke innerhalb des Rohres befindet. Nach der Hälfte der Strecke, wo eigentlich das Maximum liegen müsste, zeigt sich gar keine Veränderung mehr, wodurch wir bei der
Messung somit schon außerhalb des Rohre liegen müssen.

Bei der Messung der Streuintensität sollte sich ein umgekehrtes Parabelprofil ergeben. Dies ist jedoch definitiv nicht zu erkennen.
Eine Fehlerquelle hierbei ist das Schwanken beim Ablesen der Streuintensität, da hier ständig Schwankungen von $\pm 100\cdot 10^3\,
\si{\square\volt\per\second}$ vorlagen, wodurch das Ablesen nahezu
willkürlich erfolgte.
