Es ist relativ gut gelungen die Charakteristik des Zählrohrs zu ermitteln.
Die Plateausteigung von $1.23\,\%$ ist nur sehr gering und spricht somit für
ein gut arbeitendes Zählrohr. Die Plateaubreite von $180\Volt$ entspricht
ungefähr der Breite aus der Theoriekurve \ref{fig:bereiche}. Hierbei
ist jedoch auffällig, dass das Plateau nicht geradlinig ansteigt, sondern
nach einem kleinen Peak erst einmal abfällt. Dies liegt daran, dass die
Messwerte die ganze Zeit ein wenig hin- und herschwanken.

Die Messung der Erholzeit und der Nachentladungsimpulse hat sich als sehr
schwierig gestaltet, da man die Werte nur erraten konnte und man die
Anfänge und Enden der Peaks nicht klar lokalisieren konnte.

Bei der Bestimmung der Totzeit mithilfe des Oszilloskops wird die Breite des Primärpeaks gemessen, jedoch beginnt der Graph nicht auf Höhe der x-Achse, sondern erst ein wenig
später. Wir haben ab Beginn des Graphen gemessen, in der Realität führt
dies jedoch dazu, dass die Totzeit etwas größer ist.
Die Totzeitmessung mit der Zwei-Quellen-Methode ist gut gelungen. Bei
$U=450\Volt$ ergibt sich eine Totzeit von $T=0.15\,\si{\milli\second}$.
Dies zeigt eine relative Abweichung von $13\,\%$ zur Totzeit $T=0.13\,\si{\milli\second}$, welche mit
dem Oszilloskop bei $U=450\Volt$ bestimmt wurde. Der hohe Fehler der
Totzeitmessung kommt dadurch zustande, dass der Fehler der Impulsrate $N$
als $\sqrt{N}$ angenommen wird.
Es ist zudem klar zu erkennen, dass die Impulsrate $N_{1+2}$ kleiner ist, als die Impulsrate von $N_1$ + $N_2$.

Bei der Ladung ist auffällig, dass deutlich mehr Elektronen gemessen wurden,
als regristriete Impulse, was durch die zahlreichen Stoßionisationen
der Townsend-Lawine und den erzeugten Photonen zu erklären ist. Zudem ist
dem Graph aus Abbildung \ref{fig:ladung} zu entnehmen, dass zwischen
der angelegten Spannung und der gemessenen Ladung ein lineares Verhältnis
besteht.
