Auffällig war, dass manche Öltröpfchen bei ausgeschaltetem Feld nicht
nach unten gefallen sind, sondern meistens in Ruhe verharrten, oder
teilweise sogar nach oben gewandert sind. Des Weiteren gilt
die Relation $2v_0 = v_\su{ab}-v_\su{auf}$ bei unseren Messwerten
leider nicht. Dennoch ergibt sich für die Elementarladung nur eine
Abweichung von $5,7\,\%$ und für die Avogadro-Konstante eine
Abweichung von $5,8\,\%$ zum Literaturwert $N_\su{A} = 6,022 \,\cdot 10^{23}\si[per-mode=fraction]{\per\mol}$\,\cite{Na}.
Die Abbildung \ref{fig:ladung} lässt ungefähr erkennen, dass die Elementarladungen
bei steigendem Radius der Öltröpfchen zunehmen. Dennoch sind starke Ausreißer dabei. 
Zudem sind die Abweichungen auf $r_\su{korr}$ und auf $q$ sehr
groß, was daran liegt, dass sich mehrere Fehler fortpflanzen und sich
so immer weiter aufaddieren, da es nicht möglich ist, die Fall- und
Steigzeiten der Öltröpfchen genau zu ermitteln. Zum einen wurden die
Zeiten per Hand gestoppt, und zum anderen wurde die durchlaufende Strecke
von 5 Kästchen nur mit den Augen beobachtet, wodurch die Länge der tatsächlich
durchlaufenen Strecke jedes mal leicht verändert wurde. Auch
die Auflösung des Mikroskops war teilweise zu gering, wodurch die Tröpfchen
verschwommen dargestellt wurden. Eine weitere Schwierigkeit stellte die
Beleuchtung der Tröpfchen dar, da manche Tröpfchen nicht exakt senkrecht gefallen
sind, wodurch sie dann nicht mehr angestrahlt wurden und somit nicht mehr sichtbar waren.

Hieraus lässt sich schließen, dass die Abweichungen der Elementarladung zwar
letztendlich nur sehr gering ausfallen, jedoch liegt dies daran, dass man bei der Berechnung
die exakte Elementar Ladung voraussetzt. Des Weiteren liegen bei den Zwischenrechnungen
enorm große Fehler vor, wodurch man zwar einen guten Wert für die Elementarladung erhält,
jedoch handelt es sich nicht um eine zuverlässige Methode, wenn man die Elementarladung herausfinden möchte, und sie vorher nicht bekannt ist.
