\subsection{Bestimmung der Brennweite}
Die Linse mit einer bekannten Brennweite von
\begin{equation*}
  f = 50\mm
\end{equation*}
wird zuerst überprüft. Die gemessenen Gegenstands- und Bildweiten werden in
Tabelle \ref{tab:brenn} aufgeführt.
\begin{table}[H]
  \centering
  \caption{Messdaten zur Bestimmung der Brennweite.}
  \label{tab:brenn}
  \begin{tabular}{cccc}
    \toprule
    \mc{2}{c}{bekannte Brennweite} & \mc{2}{c}{unbekannte Brennweite} \\
    $g\,/\cm$ & $b\,/\cm$ & $g\,/\cm$ & $b\,/\cm$ \\
    \midrule
    15 & 8,0 & 10 & 14,8 \\
    25 & 7,0 & 15 & 10,8 \\
    30 & 6,7 & 30 &  8,9 \\
    35 & 6,6 & 35 &  8,4 \\
    45 & 6,5 & 45 &  8,3 \\
    50 & 6,3 & 50 &  8,1 \\
    60 & 6,2 & 60 &  7,9 \\
    65 & 6,3 & 65 &  7,9 \\
    70 & 6,2 & 70 &  7,8 \\
    75 & 6,1 & 75 &  7,9 \\
    85 & 6,1 & 85 &  8,0 \\
    \bottomrule
  \end{tabular}
\end{table}
Mit diesen Daten und Formel \eqref{eqn:linse} lässt sich dann die Brennweite $f$
der Linse berechnen. Die berechnete Brennweite hat dann einen Wert von
\begin{equation*}
f = (5,6 \pm 0,1)\cm.
\end{equation*}
Der Fehler wird hier mittels Standardabweichung berechnet.
Desweiteren lässt sich die Brennweite der Linse graphisch darstellen. Hierfür
wird die Bildweite $b$ gegen die Gegenstandsweite aufgetragen und die jeweiligen
Punkte miteinander Verbunden. Mit dem an dem sich alle Geraden schneiden lässt
sich dann die Brennweite bestimmen. Dieses Prinzip ist in Abbildung \ref{fig:bekannt}
dargestellt.
\begin{figure}
  \centering
  \includegraphics[width=8cm]{bilder/bekannt.pdf}
  \caption{Graphische Bestimmung der Brennweite.}
  \label{fig:bekannt}
\end{figure}
Die Strecke vom x- beziehungsweise y-Achsenschnittpunkt zum Schnittpunkt der Geraden
entspricht dabei der Brennweite.
Diee Bestimmung der Brennweite lässt sich so Analog für eine Linse mit unbekannter
Brennweite wiederholen. Die Aufgenommenen Daten sind ebenfalls in Tabelle \ref{tab:brenn}
zu sehen. Mit Formel \eqref{eqn:linse} erhält man dann eine Brennweite von
\begin{equation*}
  f= (6,8 \pm 0,4) \cm.
\end{equation*}
Auch hier lässt sich die Brennweite graphisch mit Abbildung \ref{fig:unb}
bestimmen.
\begin{figure}
  \centering
  \includegraphics[width=8cm]{bilder/unbekannt.pdf}
  \caption{Graphische Bestimmung der Brennweite.}
  \label{fig:unb}
\end{figure}
Die so festgestellte Brennweite beträgt dann
\begin{equation*}
  f = 7\cm.
\end{equation*}
\subsection{Brennweitenbestimmung mit der Bessel-Methode}
Um die Brennweite mit der Bessel-Methode zu bestimmen, werden die Bild- und
Gegenstandsweiten $b$ und $g$ aufgenommen. Diese sind in Tabelle \ref{tab:bessel}
zu sehen.
\begin{table}[H]
  \centering
  \caption{Daten zur Bestimmung der Brennweite mit der Bessel-Methode}
  \label{tab:bessel}
  \begin{tabular}{ccc}
    \toprule
    $g\,/\cm$ & $b_1\,/\cm$ & $b_2\,/\cm$ \\
    \midrule
    57,2 & 11,4 & 67,2 \\
    57,3 & 11,5 & 62,7 \\
    57,4 & 11,6 & 57,6 \\
    57,8 & 11,8 & 52,2 \\
    58,3 & 12,2 & 46,7 \\
    58,8 & 12,5 & 41,2 \\
    59,6 & 13,3 & 35,7 \\
    60,4 & 14,1 & 29,6 \\
    62,5 & 16,7 & 22,5 \\
    60,8 & 14,8 & 27,2 \\
    \bottomrule
  \end{tabular}
\end{table}
Mit Formel \eqref{eqn:linse} lässt sich dann eine Brennweite von
\begin{equation*}
  f = (10,6 \pm 1,2)\cm
\end{equation*}
berechnen. Anschließend wird die gleiche Linse mit rotem und blauem Licht bestrahlt.
Die gemessenen Werte sind in der untenstehenden Tabelle \ref{tab:rb} zu sehen.
\begin{table}[H]
  \centering
  \caption{Bestimmung der Brennweite für rotes und blaues Licht.}
  \label{tab:rb}
  \begin{tabular}{cccccc}
    \toprule
    \mc{3}{c}{roter Filter} & \mc{3}{c}{blauer Filter} \\
    $g\,/\cm$ & $b_1\,/\cm$ & $b_2\,/\cm$ & $g\,/\cm$ & $b_1\,/\cm$ & $b_2\,/\cm$ \\
    \midrule
    58,1 & 11,9 & 57,4 & 58,2 & 11,8 & 57,8 \\
    63,2 & 11,8 & 62,6 & 63,4 & 11,6 & 62,9 \\
    68,4 & 11,6 & 67,8 & 68,7 & 11,3 & 68,0 \\
    52,6 & 12,1 & 52,1 & 53,1 & 11,9 & 52,4 \\
    47,5 & 12,5 & 46,9 & 47,9 & 12,1 & 47,1 \\
    \bottomrule
  \end{tabular}
\end{table}
Wird nun die Brennweite mit Formel \eqref{eqn:linse} berechnet, erhält man die Werte
\begin{align*}
  f_\su{rot} &= (9,9 \pm 0,06)\cm \\
  f_\su{blau}&= (9,7 \pm 0,06)\cm. 
\end{align*}
\subsection{Brennweitenbestimmung mit der Abbe-Methode}
\begin{table}
  \centering
  \caption{abbe}
  \label{tab:abbe}
  \begin{tabular}{ccc}
    \toprule
    $g\,/\cm$ & $b_1\,/\cm$ & $b_2\,/\cm$ \\
    \midrule
    17,6 & 87,4 & 36,0 \\
    17,3 & 92,7 & 35,0 \\
    17,3 & 85,2 & 36,7 \\
    16,0 & 84,0 & 36,3 \\
    18,4 & 76,6 & 37,8 \\
    18,7 & 73,8 & 37,7 \\
    19,5 & 70,5 & 39,5 \\
    21,0 & 66,5 & 39,3 \\
    21,2 & 63,8 & 40,0 \\
    25,1 & 54,9 & 42,9 \\
  \end{tabular}
\end{table}
