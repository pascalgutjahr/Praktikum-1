Für die erste Messung gibt der Hersteller eine Linsenbrennweite von
\begin{equation*}
  f = 5\cm
\end{equation*}
an. Der experimentell berechnete Wert liegt mit einer Brennweite von
\begin{equation*}
  f = (5,6 \pm 0,1)\cm
\end{equation*}
im Rahmen der Messungenauigkeit, da die relative Abweichung nur $10\,\%$ beträgt.
Die graphische Darstellung der Brennweite ist hier jedoch nicht möglich.
Bei der Bestimmung der unbekannten Brennweite liegen der berechnete und der
graphische Wert sehr nahe beieinander, was für gute Messergebnisse spricht. Die
Brennweite lässt sich mit dieser Methode demnach sehr gut bestimmen.
Für die Brennweitenbestimmung mit der Bessel-Methode wird vom Hersteller eine
Brennweite von
\begin{equation*}
  f = 10\cm
\end{equation*}
angegeben. Die gemessenen Werte und die relative Abweichung zu diesem Theoriewert
sind in Tabelle \ref{tab:rbes} zu sehen.
\begin{table}[H]
  \centering
  \small
  \caption{Relative Abweichungen bei der Bessel-Methode}
  \label{tab:rbes}
  \begin{tabular}{cccccc}
    \toprule
    \mc{2}{c}{Halogen-Lampe} & \mc{2}{c}{rotes Licht} & \mc{2}{c}{blaues Licht} \\
    $f\,/\cm$ & Abweichung$\,/\%$ &$f\,/\cm$ & Abweichung$\,/\%$ &
    $f\,/\cm$ & Abweichung$\,/\%$ \\
    \midrule
    $(10,6\pm1,1)$ & 6 & $(9,93\pm0,06)$ & 0 & $(9,74\pm0,06)$ & 2 \\
    \bottomrule
  \end{tabular}
\end{table}
Der Einfluss der sphärischen Abberation lässt sich an diesen Werten sehr gut
feststellen.

Bei der Brennweitenbestimmung nach Abbe ist auffällig, dass die Brennweiten stark
voneinander abweichen. Dies liegt daran, dass der Bereich, indem das Bild scharf erschien,
sehr groß war und man somit keine exakten Werte für $b$ und $g$ ablesen konnte.
