Für die Messung steht eine Schiene zur Verfügung, auf der die benötigten Elemente,
also Lichtquelle, Linse und Schirm, verschoben werden können. Als Lichtquelle
wird eine Halogenlampe verwendet. Bei dem Gegenstand, der verwendet wird, handelt
es sich um ein "Perl L". Außerdem stehen sowohl Linsen mit bekannter, als auch
mit unbekannter Brennweite zur Verfügung. Bei einer der Linsen mit unbekannter
Brennweite lässt sich diese durch befüllen mit Wasser verändern.

\subsection{Brennweitenbestimmung durch Gegenstands und Bildweite}
Zunächst soll die Brennweite mittels der Linsengleichung berechnet und mit den
Herstellerangaben verglichen werden. Hierfür werden die Halogenlampe, das Objekt
und eine Linse mit bekannter Brennweite auf die Schiene gestellt. Die Position
des Schirms wird bei gegebener Gegenstandsweite so eingestellt, dass das
"Perl L" scharf abgebildet wird. Diese Messung wird für 9 Gegenstandsweiten wiederholt.
Anschließend wird die Messung für eine Linse mit unbekannter Brennweite wiederholt.
\subsection{Brennweitenbestimmung mit der Bessel-Methode}
Die Lampe, das Objekt und der Schirm werden zusammen mit einer Linse mit
bekannter Brennweite auf der Schiene platziert. Die Linse wird dann bei festem
Abstand $e$ verschoben, bis das Bild scharf zu sehen ist. Nun wird eine zweite
Position gesucht, bei der das Bild wieder scharf ist. Auch hier werden 9 weitere
Abstände aufgenommen.
Danach wird die chromatische Abberation untersucht. Hierfür wird ein roter
und ein blauer Filter vor den Gegenstand gesetzt. Um sphärische Abberation zu
vermeiden ist es sinnvoll eine Irisblende zu verwenden.
\subsection{Brennweitenbestimmung mit der Abbe-Methode}
Die Halogenlampe, der Gegenstand und der Schirm werden auf der Schiene platziert.
Hinzu kommen eine Zerstreuungslinse mit einer Brennweite von $f = -100$ und eine
Sammellinse mit einer Brennweite von $f=100$. Die Elemente werden dicht zusammen
gestellt, damit ein gleichbleibender Abstand beibehalten werden kann. Der Referenzpunkt $A$
wird gewählt und der Abbildungsmaßstab $V$ sowie die Abstände $g'$ und $b'$
werden gemessen. Die Messung wird für 9 verschiedene Gegenstandsweiten wiederholt.
