Bei der Bestimmung der Kennlinien ist auffällig, dass ein Strom unter $\SI{2}{\ampere}$
bei unserer Apparatur und unseren Messinstrumenten nicht ausreicht, um eine vernünftige
Kennlinie zu ermitteln. Dies liegt an den geringen Strömen. Zudem ist zu erkennen, dass
die Kennlinien zu Beginn nicht mit $V^{3/2}$, sondern eher linear ansteigen. Das Langmuir-Schottkysche
Raumladungsgebiet ist jedoch zu erahnen und wird von uns bis ca. $U = \SI{60}{\volt}$
festgelegt. Die Proportionalität $I \propto V^{3/2}$ ist bei uns nicht erfüllt,
wir erhalten $I \propto V^{0,07}$. Diese Abweichung ist primär durch systematische Fehler
zu erklären. 

Dennoch
sind die Sättigungswerte von allen fünf Kennlinien sehr gut abzulesen. Zudem steigt
$I_\su{s}$ bei steigender Spannung und steigener Heizleistung an, wodurch die
zuvor erwähnte Theorie bestätigt wird.

Die bestimmten Werte für die Temperatur erscheinen realistisch, da die Temperatur bei
steigender Heizleistung weiter ansteigt. Genau kann die Temperatur jedoch nicht überprüft
werden, da wir die Temperatur des Drahtes nicht messen konnten.

Die Abweichung der Austrittsarbeit von Wolfram mit $N_\su{A,lit} = \SI{4,54}{\electronvolt}$
\cite{na} beträgt $69,6\,\%$. Diese enorme Abweichung hängt auch wieder mit der Messungenauigkeit des
Stromes zusammen.
