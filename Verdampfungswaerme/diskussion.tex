Für die Verdampfungswärme von Wasser wurde L = 41,94 $\si{\kilo \joule \per \mol}$
gemessen. Verglichen mit dem Literaturwert von L = 40,8 $\si{\kilo \joule \per \mol}$
\cite{chemie} ergibt sich daraus eine relative Fehlerabweichung nach der Formel
\begin{equation}
  \delta L_{rel} = \frac{L_{gemessen}-L_{Literatur}}{L_{Literatur}}
\end{equation}
von nur 2,8 \%.

Diese Messung ist somit sehr gut gelungen, was daran liegt, dass die
Werte im Abstand von $1 \si{\celsius}$ aufgenommen wurden, wodurch eine Vielzahl
an Messwerten zustande kam. Somit wurden kleine Messungenauigkeiten herausgemittelt.
Die Ungenauigkeiten kommen beim Ablesen vor, da man nie genau erkennen kann, welche
exakte Temperatur das Thermometer gerade anzeigt. Hinzu kommt, dass die Apparaturen
nicht völlig evakuiert werden können, da die Dichtungen nicht perfekt arbeiten. Bei
dem ersten Versuch wurde bis auf 70 $\si{\milli \bar}$ evakuiert. Des Weiteren
wurde die Kühlung manuell geregelt, wobei aufgepasst werden musste, dass sie weder
zu stark, noch zu schwach eingestellt ist. Dies genau abzustimmen ist nicht
allzu einfach. Zudem kann nicht direkt davon ausgegangen werden, dass die Temperatur
des flüssigen Wassers gleich der Temperatur des Gases ist, da auch hier leichte
Unterschiede vorliegen.

Zur Bestimmung der Verdampfungswärme in Abhängigkeit der Temperatur wurde die Messung
bis 15 bar genutzt. Bei dieser Messung wurde angenommen, dass die Temperatur in
der "Stahlröhre" angenähert gleich der Temperatur des Gases ist. Hinzu kommt, dass
sich nicht überprüfen ließ, ob das Thermometer an der Heizspule oder an der Gasröhre
anliegt. Dies erklärt die anfängliche negative Steigung des Graphen (\ref{fig:plus}).
Im Bereich von 425 $\si{\kelvin}$ stimmen die Messwerte im Rahmen der Messungenauigkeiten
mit denen der Literaturwerte überein.
% ???????
