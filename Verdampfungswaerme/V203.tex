\documentclass[captions=tableheading,
  bibliography=totoc,
  titlepage=firstiscover
  ]{scrartcl}
\usepackage{scrhack}
\usepackage[a4paper,top=1.5cm,left=3.5cm,right=1.75cm,bottom=2cm,bindingoffset=5mm]{geometry}

\usepackage[aux]{rerunfilecheck}

\usepackage{multirow}

\usepackage{polyglossia}
\setmainlanguage{german}

\usepackage{amsmath}
\usepackage{amssymb}
\usepackage{mathtools}
\usepackage{xfrac}
\usepackage{fontspec}
\usepackage{biblatex}
\addbibresource{lit.bib}  %nach polyglossia

\usepackage[
  math-style=ISO,
  bold-style=ISO,
  sans-style=italic,
  nabla=upright,
  partial=upright,
]{unicode-math}

\usepackage[
  locale=DE,
  separate-uncertainty=true,
  per-mode=symbol-or-fraction,
]{siunitx}

\usepackage[section, below]{placeins}
\usepackage[
  labelfont=bf,        % Tabelle x: Abbildung y: ist jetzt fett
  font=small,          % Schrift etwas kleiner als Dokument
%  width=0.9\textwidth, % maximale Breite einer Caption schmaler
  format=plain,
  indention=1em, % Abbildung sticht links etwas hervor
]{caption}
\usepackage{graphicx}
\usepackage{wrapfig}
\usepackage{grffile}
\usepackage{subcaption}

\usepackage{booktabs}
\usepackage{float}
%\restylefloat{figure}
\floatplacement{figure}{htbp}
\floatplacement{table}{htbp}
\usepackage{rotating}
\usepackage{chemmacros}

\usepackage[unicode]{hyperref}
\usepackage{bookmark}
\usepackage{microtype}
\usepackage[
version=4,
math-greek=default,
text-greek=default,
]{mhchem}


\newcommand{\be}{\begin{equation}} %Kurzbefehl für \begin{equation}
\newcommand{\ee}{\end{equation}} %Kurzbefehl für \end{equation}
\newcommand{\su}{\symup}
\newcommand{\mOhm}{\,\si{\milli\ohm}}
\newcommand{\Ohm}{\,\si{\ohm}}
\newcommand{\kOhm}{\,\si{\kilo\ohm}}
\newcommand{\Amp}{\,\si{\ampere}}
\newcommand{\Volt}{\,\si{\volt}}
\newcommand{\mA}{\,\si{\milli\ampere}}
\newcommand{\mV}{\,\si{\milli\volt}}
\newcommand{\mt}{\,\si{\meter}}
\newcommand{\pmt}{\,\si{\pico\meter}}
\newcommand{\fmt}{\,\si{\femto\meter}}
\newcommand{\qm}{\,\si{\square\meter}}
\newcommand{\cum}{\,\si{\cubic\meter}}
\newcommand{\cm}{\,\si{\centi\meter}}
\newcommand{\ms}{\,\si{\micro\second}}
\newcommand{\nm}{\,\si{\nano\meter}}
\newcommand{\qcm}{\,\si{\square\centi\meter}}
\newcommand{\ccm}{\,\si{\cubic\centi\meter}}
\newcommand{\Watt}{\,\si{\watt}}
\newcommand{\mm}{\,\si{\milli\meter}}
\newcommand{\gr}{\,\si{\gram}}
\newcommand{\kg}{\,\si{\kilo\gram}}
\newcommand{\J}{\,\si{\joule}}
\newcommand{\kJ}{\,\si{\kilo\joule}}
\newcommand{\Hz}{\,\si{\hertz}}
\newcommand{\kHz}{\,\si{\kilo\hertz}}
\newcommand{\MHz}{\,\si{\mega\hertz}}
\newcommand{\GHz}{\,\si{\giga\hertz}}
\newcommand{\acc}{\,\si{\meter\per\square\second}}
\newcommand{\vel}{\,\si{\meter\per\second}}
\newcommand{\kmh}{\,\si{\kilo\meter\per\hour}}
\newcommand{\sek}{\,\si{\second}}
\newcommand{\New}{\,\si{\newton}}
\newcommand{\Nm}{\,\si{\newton\meter}}
\newcommand{\Kel}{\,\si{\kelvin}}
\newcommand{\Cd}{\,\si{\candela}}
\newcommand{\Hen}{\,\si{\henry}}
\newcommand{\Far}{\,\si{\farad}}
\newcommand{\pas}{\,\si{\pascal}}
\newcommand{\Dichte}{\,\si{\kilo\gram\per\cubic\meter}}
\newcommand{\kVolt}{\,\si{\kilo\volt}}
\newcommand{\MVolt}{\,\si{\mega\vot}}
\newcommand{\keV}{\,\si{\kilo\electronvolt}}
\newcommand{\MeV}{\,\si{\mega\electronvolt}}
\newcommand{\dgr}{\,\si{\degree}}
\newcommand{\eV}{\,\si{\electronvolt}}
\newcommand{\nA}{\,\si{\nano\ampere}}
\newcommand{\lt}{\,\si{\litre}}
\newcommand{\ml}{\,\si{\milli\litre}}
% \newcommand{\min}{\,\si{\minute}}
\newcommand{\rt}{\right}
\newcommand{\lf}{\left}


\title{V203 - Verdampfungswärme und Dampfdruckkurve}
\author{Julian Jung \\ julian.jung@tu-dortmund.de
  \and Pascal Gutjahr \\ pascal.gutjahr@tu-dortmund.de}
  \date{Durchführung: 11.11.2016
  \hspace{3em}
  Abgabe: 18.11.2016}
  \begin{document}
\maketitle
\newpage
\tableofcontents
\newpage
\section{Zielsetzung}
Ziel des Versuchs ist es die Verdampfungswärme L von Wasser zu Untersuchen.
Dafür wird der Druck und die Temperatur untersucht. Im ersten Teil des Versuchs
wird die Verdampfungswärme für einen niedrigen Druck $(p\le 1$ bar) untersucht. In
diesem Bereich ist L beinahe konstant. In einem zweiten Versuch wird der Zusammenhang
zwischen Druck und Temperatur bei 1 bar < $p$ < 15 bar untersucht.
\section{Theorie}
Ein RC-Schwingkreis besteht in der Regel aus einem Kondensator mit der
Kapazität $C$ und diner Spule, die eine Induktivität $L$ liefert.
Diese beiden Bauteile dienen hier als Energiespeicher.
In einem idealen Schwingkreis wird eine einmal eingespeicherte Energie
immer zwischen beiden genannten Elementen ausgetauscht.
Dieser Austausch kommt durch die Entladung des Kondensators zustande, bei welcher
in der Spule ein Magnetfeld aufgebaut wird. Wird dieses Magnetfeld abgebaut,
lädt sich der Kondensator auf und der Vorgang beginnt von neuem.
Wird ein gedämpfter, also ein realer Schwingkreis, betrachtet, ist neben der
Spule und dem Kondensator noch ein ohmscher Widerstand $R$ vorhanden.
Über diesen wird die Energie in Wärme umgewandelt und somit aus dem
Schwingkreis entfernt. $R$ ist demnach ein Dämpfungsfaktor.
Der schematische Aufbau eines $RCL$-Schwingkreises ist in Abbildung \ref{fig:rcl}
zu sehen.
\begin{figure}[H]
  \centering
  \includegraphics{Bilder/RCL.JPG}
  \caption{Schematischer Aufbau eines $RCL$-Kreises\cite{354}}
  \label{fig:rcl}
\end{figure}
Die Werte von $U_\su{R}, U_\su{C}$ und $U_\su{L}$ bezeichnen hierbei die Spannung
die über dem jeweiligen Bauteil abfällt. Gemäß des 2. Kirchhoffschen Gesetzes
gilt:
\begin{equation}
  U_\su{R}(t) + U_\su{C}(t) + U_\su{L}(t) = 0.
  \label{eqn:kirch}
\end{equation}
Um eine Differentialgleichung der 2. Ordnung zu erhalten, wird die Spannung
durch den Strom $I$ ausgedrückt. Somit erhält man:
\begin{align*}
  U_\su{R}(t) &= RI(t) \\
  U_\su{C}(t) &= \frac{Q(t)}{C} \text{mit} I = Q \\
  U_\su{L}(t) &= LI ,
\end{align*}
was zu der Differentialgleichung
\begin{equation}
  I(t) + \frac{R}{L}I(t) + \frac{1}{LC}I(t) = 0 .
\end{equation}
Mit dem entsprechendem Ansatz ergibt sich für die Gleichung die Lösung
\begin{equation}
  I(t)=\su{e}^{-2\pi\mu t}(A_1\su{e}^{i2\pi\nu t} + A_2\su{e}^{-2i\pi\nu t}).
  \label{eqn:dgl}
\end{equation}
Die Parameter $\mu$ und $\nu$ sind dabei definiert als:
\begin{align*}
  \mu &\coloneqq\frac{R}{4\pi L} \\
  \nu &\coloneqq\frac{1}{2\pi}\sqrt{\frac{1}{LC}-\frac{R^2}{4\L^2}}
\end{align*}
Für $\nu$ muss aufgrund der Wurzel eine Fallunterscheidung gemacht werden, da
$\nu$ sowohl reel, als auch rein imaginär sein kann.
Für den reelen Fall muss
\begin{equation*}
  \frac{1}{LC} > \frac{R^2}{4L^2}
\end{equation*}
gelten. Unter diesen Voraussetzungen lässt sich Formel (\ref{eqn:dgl}) mit der
Eulerschen Formel zu
\begin{equation}
  I(t)=A_\su{0}\su{e}^{-2\pi\mu t}\cos(2\pi\nu t+\mu)
\end{equation}
umschreiben. Hier entsteht eine gedämpfte Schwingung, welche bei $t\rightarrow\inf$
gegen Null strebt. Die Abklingdauer dieser Schwingung wird dann durch
\begin{equation}
  T_\su{ex}\coloneqq\frac{1}{2\pi\mu}
\end{equation}
definiert.

für den Fall dass $\nu$ imaginär ist, also der Fall
\begin{equation*}
  \frac{1}{LC} < \frac{R^2}{4L^2}
\end{equation*}
eintritt, lässt sich Formel (\ref{eqn:dgl}) zu
\begin{equation}
  I(t)\propto\su{e}^{-(2\pi\mu -\su{i}2\pi\nu)t}
\end{equation}
umschreiben.
Da sich i und $\nu$ zu einem insgesamt reelen Exponenten verrechnen lassen, fällt
die Amplitude des Stroms exponentiell ab. Die Amplitude hat dabei, je nach
Anfangsbedinungen, einen oder keinen Extremwert. Am schnellsten fällt die
Amplitude, wenn der Spezialfall
\begin{equation*}
  \frac{1}{LC} = \frac{R^2}{4L^2}
\end{equation*}
eintritt. Dieser Fall wird aperiodischer Grenzfall genannt und ist in der
Abbildung \ref{fig:agf} als gestrichelte Linie dargestellt.
\begin{figure}[h]
  \centering
  \includegraphics{Bilder/aperiod.JPG}
  \caption{Zeitverlauf des Stroms mit aperiodischer Dämpfung\cite{354}}
  \label{fig:agf}
\end{figure}
\\
Abblidung (\ref{fig:angeregt}) zeigt einen $RCL$-Schwingkreis, welcher von
außen angeregt wird. Die Anregung erfolgt hier durch eine Wechselstromquelle.
\begin{figure}[h]
  \centering
  \includegraphics{Bilder/angeregt.JPG}
  \caption{Schaltbild eines angeregten$RCL$-Schwingkreises}
  \label{fig:angeregt}
\end{figure}
Nach einer kurzen Einschwingzeit nimmt der Schwingkreis die Frequenz der
Wechselstromquelle an. Die zu Beginn aufgestellte Differentialgleichung aus
Formel (\ref{eqn:dgl}) wird nun inhomogen und lautet:
\begin{equation}
  LCU_\su{C}(t) + RCU_\su{c}(t) + U_\su{c}(t) =U_0\su{e}^{\su{i}\omega t}.
  \label{eqn:inh}
\end{equation}
Die Lösung für die Spannung in Abhängigkeit der Zeit berechnet sich dann mit:
\begin{equation}
  U(t)=\frac{U_0(1-LC\omega^2-\su{i}\omega RC)}{(1-LC\omega^2)^2+\omega^2 R^2 C^2}.
\end{equation}
Für die Phasenverschiebung zur Erregerspannung ergibt sich durch den Vergleich
von Imaginär- und Realteil ergibt sich: %Umformulieren um ':' zu vermeiden!!
\begin{equation}
  \varphi(w)=\arctan\biggr(\frac{\su{Im}(U)}{\su{Re}(U)}\biggl)
  =\arctan\biggr(\frac{-\omega RC}{1-LC\omega^2}\biggl)
\end{equation}
Für die Frequenzen $\omega_1$ und $\omega_2$ gilt bei einer Phasenverschiebung
von $\frac{\pi}{4}$ beziehungsweise $\frac{3\pi}{4}$
\begin{equation}
  \omega_\su{1,2}=\pm\frac{R}{2L}+\sqrt{\frac{R^2}{4L^2}+\frac{1}{LC}}.
\end{equation}

Die Spannung kann zusätzlich mit
\begin{equation}
  U_\su{C}(\omega)=\frac{U_0}{\sqrt{(1-LC\omega^2)^2+\omega^2R^rC^2}}
\end{equation}
in Abhängigkeit von der Frequenz $\omega$ angegeben werden.
Hierbei zeigt sich, dass die Spannungsamplitude für sehr hohe Frequenzen gegen
Null strebt, während sie für kleine Frequenzen gegen $U_0$ strebt.
Die Resonanzfrequenz
\begin{equation}
  \omega_\su{res}=\sqrt{\frac{1}{LC}-\frac{R^2}{2L^2}}
\end{equation}
beschreibt den Zustand, bei dem $U_\su{C}$ einen Maximalwert erreicht der auch
größer als $U_\su{0}$ sein kann.
Von schwacher Dämpfung wird gesprochen, wenn
\begin{equation}
  \frac{R^2}{2L^2} \ll \frac{1}{LC}
\end{equation}
gilt. In diesem Zustand gilt $\omega_\su{res}\approx\omega_0$, wobei $\omega_0$
die Kreisfrequenz der ungedämpften Schwingung ist und den Wert
\begin{equation}
  \omega_0 =\sqrt{\frac{1}{LC}}
\end{equation}
annimmt. Das Maximum der Kondensatorspannung ist dann um den Faktor
\begin{equation}
  q=\frac{1}{\omega_0RC}
\end{equation}
größer als die Erregerspannung. Der Faktor $q$ wird als Güte des Schwingkreises
bezeichnet.

\section{Aufbau und Durchführung}
Im ersten Schritt wird die Leerlaufspannung $U_\symup{0}$ einer Monozelle
gemessen. Hierfür werden die Anschlüsse des Voltmeters direkt mit den Anschlüssen
der Monozelle verbunden. Die angezeigte Spannung und der Innenwiderstand des
Voltmeters $R_\symup{i,V}$ werden notiert.
Nach dieser Messung werden ein Amperemeter und ein regelbarer Widerstand
$R_\symup{a}$ wie in Abbildung \ref{fig:schlt1} mit der Monozelle in Reihe geschaltet.
\begin{figure}[H]
  \centering
  \begin{subfigure}{0.48\textwidth}
    \centering
    \includegraphics[width=4cm]{bilder/sinrecht.jpg}
    \caption{Messreihe 1}
    \label{fig:schlt1}
  \end{subfigure}
  \begin{subfigure}{0.48\textwidth}
    \centering
    \includegraphics[width=4cm]{bilder/gegenspannung.jpg}
    \caption{Messreihe 2}
    \label{fig:schlt2}
  \end{subfigure}
  \caption{Schaltbilder \cite{301}}
  \label{fig:schlt}
\end{figure}
Das Voltmeter bleibt parallel geschaltet um den Spannungsverlauf über der
Monozelle zu messen. Der Widerstand wird hierbei von $(0-50)\Ohm$
variiert. Die Werte für $U_\symup{k}$ und $I$ werden notiert.
Abbildung \ref{fig:schlt2} zeigt die zweite Messreihe. Hier wird eine
Gegenspannung an die Monozelle gelegt. Die Gegenspannung ist $2\,\symup{\si{\volt}}$
größer als $U_\symup{0}$. Der Widerstand wird erneut von $(0-50)\,\symup{\si{\ohm}}$
variiert und die Werte für Strom und Spannung werden notiert.
Nach dieser Messung wird erneut die Schaltung aus \ref{fig:schlt1} aufgebaut.
Die Monozelle wird jedoch durch einen RC-Generator ersetzt. Dieser soll zunächst
eine Rechteckspannung liefern. Der Widerstand wird nun zwischen $(20-250)\,\symup{\si{\ohm}}$
variiert. Die Werte für $I$ und $U_\symup{k}$ werden an den Messgeräten
abgelesen und notiert.
Die Messung wird mit einer Sinusspannung und einem Widerstand von
$(0,1-5)\,\symup{\si{\kilo\ohm}}$ wiederholt.
\newpage

\newpage %muss hier oder in Aufbau.tex
\section{Auswertung}
\subsection{p < 1 bar}
Um die Verdampfungswärme L zu berechnen, wird die Formel
\begin{equation*}
  \ln \bigl(\frac{p}{p_0}\bigr)=-\frac{\symup{L}}{\symup{R}} \cdot \frac{1}{\symup{T}}
  %schon verwendet?
\end{equation*}
verwendet. Hierbei ist $p_0$ der Umgebungsdruck und beträgt
\begin{equation*}
  p_0=995\cdot 10^2 \si{\pascal}.
\end{equation*}
Die in Tabelle\ref{tab:data1} gemessenen Werte für den Druck werden logarithmiert und in einem
Diagramm gegen den Kehrwert der absoluten Temperatur aufgetragen.
  \begin{table}
    \centering
    \caption{Messung bis 1 bar}
    \label{tab:data1}
    \begin{tabular}{c c c c}
      \toprule
      T/°C & P/bar & T/°C & P/bar
      \\
      \midrule
      52    &   137  &  76  &  400  \\
      53    &   146  &  77  &  417  \\
      54    &   153  &  78  &  439  \\
      55    &   160  &  79  &  454  \\
      56    &   167  &  80  &  474  \\
      57    &   174  &  81  &  490  \\
      58    &   183  &  82  &  515  \\
      59    &   191  &  83  &  532  \\
      60    &   200  &  84  &  555  \\
      61    &   209  &  85  &  577  \\
      62    &   219  &  86  &  600  \\
      63    &   229  &  87  &  625  \\
      64    &   239  &  88  &  650  \\
      65    &   249  &  89  &  666  \\
      66    &   261  &  90  &  702  \\
      67    &   271  &  91  &  726  \\
      68    &   286  &  92  &  756  \\
      69    &   299  &  93  &  780  \\
      70    &   311  &  94  &  807  \\
      71    &   325  &  95  &  837  \\
      72    &   339  &  96  &  881  \\
      73    &   352  &  97  &  912  \\
      74    &   369  &  98  &  947  \\
      75    &   384  &  99  &  988  \\
      %Wert für 100°C fehlt wegen Formatierung
      \bottomrule
    \end{tabular}
  \end{table}

\begin{figure}[H]
  \centering
  \includegraphics[height=9cm , width=13.5cm]{fit1.jpg}
  \caption{fit1}
  \label{fig:fit1}
  \end{figure}
Die Parameter der linearen Regression werden mit Python 3.5.2 berechnet und
lauten:
\begin{align*}
  a =& (-5,044 \pm 0,006) \cdot 10^3 \si{\kelvin} \\
  b =& (13,54 \pm 0,02)
\end{align*}
Unter Verwendung von Formel \eqref{eq:4} und der allgemeinen Geradengleichung
lässt sich L mit
\begin{equation}
  a=-\frac{L}{\symup{R}} \iff L = -a \cdot \symup{R}
\end{equation}
darstellen. Der dazugehörige Fehler wird über die Gaußsche Fehlerfortpflanzung
berechnet und liefert:
\begin{equation}
  \Delta f = \sqrt{-\symup{R}^2 \cdot \Delta a^2}.
\end{equation}
Wobei R die allgemeine Gaskontante und $\Delta a$ der Fehler von $a$ ist.
Somit ergibt sich:
\begin{equation*}
  L=(41,94 \pm 0,05)\cdot 10^3 \frac{\si{\joule}}{\si{\mol}}.
\end{equation*}
% \newpage
Weiter soll die äußere Verdampfungswärme $L_a$, also die Energie, welche
benötigt wird um das Volumen der Flüssigkeit auf das Volumen des Gases
zu bringen, berechnet werden. Hierbei wird eine Volumenarbeit $W=pV$ geleistet.
Über gleichsetzen der idealen Gasgleichung \eqref{eq:ideal} mit der
Volumenarbeit, lässt sich $L_a$ für eine Temperatur von $T=373 \si{\kelvin}$
berechnen:
\begin{align*}
  L_a =& W =pV=RT \\
      =& 3,10 \cdot 10^3 \si{\joule \per \mol}
\end{align*}
Die innere Energie $L_i$ muss aufgebracht werden, um die molekularen
Anziehungskräfte zu überwinden. Diese ist
\begin{align*}
  L_i=&L-L_a \\
  =& (38,84 \pm 0,05)\cdot 10^3 \si{\joule\per\mol}.
\end{align*}
 Die innere Energie pro Molekül erhält man durch die Division mit der Avogadrokonstanten
 $\symup{N_A}=6,022 \cdot 10^{23} \frac{1}{\si{\mol}}$.
 Dieses Ergebnis wird der Anschaulichkeit halber in Elektronenvolt angegeben
 ($1\si{\electronvolt}=1,602\cdot 10{{-19}}\si{\joule})$.
 Der Fehler wird erneut über die Gaußsche Fehlerfortpflanzung berechnet:
 \begin{equation*}
   \Delta f=\sqrt{\frac{1}{\symup N_A^2}\cdot 0,050^2}.
 \end{equation*}
 Somit:
 \begin{align*}
   L_i =&(4,03\cdot 10^{-4} \pm 1\cdot 10^{-6})\ \si{\electronvolt} \\
   L_i =&(0,403 \pm 0,001) \ \si{\milli\electronvolt}
 \end{align*}
\subsection{p > 1 bar}
Um die Verdampfungswärme, bei höheren Drücken, in Abhängigkeit der Temperatur zu
ermitteln, benötigen wir die Clausis-Clapeyronsche Gleichung (\ref{eq:clausius}).
Diese wird nach der Verdampfungswärme L umgeformt:
\begin{equation}
  L = (V_D - V_F) \;T\; \frac{dp}{dT} \label{eq:8}
\end{equation}
$V_F$ kann hierbei weiterhin vernachlässigt werden, $V_D$ kann nun jedoch nicht mehr
mit der allgemeinen Gaslgeichung (\ref{eq:ideal}) bestimmt werden.
Eine Gute Näherung ist durch die Gleichung
\begin{align}
  \biggr(p + \frac{A}{V^2}\biggl) \; V = RT \qquad mit \qquad A = 0,9 \frac{\si{\joule \meter^3}}
  {\si{\mol^2}}
\end{align}
gegeben.
Diese Gleichung wird dann nach $V_D$ umgeformt:
\begin{equation}
  V_D = \frac{RT}{2p} \pm \sqrt{\biggr(\frac{RT}{2p}\biggl)^2-\frac{A}{p}}
\end{equation}
Anschließend wird dieser Term in (\ref{eq:8}) eingesetzt:
\begin{equation}
  L = \Biggr(\; \frac{RT}{2p} \pm \sqrt{\biggr(\frac{RT}{2p}\biggl)^2-\frac{A}{p}} \; \Biggl)
  \; T \; \frac{dp}{dT}
\end{equation}
Für die Lösung von $\frac{dp}{dT}$ wird ein Ausgleichpolynom 3. Grades verwendet,
\begin{align}
  p(T) &= a \cdot T^3 + b \cdot T^2 + c \cdot T + d  \label{eq:pt}\\
  \frac{dp}{dT} &= 3a \cdot T^2 + 2b \cdot T +    \label{eq:dp}
\end{align}
mit den Parametern:
\begin{align*}
  a &= (0,009 \pm 0,001) \; \si{\frac{\pascal}{\kelvin^3}} \\
  b &= (-11 \pm 1) \; \si{\frac{\pascal}{\kelvin^2}} \\
  c &= (4,3 \pm 0,6) \; \cdot 10^3 \; \si{\frac{\pascal}{\kelvin}} \\
  d &= (-5,6 \pm 0,8)\; \cdot 10^5 \; \si{\pascal}
\end{align*}

\begin{table}
  \centering
  \caption{Messung bis 15 bar}
  \label{tab:data2}
\begin{tabular}{c c c c}
  \toprule
  T/\si{\celsius} &   P/\si{\bar}  &   T/\si{\celsius} &   P/\si{\bar} \\
  \midrule
  100  &   1.24   &   132  &   3.69  \\
  102  &   1.33   &   134  &   3.94  \\
  104  &   1.43   &   136  &   4.20  \\
  106  &   1.54   &   138  &   4.48  \\
  108  &   1.66   &   140  &   4.78  \\
  110  &   1.78   &   142  &   5.11  \\
  112  &   1.90   &   144  &   5.48  \\
  114  &   2.04   &   146  &   5.87  \\
  116  &   2.18   &   148  &   6.31  \\
  118  &   2.34   &   150  &   6.79  \\
  120  &   2.50   &   152  &   7.36  \\
  122  &   2.67   &   154  &   8.03  \\
  124  &   2.85   &   156  &   8.81  \\
  126  &   3.04   &   158  &   9.86  \\
  128  &   3.25   &   160  &  11.26  \\
  130  &   3.47   &   162  &  13.46  \\
  \bottomrule
  \end{tabular}
\end{table}

\begin{figure}[H] % ein großes H steht dafür, dass das Objekt auf jeden Fall den Platz hält
  \centering
  \includegraphics[height=8cm]{15bar.jpg}
  \caption{Ausgleichspolynom 3. Grades}
  \label{fig:15}
\end{figure}

Werden nun (\ref{eq:pt}) und (\ref{eq:dp}) in (\ref{eq:8}) eingesetzt, so erhält man
L in Abhängigkeit von T:
\begin{align*}
  L(T) = \left[\frac{RT}{2 (aT^3+bT^2+cT+d)} \; \pm \sqrt{\biggl(\frac{RT}{2(aT^3+bT^2+cT+d)}\biggr)^2
  -\frac{A}{aT^3+bT^2+cT+d}}\; \right] \\
  \;\cdot T \; (3aT^2+2bT+c)
\end{align*}
Die Lösung mit der negativen Wurzel ergibt hier im Sachzusammenhang jedoch kein Sinn,
da es keine negative Verdampfungswärme gibt. Somit bleibt nur noch eine Lösung übrig:
\begin{figure}[H]
  \centering
  \includegraphics[height=5.5cm]{Lplus.jpg}
  \caption{Der Wurzelterm wird addiert}
  \label{fig:plus}
\end{figure}

\section{Diskussion}
%\input{Diskussion.tex}
\section{Literaturverzeichnis}
\end{document}
