\subsection{p < 1 bar}
<<<<<<< HEAD
Um die Verdampfungswärme L zu berechnen, wird die Formel
\begin{equation*}
  \ln \bigl(\frac{p}{p_0}\bigr)=-\frac{L}{R} \cdot \frac{1}{T}
  %schon verwendet?
\end{equation*}
verwendet. Hierbei ist $p_0$ der Umgebungsdruch und beträgt
\begin{equation*}
  p_0=995\cdot 10^2 \si{\pascal}.
\end{equation*}
Die in Tabelle\ref{tab:data1} gemessenen Werte für den Druck werden logarithmiert und in eien
Diagramm gegen den Kehrwert der absoluten Temperatur aufgetragen \\
   \begin{table}
    \centering
    \caption{Messung bis 1 bar}
    \label{tab:data1}
    \begin{tabular}{c c c c}
      \toprule
      T/°C & P/bar & T/°C & P/bar
      \\
      \midrule
      52    &   137  &  76  &  400  \\
      53    &   146  &  77  &  417  \\
      54    &   153  &  78  &  439  \\
      55    &   160  &  79  &  454  \\
      56    &   167  &  80  &  474  \\
      57    &   174  &  81  &  490  \\
      58    &   183  &  82  &  515  \\
      59    &   191  &  83  &  532  \\
      60    &   200  &  84  &  555  \\
      61    &   209  &  85  &  577  \\
      62    &   219  &  86  &  600  \\
      63    &   229  &  87  &  625  \\
      64    &   239  &  88  &  650  \\
      65    &   249  &  89  &  666  \\
      66    &   261  &  90  &  702  \\
      67    &   271  &  91  &  726  \\
      68    &   286  &  92  &  756  \\
      69    &   299  &  93  &  780  \\
      70    &   311  &  94  &  807  \\
      71    &   325  &  95  &  837  \\
      72    &   339  &  96  &  881  \\
      73    &   352  &  97  &  912  \\
      74    &   369  &  98  &  947  \\
      75    &   384  &  99  &  988  \\
      %Wert für 100°C fehlt wegen Formatierung
      \bottomrule
    \end{tabular}
  \end{table}

\begin{figure}
  \centering
  \includegraphics[height=9cm , width=13.5cm]{fit1.jpg}
  \caption{fit1}
  \label{fig:fit1}
  \end{figure}
  \\
Die Parameter der linearen Regression werden mit Python 3.5.2 berechnet und
lauten:
\begin{align*}
  a=(-5,044 \pm 0,006) \cdot 10^3 \si{\kelvin}
\end{align*}
=======
Die gemessenen Werte für Drücke unter 1 bar sind in Tabelle 1 zu sehen.
  \begin{table}
    \centering
    \caption{Messung bis 1 bar}
    \label{tab:data1}
    \begin{tabular}{c c c c}
      \toprule
      T/°C & P/bar & T/°C & P/bar
      \\
      \midrule
      52    &   137  &  76  &  400  \\
      53    &   146  &  77  &  417  \\
      54    &   153  &  78  &  439  \\
      55    &   160  &  79  &  454  \\
      56    &   167  &  80  &  474  \\
      57    &   174  &  81  &  490  \\
      58    &   183  &  82  &  515  \\
      59    &   191  &  83  &  532  \\
      60    &   200  &  84  &  555  \\
      61    &   209  &  85  &  577  \\
      62    &   219  &  86  &  600  \\
      63    &   229  &  87  &  625  \\
      64    &   239  &  88  &  650  \\
      65    &   249  &  89  &  666  \\
      66    &   261  &  90  &  702  \\
      67    &   271  &  91  &  726  \\
      68    &   286  &  92  &  756  \\
      69    &   299  &  93  &  780  \\
      70    &   311  &  94  &  807  \\
      71    &   325  &  95  &  837  \\
      72    &   339  &  96  &  881  \\
      73    &   352  &  97  &  912  \\
      74    &   369  &  98  &  947  \\
      75    &   384  &  99  &  988  \\
      %Wert für 100°C fehlt wegen Formatierung
      \bottomrule
    \end{tabular}
  \end{table}

\\
Die abgelesenen Drücke werden weiter logarithmiert und in einem Diagramm gegen
die absolute reziproke Temperatur aufgetragen. \\
% \input{fit1.pdf} \\
>>>>>>> c993d95d0c42eae2c26e5710db227e839707d41a
\newpage
Hier kann man dann mit Tabelle 2 weiter machen:
\begin{table}
  \centering
  \caption{Messung bis 15 bar}
  \label{tab:data2}
\begin{tabular}{c c c c}
  \toprule
  T/\si{\celsius} &   P/\si{\bar}  &   T/\si{\celsius} &   P/\si{\bar} \\
  \midrule
  100  &   1.24   &   132  &   3.69  \\
  102  &   1.33   &   134  &   3.94  \\
  104  &   1.43   &   136  &   4.20  \\
  106  &   1.54   &   138  &   4.48  \\
  108  &   1.66   &   140  &   4.78  \\
  110  &   1.78   &   142  &   5.11  \\
  112  &   1.90   &   144  &   5.48  \\
  114  &   2.04   &   146  &   5.87  \\
  116  &   2.18   &   148  &   6.31  \\
  118  &   2.34   &   150  &   6.79  \\
  120  &   2.50   &   152  &   7.36  \\
  122  &   2.67   &   154  &   8.03  \\
  124  &   2.85   &   156  &   8.81  \\
  126  &   3.04   &   158  &   9.86  \\
  128  &   3.25   &   160  &  11.26  \\
  130  &   3.47   &   162  &  13.46  \\
  \bottomrule
  \end{tabular}
\end{table}
 \\
Hier geht die Auswertung dann Normal weiter
