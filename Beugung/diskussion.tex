Die experimentell bestimmte Spaltbreite wird nun mit der Herstellerangabe verglichen.
Die relative Abweichung ist in Tabelle \ref{tab:rel} zu sehen.
\begin{table}
  \centering
  \caption{Relative Abweichung der Spaltbreite}
  \begin{tabular}{ccc}
    \toprule
    Herstellerangabe in & gemessener Wert & relative Abweichung \\
    $b\,/\mm$ & $b\,/\mm$ & \% \\
    \midrule
    0,15 & 0,18 & 20 \\
    \bottomrule
  \end{tabular}
  \label{tab:rel}
\end{table}
Die Abweichung von $20\,\%$ liegt hierbei im Rahmen der Messungenauigkeiten, die
beim Ablesen des Stroms und von $\zeta$ zustande kommen. Desweiteren kann nicht
garantiert werden, dass das Lichtbündel Zentral auf den Spalt trifft.
Bei der Kurve für den ersten Doppelspalt fällt auf, dass die Abstände der
gemessenen Maxima geringer ist als bei der Theoriekurve. Auch hier handelt es sich
nur um kleinere Messfehler. Bei dem zweitenDoppelspalt ist lediglich das Hauptmaxima
schwächer ausgeprägt ist, als in der Theoriekurve. bei diesem Doppelspalt besteht
 der Fehler in der Anordnung des Spalts, da hier der Lichtstrahl nur teilweise
 durch den Doppelspalt geht.
