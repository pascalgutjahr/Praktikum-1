Der gemessene Abstand $\su{L}$ zwischen Laser und Photodiode beträgt
\begin{equation*}
  \su{L} = 1\mt.
\end{equation*}
Die Wellenlänge $\lambda$ des verwendeten Lasers hat einen Wert von
\begin{equation*}
  \lambda = 532 \nm.
\end{equation*}
Der gemessene Dunkelstrom $\su{I_{du}}$ liegt bei
\begin{equation*}
  \su{I_{du}} = 0,3\nA
\end{equation*}
und wird von den gemessenen Werten subtrahiert.
Zuerst wird die Beugung am Einzelspalt gemessen. Der Spalt besitzt eine Breite
$\su{b}$
\begin{equation*}
  b=0,15\mm.
\end{equation*}
Diese gilt es mittels Interferenzmuster zu überprüfen.

Tabelle \ref{tab:eins} zeigt den an der Photodiode gemessenen Strom in Abhängigkeit
des Abstandes zum Hauptmaxima $\zeta$. Der Dunkelstrom ist hier noch nicht
berücksichtigt.
\begin{table}[H]
  \centering
  \caption{Messungen am Einzelspalt.}
  \begin{tabular}{cccc}
    \toprule
    % \mc{2}{c}{1}&\mc{2}{c}{2} \\
    $\zeta\,/\mm$ & $\su{I}\,/\nA$ & $\zeta\,/\mm$ & $\su{I}\,/\nA$ \\
    \midrule
    16.85 &   3 & 28.60 & 840 \\
    17.35 &   2 & 28.85 & 800 \\
    17.85 &   1 & 29.10 & 740 \\
    18.35 &   2 & 29.35 & 640 \\
    18.85 &   4 & 29.85 & 440 \\
    19.35 &   7 & 30.35 & 240 \\
    19.85 &   8 & 30.85 & 100 \\
    20.35 &   7 & 31.35 &  10 \\
    20.85 &   4 & 31.85 &   5 \\
    21.35 &   2 & 32.35 &  10 \\
    21.85 &   6 & 32.85 &  20 \\
    22.35 &  14 & 33.35 &  42 \\
    22.85 &  24 & 33.85 &  32 \\
    23.35 &  28 & 34.35 &  16 \\
    23.85 &  21 & 34.85 &   8 \\
    24.35 &   8 & 35.35 &  10 \\
    24.85 &   6 & 35.85 &  16 \\
    25.35 &  42 & 36.35 &  22 \\
    25.85 & 120 & 36.85 &  22 \\
    26.35 & 280 & 37.35 &  16 \\
    26.85 & 490 & 37.85 &   8 \\
    27.35 & 680 & 38.35 &   4 \\
    27.60 & 780 & 38.85 &   5 \\
    27.85 & 840 & 39.35 &   8 \\
    28.10 & 850 & 39.85 &  12 \\
    28.35 & 860 & \mc{2}{c}{\hrulefill}\\
    \bottomrule
  \end{tabular}
  \label{tab:eins}
\end{table}
Aus diesen Werten ergibt sich das Interferenzmuster, welches in der untenstehenden
Abbildung \ref{fig:eins} zu sehen ist.
\begin{figure}[H]
  \centering
  \includegraphics[width=0.7\textwidth]{bilder/Einzelspalt.pdf}
  \caption{Interferenzmuster am Einzelspalt.}
  \label{fig:eins}
\end{figure}
Mittels Ausgleichskurve lässt sich eine Spaltbreite von
\begin{equation*}
  b = 0,18\,\mm
\end{equation*}
feststellen. Die vom Hersteller angegebene Spaltbreite liegt bei
\begin{equation*}
  b = 0,15\,\mm.
\end{equation*}

Anschließend wird die Messung für zwei Doppelspalte mit unterschiedlichen Spaltbreiten
wiederholt. Die aufgenommenen Intensitäten für die verschiedenen $\zeta$ sind in
Tabelle \ref{fehlt} zu finden.
\begin{table}[H]
  \centering
  \caption{Aufgenommene Daten für Doppelspalte.}
  \begin{tabular}{cccc||cccc}
    \toprule
    \mc{4}{c||}{Doppelspalt 1} & \mc{4}{c}{Doppelspalt 2} \\
    $\zeta\,/\mm$ & $\su{I}\,/\nA$ & $\zeta\,/\mm$ & $\su{I}\,/\nA$ \\
    \midrule
    22,65 &  32,0 & 29.15 & 740.0 & 22.85 &  1.1 & 29.35 & 46.0 \\
    22,90 &  14,0 & 29.40 & 240.0 & 23.10 &  0.9 & 29.60 & 18.0 \\
    23,15 &   4,4 & 29.65 & 420.0 & 23.35 &  0.6 & 29.85 &  5.0 \\
    23,40 &   5,0 & 29.90 & 840.0 & 23.60 &  0.4 & 30.10 &  8.0 \\
    23,65 &   3,6 & 30.15 & 720.0 & 23.85 &  0.8 & 30.35 & 19.0 \\
    23,90 &   1,8 & 30.40 & 250.0 & 24.10 &  1.9 & 30.60 & 26.0 \\
    24,15 &   2,0 & 30.65 & 250.0 & 24.35 &  3.0 & 30.85 & 25.0 \\
    24,40 &   8,0 & 30.90 & 540.0 & 24.60 &  3.1 & 31.10 & 27.0 \\
    24,65 &  30,0 & 31.15 & 520.0 & 24.85 &  2.6 & 31.35 &  8.0 \\
    24,90 &  40,0 & 31.40 & 210.0 & 25.10 &  1.7 & 31.60 &  3.0 \\
    25,15 &   2,0 & 31.65 & 120.0 & 25.35 &  0.9 & 31.85 &  1.0 \\
    25,40 &   4,0 & 31.90 & 220.0 & 25.60 &  0.5 & 32.10 &  0.5 \\
    25,65 & 160,0 & 32.15 & 240.0 & 25.85 &  0.6 & 32.35 &  0.7 \\
    25,90 & 210,0 & 32.40 & 120.0 & 26.10 &  2.5 & 32.60 &  1.6 \\
    26,15 & 100,0 & 32.65 &  50.0 & 26.35 &  8.0 & 32.85 &  3.0 \\
    26,40 & 100,0 & 32.90 &  64.0 & 26.60 & 16.0 & 33.10 &  3.7 \\
    26,65 & 380,0 & 33.15 &  65.0 & 26.85 & 23.0 & 33.35 &  3.2 \\
    26,90 & 540,0 & 33.40 &  41.0 & 27.10 & 25.0 & 33.60 &  1.8 \\
    27,15 & 270,0 & 33.65 &  22.0 & 27.35 & 17.0 & 33.85 &  0.7 \\
    27,40 & 140,0 & 33.90 &  10.0 & 27.60 &  6.5 & 34.10 &  0.6 \\
    27,65 & 550,0 & 34.15 &   6.0 & 27.85 &  4.5 & 34.35 &  1.3 \\
    27,90 & 860,0 & 34.40 &  10.0 & 28.10 & 20.0 & 34.60 &  1.9 \\
    28,15 & 530,0 & 34.65 &   9.0 & 28.35 & 50.0 & 34.85 &  2.1 \\
    28,40 & 200,0 & 34.90 &   9.0 & 28.60 & 78.0 & 35.10 &  1.8 \\
    28,65 & 780,0 & 35.15 &  10.0 & 28.85 & 90.0 & 35.35 &  1.2 \\
    28,90 & 970,0 & \mc{2}{c}{\hrulefill} & 29.10 & 76.0 & \mc{2}{c}{\hrulefill} \\
    \bottomrule
  \end{tabular}
  \label{fehlt}
\end{table}
Die aufgenommenen Daten werden graphisch mit der dazugehörigen Theoriekurve dargestellt.
Abbildung \ref{dopp1} zeigt hierbei den ersten Doppelspalt mit einer angegebenen
Spaltbreite von $b=0,1\,\mm$ und Abbildung \ref{dopp2} den Doppelspalt mit einer
Spaltbreite von $b=0,15\,\mm$.
\begin{figure}[H]
  \centering
  \includegraphics[width=0.7\textwidth]{bilder/Doppelspalt1.pdf}
  \caption{Theoriekurve und Messwerte für Doppelspalt 1.}
  \label{dopp1}
\end{figure}
\begin{figure}
  \centering
  \includegraphics[width=0.7\textwidth]{bilder/Doppelspalt2.pdf}
  \caption{Theoriekurve und Messwerte für Doppelspalt 2.}
  \label{dopp2}
\end{figure}
