Der gemessene Abstand $\su{L}$ zwischen Laser und Photodiode beträgt
\begin{equation*}
  \su{L} = 1\mt.
\end{equation*}
Die Wellenlänge $\lambda$ des verwendeten Lasers hat einen Wert von
\begin{equation*}
  \lambda = 532 \nm.
\end{equation*}
Der gemessene Dunkelstrom $\su{I_{du}}$ liegt bei
\begin{equation*}
  \su{I_{du}} = 3\nA
\end{equation*}
und ist in den nachfolgenden Tabellen und Rechnungen bereits berücksichtigt.
Zuerst wird die Beugung am Einzelspalt gemessen. Der Spalt besitzt eine Breite
$\su{b}$
\begin{equation*}
  b=0,15\mm.
\end{equation*}
Diese gilt es mittels Interferenzmuster zu überprüfen.

Tabelle \ref{tab:eins} zeigt den an der Photodiode gemessenen Strom in Abhängigkeit
des Abstandes zum Hauptmaxima $\zeta$.
\begin{table}
  \centering
  \begin{tabular}{cccc}
    \mc{2}{c}{Nebenmaxima links}&\mc{2}{c}{Nebenmaxima rechts}
  \end{tabular}
  \caption{}
  \label{}

\end{table}
