Der gemessene Abstand $\su{L}$ zwischen Laser und Photodiode beträgt
\begin{equation*}
  \su{L} = 1\mt.
\end{equation*}
Die Wellenlänge $\lambda$ des verwendeten Lasers hat einen Wert von
\begin{equation*}
  \lambda = 532 \nm.
\end{equation*}
Der gemessene Dunkelstrom $\su{I_{du}}$ liegt bei
\begin{equation*}
  \su{I_{du}} = 3\nA
\end{equation*}
und ist in den nachfolgenden Tabellen und Rechnungen bereits berücksichtigt.
Zuerst wird die Beugung am Einzelspalt gemessen. Der Spalt besitzt eine Breite
$\su{b}$
\begin{equation*}
  b=0,15\mm.
\end{equation*}
Diese gilt es mittels Interferenzmuster zu überprüfen.

Tabelle \ref{tab:eins} zeigt den an der Photodiode gemessenen Strom in Abhängigkeit
des Abstandes zum Hauptmaxima $\zeta$.
\begin{table}[H]
  \centering
  \begin{tabular}{cccc}
    \toprule
    \mc{2}{c}{1}&\mc{2}{c}{2} \\
    $\zeta\,/\mm$ & $\su{I}\,/\nA$ & $\zeta\,/\mm$ & $\su{I}\,/\nA$ \\
    \midrule
    16.85 &   3 & 28.60 & 840 \\
    17.35 &   2 & 28.85 & 800 \\
    17.85 &   1 & 29.10 & 740 \\
    18.35 &   2 & 29.35 & 640 \\
    18.85 &   4 & 29.85 & 440 \\
    19.35 &   7 & 30.35 & 240 \\
    19.85 &   8 & 30.85 & 100 \\
    20.35 &   7 & 31.35 &  10 \\
    20.85 &   4 & 31.85 &   5 \\
    21.35 &   2 & 32.35 &  10 \\
    21.85 &   6 & 32.85 &  20 \\
    22.35 &  14 & 33.35 &  42 \\
    22.85 &  24 & 33.85 &  32 \\
    23.35 &  28 & 34.35 &  16 \\
    23.85 &  21 & 34.85 &   8 \\
    24.35 &   8 & 35.35 &  10 \\
    24.85 &   6 & 35.85 &  16 \\
    25.35 &  42 & 36.35 &  22 \\
    25.85 & 120 & 36.85 &  22 \\
    26.35 & 280 & 37.35 &  16 \\
    26.85 & 490 & 37.85 &   8 \\
    27.35 & 680 & 38.35 &   4 \\
    27.60 & 780 & 38.85 &   5 \\
    27.85 & 840 & 39.35 &   8 \\
    28.10 & 850 & 39.85 &  12 \\
    28.35 & 860 & \hrulefill & \hrulefill \\
    \bottomrule
  \end{tabular}
  \caption{Messungen am Einzelspalt.}
  \label{tab:eins}
\end{table}
