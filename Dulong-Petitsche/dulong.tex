\documentclass[captions=tableheading,
  bibliography=totoc,
  titlepage=firstiscover
  ]{scrartcl}
\usepackage{scrhack}
\usepackage[a4paper,top=1.5cm,left=3.5cm,right=1.75cm,bottom=2cm,bindingoffset=5mm]{geometry}

\usepackage[aux]{rerunfilecheck}

\usepackage{multirow}

\usepackage{polyglossia}
\setmainlanguage{german}

\usepackage{amsmath}
\usepackage{amssymb}
\usepackage{mathtools}
\usepackage{xfrac}
\usepackage{fontspec}
\usepackage{biblatex}
\addbibresource{lit.bib}  %nach polyglossia

\usepackage[
  math-style=ISO,
  bold-style=ISO,
  sans-style=italic,
  nabla=upright,
  partial=upright,
]{unicode-math}

\usepackage[
  locale=DE,
  separate-uncertainty=true,
  per-mode=symbol-or-fraction,
]{siunitx}

\usepackage[section, below]{placeins}
\usepackage[
  labelfont=bf,        % Tabelle x: Abbildung y: ist jetzt fett
  font=small,          % Schrift etwas kleiner als Dokument
%  width=0.9\textwidth, % maximale Breite einer Caption schmaler
  format=plain,
  indention=1em, % Abbildung sticht links etwas hervor
]{caption}
\usepackage{graphicx}
\usepackage{wrapfig}
\usepackage{grffile}
\usepackage{subcaption}

\usepackage{booktabs}
\usepackage{float}
%\restylefloat{figure}
\floatplacement{figure}{htbp}
\floatplacement{table}{htbp}
\usepackage{rotating}
\usepackage{chemmacros}

\usepackage[unicode]{hyperref}
\usepackage{bookmark}
\usepackage{microtype}
\usepackage[
version=4,
math-greek=default,
text-greek=default,
]{mhchem}


\newcommand{\be}{\begin{equation}} %Kurzbefehl für \begin{equation}
\newcommand{\ee}{\end{equation}} %Kurzbefehl für \end{equation}
\newcommand{\su}{\symup}
\newcommand{\mOhm}{\,\si{\milli\ohm}}
\newcommand{\Ohm}{\,\si{\ohm}}
\newcommand{\kOhm}{\,\si{\kilo\ohm}}
\newcommand{\Amp}{\,\si{\ampere}}
\newcommand{\Volt}{\,\si{\volt}}
\newcommand{\mA}{\,\si{\milli\ampere}}
\newcommand{\mV}{\,\si{\milli\volt}}
\newcommand{\mt}{\,\si{\meter}}
\newcommand{\pmt}{\,\si{\pico\meter}}
\newcommand{\fmt}{\,\si{\femto\meter}}
\newcommand{\qm}{\,\si{\square\meter}}
\newcommand{\cum}{\,\si{\cubic\meter}}
\newcommand{\cm}{\,\si{\centi\meter}}
\newcommand{\ms}{\,\si{\micro\second}}
\newcommand{\nm}{\,\si{\nano\meter}}
\newcommand{\qcm}{\,\si{\square\centi\meter}}
\newcommand{\ccm}{\,\si{\cubic\centi\meter}}
\newcommand{\Watt}{\,\si{\watt}}
\newcommand{\mm}{\,\si{\milli\meter}}
\newcommand{\gr}{\,\si{\gram}}
\newcommand{\kg}{\,\si{\kilo\gram}}
\newcommand{\J}{\,\si{\joule}}
\newcommand{\kJ}{\,\si{\kilo\joule}}
\newcommand{\Hz}{\,\si{\hertz}}
\newcommand{\kHz}{\,\si{\kilo\hertz}}
\newcommand{\MHz}{\,\si{\mega\hertz}}
\newcommand{\GHz}{\,\si{\giga\hertz}}
\newcommand{\acc}{\,\si{\meter\per\square\second}}
\newcommand{\vel}{\,\si{\meter\per\second}}
\newcommand{\kmh}{\,\si{\kilo\meter\per\hour}}
\newcommand{\sek}{\,\si{\second}}
\newcommand{\New}{\,\si{\newton}}
\newcommand{\Nm}{\,\si{\newton\meter}}
\newcommand{\Kel}{\,\si{\kelvin}}
\newcommand{\Cd}{\,\si{\candela}}
\newcommand{\Hen}{\,\si{\henry}}
\newcommand{\Far}{\,\si{\farad}}
\newcommand{\pas}{\,\si{\pascal}}
\newcommand{\Dichte}{\,\si{\kilo\gram\per\cubic\meter}}
\newcommand{\kVolt}{\,\si{\kilo\volt}}
\newcommand{\MVolt}{\,\si{\mega\vot}}
\newcommand{\keV}{\,\si{\kilo\electronvolt}}
\newcommand{\MeV}{\,\si{\mega\electronvolt}}
\newcommand{\dgr}{\,\si{\degree}}
\newcommand{\eV}{\,\si{\electronvolt}}
\newcommand{\nA}{\,\si{\nano\ampere}}
\newcommand{\lt}{\,\si{\litre}}
\newcommand{\ml}{\,\si{\milli\litre}}
% \newcommand{\min}{\,\si{\minute}}
\newcommand{\rt}{\right}
\newcommand{\lf}{\left}


\title{V201 - Das Dulong-Petitsche-Gesetz}
\author{Julian Jung \\ julian.jung@tu-dortmund.de
  \and Pascal Gutjahr \\ pascal.gutjahr@tu-dortmund.de}
  \date{Durchführung: 23.12.2016
  \hspace{3em}
  Abgabe: 13.01.2017}
  \begin{document}
\maketitle
\newpage
\tableofcontents
\newpage
\section{Zielsetzung}
Ziel des Experiments ist es, zu überprüfen ob die Schwingungsbewegungen von
Atomen, beziehungsweise Molekülen, in Festkörpern auf die Klassische oder über
die Methode der Quantenmechanik berechnet werden können. Zur Überprüfung wird
die Molwärme benutzt.
\section{Theorie}
 Ein RC-Schwingkreis besteht in der Regel aus einem Kondensator mit der
Kapazität $C$ und diner Spule, die eine Induktivität $L$ liefert.
Diese beiden Bauteile dienen hier als Energiespeicher.
In einem idealen Schwingkreis wird eine einmal eingespeicherte Energie
immer zwischen beiden genannten Elementen ausgetauscht.
Dieser Austausch kommt durch die Entladung des Kondensators zustande, bei welcher
in der Spule ein Magnetfeld aufgebaut wird. Wird dieses Magnetfeld abgebaut,
lädt sich der Kondensator auf und der Vorgang beginnt von neuem.
Wird ein gedämpfter, also ein realer Schwingkreis, betrachtet, ist neben der
Spule und dem Kondensator noch ein ohmscher Widerstand $R$ vorhanden.
Über diesen wird die Energie in Wärme umgewandelt und somit aus dem
Schwingkreis entfernt. $R$ ist demnach ein Dämpfungsfaktor.
Der schematische Aufbau eines $RCL$-Schwingkreises ist in Abbildung \ref{fig:rcl}
zu sehen.
\begin{figure}[H]
  \centering
  \includegraphics{Bilder/RCL.JPG}
  \caption{Schematischer Aufbau eines $RCL$-Kreises\cite{354}}
  \label{fig:rcl}
\end{figure}
Die Werte von $U_\su{R}, U_\su{C}$ und $U_\su{L}$ bezeichnen hierbei die Spannung
die über dem jeweiligen Bauteil abfällt. Gemäß des 2. Kirchhoffschen Gesetzes
gilt:
\begin{equation}
  U_\su{R}(t) + U_\su{C}(t) + U_\su{L}(t) = 0.
  \label{eqn:kirch}
\end{equation}
Um eine Differentialgleichung der 2. Ordnung zu erhalten, wird die Spannung
durch den Strom $I$ ausgedrückt. Somit erhält man:
\begin{align*}
  U_\su{R}(t) &= RI(t) \\
  U_\su{C}(t) &= \frac{Q(t)}{C} \text{mit} I = Q \\
  U_\su{L}(t) &= LI ,
\end{align*}
was zu der Differentialgleichung
\begin{equation}
  I(t) + \frac{R}{L}I(t) + \frac{1}{LC}I(t) = 0 .
\end{equation}
Mit dem entsprechendem Ansatz ergibt sich für die Gleichung die Lösung
\begin{equation}
  I(t)=\su{e}^{-2\pi\mu t}(A_1\su{e}^{i2\pi\nu t} + A_2\su{e}^{-2i\pi\nu t}).
  \label{eqn:dgl}
\end{equation}
Die Parameter $\mu$ und $\nu$ sind dabei definiert als:
\begin{align*}
  \mu &\coloneqq\frac{R}{4\pi L} \\
  \nu &\coloneqq\frac{1}{2\pi}\sqrt{\frac{1}{LC}-\frac{R^2}{4\L^2}}
\end{align*}
Für $\nu$ muss aufgrund der Wurzel eine Fallunterscheidung gemacht werden, da
$\nu$ sowohl reel, als auch rein imaginär sein kann.
Für den reelen Fall muss
\begin{equation*}
  \frac{1}{LC} > \frac{R^2}{4L^2}
\end{equation*}
gelten. Unter diesen Voraussetzungen lässt sich Formel (\ref{eqn:dgl}) mit der
Eulerschen Formel zu
\begin{equation}
  I(t)=A_\su{0}\su{e}^{-2\pi\mu t}\cos(2\pi\nu t+\mu)
\end{equation}
umschreiben. Hier entsteht eine gedämpfte Schwingung, welche bei $t\rightarrow\inf$
gegen Null strebt. Die Abklingdauer dieser Schwingung wird dann durch
\begin{equation}
  T_\su{ex}\coloneqq\frac{1}{2\pi\mu}
\end{equation}
definiert.

für den Fall dass $\nu$ imaginär ist, also der Fall
\begin{equation*}
  \frac{1}{LC} < \frac{R^2}{4L^2}
\end{equation*}
eintritt, lässt sich Formel (\ref{eqn:dgl}) zu
\begin{equation}
  I(t)\propto\su{e}^{-(2\pi\mu -\su{i}2\pi\nu)t}
\end{equation}
umschreiben.
Da sich i und $\nu$ zu einem insgesamt reelen Exponenten verrechnen lassen, fällt
die Amplitude des Stroms exponentiell ab. Die Amplitude hat dabei, je nach
Anfangsbedinungen, einen oder keinen Extremwert. Am schnellsten fällt die
Amplitude, wenn der Spezialfall
\begin{equation*}
  \frac{1}{LC} = \frac{R^2}{4L^2}
\end{equation*}
eintritt. Dieser Fall wird aperiodischer Grenzfall genannt und ist in der
Abbildung \ref{fig:agf} als gestrichelte Linie dargestellt.
\begin{figure}[h]
  \centering
  \includegraphics{Bilder/aperiod.JPG}
  \caption{Zeitverlauf des Stroms mit aperiodischer Dämpfung\cite{354}}
  \label{fig:agf}
\end{figure}
\\
Abblidung (\ref{fig:angeregt}) zeigt einen $RCL$-Schwingkreis, welcher von
außen angeregt wird. Die Anregung erfolgt hier durch eine Wechselstromquelle.
\begin{figure}[h]
  \centering
  \includegraphics{Bilder/angeregt.JPG}
  \caption{Schaltbild eines angeregten$RCL$-Schwingkreises}
  \label{fig:angeregt}
\end{figure}
Nach einer kurzen Einschwingzeit nimmt der Schwingkreis die Frequenz der
Wechselstromquelle an. Die zu Beginn aufgestellte Differentialgleichung aus
Formel (\ref{eqn:dgl}) wird nun inhomogen und lautet:
\begin{equation}
  LCU_\su{C}(t) + RCU_\su{c}(t) + U_\su{c}(t) =U_0\su{e}^{\su{i}\omega t}.
  \label{eqn:inh}
\end{equation}
Die Lösung für die Spannung in Abhängigkeit der Zeit berechnet sich dann mit:
\begin{equation}
  U(t)=\frac{U_0(1-LC\omega^2-\su{i}\omega RC)}{(1-LC\omega^2)^2+\omega^2 R^2 C^2}.
\end{equation}
Für die Phasenverschiebung zur Erregerspannung ergibt sich durch den Vergleich
von Imaginär- und Realteil ergibt sich: %Umformulieren um ':' zu vermeiden!!
\begin{equation}
  \varphi(w)=\arctan\biggr(\frac{\su{Im}(U)}{\su{Re}(U)}\biggl)
  =\arctan\biggr(\frac{-\omega RC}{1-LC\omega^2}\biggl)
\end{equation}
Für die Frequenzen $\omega_1$ und $\omega_2$ gilt bei einer Phasenverschiebung
von $\frac{\pi}{4}$ beziehungsweise $\frac{3\pi}{4}$
\begin{equation}
  \omega_\su{1,2}=\pm\frac{R}{2L}+\sqrt{\frac{R^2}{4L^2}+\frac{1}{LC}}.
\end{equation}

Die Spannung kann zusätzlich mit
\begin{equation}
  U_\su{C}(\omega)=\frac{U_0}{\sqrt{(1-LC\omega^2)^2+\omega^2R^rC^2}}
\end{equation}
in Abhängigkeit von der Frequenz $\omega$ angegeben werden.
Hierbei zeigt sich, dass die Spannungsamplitude für sehr hohe Frequenzen gegen
Null strebt, während sie für kleine Frequenzen gegen $U_0$ strebt.
Die Resonanzfrequenz
\begin{equation}
  \omega_\su{res}=\sqrt{\frac{1}{LC}-\frac{R^2}{2L^2}}
\end{equation}
beschreibt den Zustand, bei dem $U_\su{C}$ einen Maximalwert erreicht der auch
größer als $U_\su{0}$ sein kann.
Von schwacher Dämpfung wird gesprochen, wenn
\begin{equation}
  \frac{R^2}{2L^2} \ll \frac{1}{LC}
\end{equation}
gilt. In diesem Zustand gilt $\omega_\su{res}\approx\omega_0$, wobei $\omega_0$
die Kreisfrequenz der ungedämpften Schwingung ist und den Wert
\begin{equation}
  \omega_0 =\sqrt{\frac{1}{LC}}
\end{equation}
annimmt. Das Maximum der Kondensatorspannung ist dann um den Faktor
\begin{equation}
  q=\frac{1}{\omega_0RC}
\end{equation}
größer als die Erregerspannung. Der Faktor $q$ wird als Güte des Schwingkreises
bezeichnet.

\section{Aufbau und Durchführung}
 % Im ersten Schritt wird die Leerlaufspannung $U_\symup{0}$ einer Monozelle
gemessen. Hierfür werden die Anschlüsse des Voltmeters direkt mit den Anschlüssen
der Monozelle verbunden. Die angezeigte Spannung und der Innenwiderstand des
Voltmeters $R_\symup{i,V}$ werden notiert.
Nach dieser Messung werden ein Amperemeter und ein regelbarer Widerstand
$R_\symup{a}$ wie in Abbildung \ref{fig:schlt1} mit der Monozelle in Reihe geschaltet.
\begin{figure}[H]
  \centering
  \begin{subfigure}{0.48\textwidth}
    \centering
    \includegraphics[width=4cm]{bilder/sinrecht.jpg}
    \caption{Messreihe 1}
    \label{fig:schlt1}
  \end{subfigure}
  \begin{subfigure}{0.48\textwidth}
    \centering
    \includegraphics[width=4cm]{bilder/gegenspannung.jpg}
    \caption{Messreihe 2}
    \label{fig:schlt2}
  \end{subfigure}
  \caption{Schaltbilder \cite{301}}
  \label{fig:schlt}
\end{figure}
Das Voltmeter bleibt parallel geschaltet um den Spannungsverlauf über der
Monozelle zu messen. Der Widerstand wird hierbei von $(0-50)\Ohm$
variiert. Die Werte für $U_\symup{k}$ und $I$ werden notiert.
Abbildung \ref{fig:schlt2} zeigt die zweite Messreihe. Hier wird eine
Gegenspannung an die Monozelle gelegt. Die Gegenspannung ist $2\,\symup{\si{\volt}}$
größer als $U_\symup{0}$. Der Widerstand wird erneut von $(0-50)\,\symup{\si{\ohm}}$
variiert und die Werte für Strom und Spannung werden notiert.
Nach dieser Messung wird erneut die Schaltung aus \ref{fig:schlt1} aufgebaut.
Die Monozelle wird jedoch durch einen RC-Generator ersetzt. Dieser soll zunächst
eine Rechteckspannung liefern. Der Widerstand wird nun zwischen $(20-250)\,\symup{\si{\ohm}}$
variiert. Die Werte für $I$ und $U_\symup{k}$ werden an den Messgeräten
abgelesen und notiert.
Die Messung wird mit einer Sinusspannung und einem Widerstand von
$(0,1-5)\,\symup{\si{\kilo\ohm}}$ wiederholt.
\newpage

\section{Auswertung}
 % Tabelle \ref{tab:messwerte} zeigt die im Versuch gemessenen Werte.
$T_\su{1}$ entspricht hierbei der Temperatur des wärmeren Reservoirs, während $T_\su{2}$
die Temperatur des kälteren Reservoires beschreibt. Der Druck $P_\su{a}$ wird dem warmen Reservoir
zugeordnet und $p_\su{b}$ dem kälteren.
\begin{table}[h]
  \centering
  \begin{tabular}{c c c c c c}
    \toprule
    $t\,/\sek$ &$T_\su{1}$\,/$\Kel$ & $T_\su{2}\,/\Kel$ & $p_\su{a}\,/\,\pas $ &
    $p_\su{b}\,/ \pas$ & N\,/$\,\si{\watt}$  \\
    \midrule
       0   &   294.65   &   294.55   &   5100000    &     5250000   &      0 \\
      60   &   295.55   &   294.55   &   2400000    &     6900000   &    165 \\
     120   &   296.35   &   294.45   &   2600000    &     7000000   &    175 \\
     180   &   297.65   &   293.55   &   2900000    &     7500000   &    185 \\
     240   &   299.25   &   292.05   &   3000000    &     7750000   &    195 \\
     300   &   301.15   &   290.25   &   3100000    &     8250000   &    200 \\
     360   &   303.15   &   288.35   &   3200000    &     8500000   &    203 \\
     420   &   305.15   &   286.55   &   3200000    &     9000000   &    205 \\
     480   &   307.05   &   284.65   &   3200000    &     9500000   &    206 \\
     540   &   308.95   &   282.75   &   3200000    &     9800000   &    208 \\
     600   &   310.85   &   281.05   &   3200000    &    10250000   &    209 \\
     660   &   312.55   &   279.45   &   3200000    &    10500000   &    211 \\
     720   &   314.35   &   277.85   &   3200000    &    11000000   &    212 \\
     780   &   316.05   &   276.25   &   3200000    &    11250000   &    212 \\
     840   &   317.55   &   274.95   &   3200000    &    11750000   &    212 \\
     900   &   319.05   &   273.95   &   3200000    &    12000000   &    213 \\
     960   &   320.55   &   273.35   &   3200000    &    12500000   &    213 \\
    1020   &   321.95   &   272.85   &   3200000    &    13000000   &    210 \\
    1080   &   323.15   &   272.45   &   3200000    &    13250000   &    207 \\
    \bottomrule
  \end{tabular}
  \caption{Messwerte}
  \label{tab:messwerte}
\end{table}
\\
Der Verlauf der Temperaturen $T_1$ und $T_2$ wird graphisch in Abbildung
\ref{fig:T1T2} dargestellt.
\begin{figure}[h]
  \centering
  \includegraphics[width=0.7\textwidth]{Bilder/Plot1.pdf}
  \caption{Verlauf der Temperaturen in Abhängigkeit der Zeit}
  \label{fig:T1T2}
\end{figure}
\\
Die Ausgleichsrechnung ergibt sich über die Formel
\begin{equation*}
  T(t) = At^2+Bt+C.
\end{equation*}
Hieraus ergeben sich für $T_1$ die Parameter
\begin{align*}
  A &= (-2.6 \pm 1.3) \cdot10^{-6}\,\si{\kelvin\per\square\second} \\
  B &= (0.032 \pm 0.002) \,\si{\kelvin\per\second} \\
  C &= (293.8 \pm 0.4) \Kel
\end{align*}
und für $T_2$
\begin{align*}
  A &= (7.0 \pm 2.5) \cdot 10^{-6} \,\si{\kelvin\per\square\second}\\
  B &= (-0.032 \pm 0.003)  \,\si{\kelvin\per\second} \\
  C &= (298.9 \pm 0.7) \Kel.
\end{align*}

\noindent Aus diesen Parametern wird der Differenzenquotient
\begin{equation}
  \frac{dT}{dt} = 2At+B
\end{equation}
für 4 Zeitpunkte bestimmt.
\begin{table}[!h]
  \centering
  \begin{tabular}{c c c c c}
    \toprule
    $t\,/\sek$ & $T_1 \,/\Kel$ & $\frac{dT_1}{dt}\,/\si{\milli\kelvin\per\second}$ &
    $T_2\,/\Kel$ & $\frac{dT_2}{dt}\,/\si{\milli\kelvin\per\second}$ \\
    \midrule
    180 &  297.65 &  31\pm2 &  293.55 &  -29\pm4  \\
    540 &  308.95 &  29\pm3 &  282.75 &  -24\pm6  \\
    720 &  314.35 &  28\pm4 &  277.85 &  -21\pm7  \\
    900 &  319.05 &  27\pm4 &  273.95 &  -18\pm8  \\
    \bottomrule
  \end{tabular}
  \caption{Differenzenquotienten zu $T_1$ und $T_2$}
  \label{tab:diff}
\end{table}
\\
Im Folgenden wird die Güteziffer $\nu$ der Wärmepumpe mit Formel \eqref{eqn:güte}
ermittelt. Dabei wird nur $T_1$ betrachtet, da im Idealfall davon ausgegangen wird,
dass $T_1$ und $T_2$ gleich sind (reversibler Kreisprozess).
Die Wärmekapazität der Eimer und Kupferschlangen kann an der Apparatur abgelesen werden,
sie beträgt $660\,\si{\joule\per\kelvin}$. In einem Eimer befinden sich
$3\,\si{\kilo\gram}$ Wasser mit der Wärmekapazität
$c_\su{w}= 4.183 \,\si{\kilo\joule\per\kilo\gram\kelvin}$\,\cite{chemie}.
Für die Leistung $N$ wird der Mittelwert mit $\overline{N}=(190\pm50)\,\si{\watt}$ verwendet.
Der Fehler der Güteziffer berechnet sich nach der Gauß'schen Fehlerfortpflanzung
mit:
\begin{equation*}
\Delta\nu = \sqrt{\biggl(\frac{m_\su{w}c_\su{w}+m_\su{w}c_\su{w}}{N}\biggr)^2
\cdot\sigma_\su{\frac{dT}{dt}}^2 + \biggl(\frac{m_\su{w}c_\su{w}+m_\su{w}c_\su{w}}{N^2}\biggr)^2
\cdot\sigma_\su{N}}
\end{equation*}
\begin{table}[!h]
  \centering
  \begin{tabular}{c c}
    \toprule
    $t\,/\sek$ & $\nu_\su{exp}$ \\
    \midrule
    180 &  2.2\pm 0.6 \\
    540 &  2.0\pm 0.6 \\
    720 &  2.0\pm 0.6 \\
    960 &  1.9\pm 0.6 \\
    \bottomrule
  \end{tabular}
  \caption{Experimentelle Güteziffer}
  \label{tab:expgüte}
\end{table}
Der Massendurchsatz $\frac{dm}{dt}$ lässt sich nach Formel \eqref{eqn:dm/dt}
berechnen. Für die vier verschiedenen Zeiten erhält man die in Tabelle
\ref{tab:dm/dt} gelisteten Massendurchsätze
\begin{table}
  \centering
  \begin{tabular}{c c}
    \toprule
    $t\,/\sek$ & $\frac{dm}{dt}\,/\,\frac{\si{\mol}\cdot\Kel}{\sek}$
    180  &  (-19.1 \pm 3)\cdot\,10^{-3}
    540  &  (-15.8 \pm 4)\cdot\,10^{-3}
    720  &  (-13.8 \pm 5)\cdot\,10^{-3}
    960  &  (-11.8 \pm 5)\cdot\,10^{-3}
  \end{tabular}
  \caption{Massendurchsatz zu verschiedenen Zeiten}
  \label{tab:dm/dt}
\end{table}
Um die mechanische Leistung des Kompressors zu berechnen, wird Formel \eqref{eqn:Nmech}
verwendet. Das darin enthaltene $\rho$ berechnet sich nach Formel \eqref{eqn:rho}.
Für die verschiedenen Zeiten erhält man die in Tabelle \ref{tab:Nmech} aufgelisteten
Werte.
\begin{table}[!h]
  \centering
  \begin{tabular}{c c}
    \toprule
    $t\,/\sek$ & $N_\su{mech}\,/\si{\watt}$ \\
    \midrule
    180 &  417757.2 \pm  8058.1 %evtl verrechnet
    540 &  341234.4 \pm 10609.1
    720 &  297056.2 \pm 13217.6
    960 &  296612.9 \pm 13197.8
    \bottomrule
  \end{tabular}
  \caption{Mechanische Leistung}
  \label{tab:Nmech}
\end{table}

\section{Diskussion}
 % Die Abweichungen der Einzelmesungen für die Fourier-Analyse ergeben sich aus der Formel
\begin{equation}
  \sigma = \frac{a_\su{theo}-a_\su{exp}}{a_\su{theo}}.
\end{equation}
Die jeweiligen Werte wurden dabei schon in der dazugehörigen Tabelle eingetragen.

Für die Abweichung der gesamten Kurve wird Formel \eqref{eqn:fehler} verwendet.
Der Fit der Dreieckspannung weicht somit um $50\,\%$ ab, der Fit der Rechteckspannung
um $19\,\%$ und der Fit der Sägezahnspannung zeigt keine Abweichung. Die hohe Abweichung
der Dreieckspannung folgt aus den wenigen Messwerten, die dazu aufgenommen wurden.
Zudem ergeben sich Messungenauigkeiten beim Ablesen der Amplituden.

\noindent Bei der Fourier-Synthese haben wir nahezu alle Werte exakt einstellen können,
nur bei der Dreieckspannung war dies etwas schwierig, da wir für den vierten Wert $38 \mV$
einstellen sollten, jedoch ließ sich die Spannung nur bis zu einem Minimalwert von $48 \mV$
einstellen. Diesen Wert haben wir aber trotzdem noch in die Messung mit einbezogen.

\printbibliography
\end{document}
