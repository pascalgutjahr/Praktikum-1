\subsection{Spezifische Wärmekapazität}
Die Fähigkeit eines Stoffes Wärmeenergie zu speichern wird spezifische
Wärmekapazität genannt. Sie ist durch
\begin{equation}
  \Delta Q = mc\Delta T \Leftrightarrow c = \frac{\Delta Q}{m\Delta T}
\end{equation}
gegeben und folgt aus dem ersten Hauptsatz der Thermodynamik mit $\Delta A=0$.
$\Delta Q$ ist hierbei die zu- bzw. abgeführte Wärmemenge, $\Delta T$ die
Temperaturdifferenz und $m$ die Masse des Stoffs.
\subsection{Molwärme}
Um ein Mol eines Stoffes um $1\Kel$ zu erwärmen, wird eine Wärmemenge benötigt,
welche auch als Molwärme C bezeichnet wird. Es muss hier zwischen der Molwärme
bei isobaren Prozessen $C_\su{p}$ und der Molwärme bei isochoren Prozessen
$C_\su{V}$ unterschieden werden. Letztere lässt sich mit
\begin{equation}
  C_\su{V}=\left(\frac{dU}{dt}\right)_\su{V}
  \label{eq:cv}
\end{equation}
berechnen.

nach dem Dulong-Petitsche-Gesetz beträgt die Molwärme stets $3R$, unabhängig
vom betrachteten Stoff und der Temperatur. $R$ ist die allgemeine Gaskonstante.
Die Begründung hierfür liegt in der klassischen Methode. Demnach wird die
Bewegung von Atomen in einem Festkörper aufgrund der vorhandenen Gitterkräfte
durch eine harmonische Schwingung beschrieben. Es ist:
\begin{equation}
  \langle u\rangle = \langle E_\su{kin}\rangle+\langle E_\su{pot}\rangle
\end{equation}
Mit dem Äquipartitionstheorem und Formel \eqref{eq:cv} folgt
\begin{equation}
  C_\su{V}= 3R
  \label{eq:C3R}
\end{equation}
In der Quantenmechanik sind die möglichen Energieänderungen des harmonischen
Oszillators durch
\begin{equation}
  \Delta U =n\hbar\omega
\end{equation}
beschrieben. Über die Boltzmann-Statistik ergibt sich der Zusammenhang
\begin{equation}
  \langle U_\su{qu}\rangle = \frac{3N_\su{L}\hbar\omega}{\exp{\left(\frac{\hbar
  \omega}{kT}\right)}-1}
  \label{eq:quant}
\end{equation}
Der angezeigte Wert aus Formel \eqref{eq:C3R} ist ein Grenzfall der
quantenmechanischen Beschreibung aus \eqref{eq:quant} für hohe Temperaturen.
Dieser Fall wird schneller erreicht je größer die Atommasse des Stoffes ist.
\subsection{Messung der spezfischen Wärmekapazität}
Da die Umsetzung für konstantes Volumen schwierig umzusetzen ist, wird der
Versuch bei konstantem Druck durchgeführt. Hierbei gilt der Zusammenhang
\begin{equation}
  C_\su{k} = C_\su{p}-9\alpha^2\kappa V_0T
  \label{eq:mol}
\end{equation}
mit $\alpha$ als linearer Ausdehnungskoeffizient, $\kappa$ das Kompressionsmodul
, $V_0$ das Molvolumen und $T$ die absolute Temperatur. Die spezifische
Wärmekapazität $c_\su{g}m_\su{g}$ wird über Mischkalometrie bestimmt.
Hierfür wird die Wärmemenge vom warmen Wasser
\begin{equation}
  Q_\su{w}=c_\su{w}m_\su{y}(T_\su{y}-T)
\end{equation}
mit der vom kalten Wasser und der vom Kalorimeter aufgenommenen Wärme
\begin{equation}
  Q_\su{kalt+Kalometrie}=(c_\su{w}m_\su{x}+c_\su{g}m_\su{g})(T_\su{m}-T_\su{x})
\end{equation}
gleichgesetzt und nach $c_\su{g}m_\su{g}$ umgestellt.
\begin{equation}
  c_\su{g}m_\su{g}=\frac{c_\su{w}m_\su{y}(T_\su{y}-T_\su{m})-c_\su{w}m_\su{x}
  (T_\su{m}-T_\su{w})}{(T_\su{m}-T_\su{x})}
  \label{eq:kalo}
\end{equation}
Die Wärmekapazität der Proben wird ähnlich bestimmt, jedoch wird die
Mischtemperatur der Probe und Wasser bestimmt wird. Es gilt:
\begin{equation}
  c_\su{k}=\frac{(c_\su{w}m_\su{w}+c_\su{g}m_\su{g})(T_\su{m}-T_\su{w})}
  {m_\su{k}(T_\su{k}-T_\su{w})}
  \label{eq:probe}
\end{equation}
