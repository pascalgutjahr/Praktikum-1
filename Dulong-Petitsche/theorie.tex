\subsection{Klassische Methode}
Der Zusammenhang zwischen der Wärmemenge Q und der Temperatur T wird durch die
Molwärme C beschrieben. Wenn ein Mol eines Stoffes um eine Temperatur $dT$
erhöt wird, nimmt er die Wärmemenge $dQ$ auf. Daraus ergibt sich der
Zusammenhang
\begin{equation}
  C=ßfrac{dQ}{dT}
  \label{eq:molwarm}
\end{equation}
Die Gleichheit von Wärmemenge und Energie tritt ein, wenn an dem Stoff keine
Arbeit verrichtet wird.
Bei der Erwärmung wird unterschieden, ob diese unter gleichbleibenden
Druckverhältnissen (isobar) oder bei konstantem Volumen (isochor) stattfindet.
Um unterscheiden zu können, welche Größe konstant gehalten wird, erhält die
Gleichung \eqref{eq:molwarm} passende Indizes. Sodass
\begin{equation}
\begin{split}
  C_\su{P}&=\biggl(\frac{dQ}{dT}\biggr)_\su{P} \\
  C_\su{V}&=\biggl(\frac{dQ}{dT}\biggr)_\su{V}
  \label{eq:index}
\end{split}
\end{equation}
Der Unterschied der Molwärme bei einer isobaren Erwärmung und der Molwärme bei
isochorer Erwärmung lässt sich mit
\begin{equation}
\begin{split}
  c_\su{V}&= C_\su{P}-9\alpha^2\kappa V_0 T \\
  &= c_\su{K}\,\cdot M - 9\alpha^2\kappa\frac{M}{\rho}T
  \label{eq:molwarm2}
\end{split}
\end{equation}
einfach berechnen. Hierbei wird betrachtet, dass sich die Molwärme über den
Zusammenhang zwischen Wärmekapazität $c_\su{K}$ und molare Masse berechnet
werden kann. Weiter sind $V_0$ das Molvolumen und $\alpha$ und $\kappa$
Materialkonstanten, hinsichtlich der Volumenänderung.

Die spezifische Wärmekapazität ist hierbei der Proportionalitätsfaktor der
angibt, wie sich die Wärmemenge bei einer Temeraturänderung ändert:
\begin{equation}
  \Delta Q = mc_\su{K}\cdot\Delta T.
\end{equation}

Das Dulong-Petitsche-Gesetz besagt, dass die Molwärme unabhängig von der
Stoffeigenschaft eines Körpers immer 3 mal größer als allgemeine
Gaskonstante $R=8,314\,\si{\joule\per\mol\kelvin}$\cite{chemie} ist.
Nach den Methoden der klassischen Mechanik lässt sich dieser Wert berechnen,
indem die Energien betrachtet werden, die durch wärmebedingte Bewegung der Atome
entstehen. Diese Atome führen harmonische Schwingungen aus, da sie feste Plätze
im Festkörper besitzen. Wird die Energiebetrachtung für harmonische Oszillatoren
durchgeführt, lässt sich feststellen, dass kinetische und potentielle Energie
im zeitlichen Mittel gleich groß sind. Die Gesamtenergie eines Atoms ergibt sich
dann aus
\begin{equation}
  E= E_\su{kin}+E_\su{pot} = 2\,\cdot\,E_\su{kin}.
\end{equation}
Verglichen mit dem Äquipartitionstheorem, ergibt sich für die Gesamtenergie
$k_\su{B}T$. Wobei $k_\su{B}$ die Boltzmann-Konstante ist.
Betrachtet man alle Atome und den Fakt, dass jedes Atom sich in 3 Raumrichtungen
bewegen kann, ergibt sich nach Gleichung \eqref{eq:molwarm} die Formeln
\begin{equation}
  E=3RT\quad\text{bzw.}\quad C_\su{V}=3R
\end{equation}
\subsection{quantenmechanische Methode}
Es ist auffällig, dass die Molwärme bei niedrigen Temperaturen nicht mehr den
zuvor hergeleiteten Wert 3R annimmt. Die Erklärung folgt aus der Quantenmechanik
. Ein Atom, welches mit einer Frequenz $\omega$ oszilliert, kann nur Energien von
bestimmten Größen
\begin{equation}
  \Delta E = n\cdot\hbar\cdot\omega
\end{equation}
aufnehmen oder abgeben. Der Zusammenhang ergibt sich dann durch
\begin{equation}
  E=\frac{3N_\su{A}\hbar\omega}{\exp[\frac{\hbar\omega}{kT}]-1},
\end{equation}
mit $N_\su{A}$ als Avogadrokonstante. Im Fall $kT\gg\hbar\omega$ strebt die
Energie erneut gegen $3R$. Für kleine Massen ist dies wegen $\omega\propto
\frac{1}{\sqrt{m}}$ erst bei hohen Temperaturen erfüllt.
