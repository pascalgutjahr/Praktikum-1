Um die Molwärme berechnen zu können, müssen zunächst die spezifischen
Wärmekapazitäten der einzelnen Proben experimentell bestimmt werden.
Die Molwärme lässt sich dann ohne weitere experimentelle Durchführungen
berechnen.
\subsection{Bestimmung der spezifischen Wärmekapazität}
Zuerst wird die Masse $m_\su{K}$ der Probe mit einer Schnellwaage ermittelt.
Danach wird diese in einem Wasserbad auf eine Temperatur $T_\su{K}$ von
mindestens $60\,\si{\celsius}$ erhitzt.
