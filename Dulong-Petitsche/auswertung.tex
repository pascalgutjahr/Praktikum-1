\subsection{Spezifische Wärmekapazität des Kalorimeters}
Zunächst soll die spezifische Wärmekapazität des Kalorimeters nach Formel
\eqref{eq:kalo} bestimmt werden. Hierfür werden die Messwerte der Tabelle
\ref{tab:kalo} benötigt.
\begin{table}
  \centering
  \begin{tabular}{c c c c c c}
    \toprule
    $m_\su{x}\,/\gr$ & $T_\su{x}\,/\Kel$ & $m_\su{y}\,/\gr$ & $T_\su{y}\,/\Kel$
    & $T_\su{M}$ & $c_\su{Km}\,/\si{\joule\per\gram\kelvin}$ \\
    \midrule
    \bottomrule
  \end{tabular}
  \caption{Messwerte zur Bestimmung der Wärmekapazität}
  \label{tab:kalo}
\end{table}
Der gemittelte Wert für die spezifische Wärmekapazität beträgt dann
\begin{equation*}
  c_\su{g}m_\su{g} = (\pm)\,\si{\joule\per\kelvin}
\end{equation*}
Der dazugehörige Fehler berechnet sich über die Standardabweichung.
\subsection{Spezifische Wärmekapazität der Proben}
Die spezifische Wärmekapazität wird mittels Formel \eqref{eq:probe} berechnet.
Die gemessenen Werte für Blei sind in der untenstehenden Tabelle \ref{tab:pb}
zu sehen.
\begin{table}
  \centering
  \begin{tabular}{c c c c c}
    \toprule
    $m_\su{Pb}\,/\gr$ & $T_\su{Pb}\,/\Kel$ & $m_\su{K}\,/\gr$ & $T_\su{K}\,/\Kel
    $ & $T_\su_{M}\,/\Kel$\\
    \midrule
    \bottomrule
  \end{tabular}
  \caption{Messwerte Blei}
  \label{tab:pb}
\end{table}
Aus diesen Werten ergibt sich eine Mittlere Wärmekapazität von
\begin{equation*}
  c_\su{Pb}=(\pm)\,\si{\joule\per\kelvin}
\end{equation*}
Der Fehler ergibt sich auch hier aus der Standardabweichung.

Die aufgenommenen Werte für Aluminium und Graphit sind in Tabelle \ref{tab:alug}
zu finden.
\begin{table}
  \begin{tabular}{c c c}
    \toprule
    \hbarfill & Aluminium & Graphit \\
    \midrule
    $m_\su{Probe}\,/\gr$ &
    $T_\su{Probe}\,/\Kel$&
    $m_\su{Kalo}\,/\gr$  &
    $T_\su{Kalo}\,/\Kel$ &
    $T_\su{M}\,/\Kel$    &
  \end{tabular}
  \caption{Messwerte fpr Aluminium und Graphit}
  \label{\tab:alug}
\end{table}
Daraus ergeben sich die folgenden Wärmekapazitäten:
\begin{align*}
  c_\su{alu} &= (\pm)\,\si{\jouler\per\kelvin} \\
  c_\su{Gr}  &= (\pm)\,\si{\jouler\per\kelvin}
\end{align*}
Die Formel für den Fehler ergibt sich mit
% \begin{equation}
% \Delta c_\su{k} = \sqrt{\left(\frac{(T_\su{m}-T_\su{w})}{m_\su{k
% (T_\su{k}-T_\su{w})}}\right)^2*\sigma_\su{Kalorimeter}^2}
% \end{equation}
aus der Gauß'schen Fehlerfortpflanzung.
\subsection{Molwärme der Proben}
Die Molwärme der Proben wird mit Formel \eqref{eq:mol} und der Umstellung
\begin{equation}
  V=\frac{M}{\rho}
\end{equation}
berechnet. Werte für $M$, $\alpha$, $\rho$ und $\kappa$ sind aus der Anleitung
\cite{201} entnommen und in Tabelle \ref{tab:mol} aufgeführt.
\begin{table}
  \centering
  \begin{tabular}{c | c c c c}
    \toprule
    Probe & $M\,/\,\si{\gram\per\mol}$&$\alpha\,/\,\si{1\per\kelvin}$
    &$\kappa\,/\,\si{\newton\per\meter}$&$\rho\,/\,\si{\kilo\gram\per\cubic
    \meter}$ \\
    \midrule
    Blei
    Aluminium
    Graphit
    \bottomrule
  \end{tabular}
  \caption{Werte der Proben}
  \label{tab:mol}
\end{table}
Mit der Mischtemperaturen aus den Tabellen \ref{tab:pb} und \ref{tab:alug}
ergibt sich für Blei eine mittlere Molwärme von
\begin{equation*}
  C_\su{Pb}= (\pm)\,si{\joule\per\mol}
\end{equation*}
und für Aluminium und Graphit
\begin{align*}
  C_\su{Alu} &= (\pm)\,\si{\joule\per\mol}
  C_\su{Gra} &= (\pm)\,\si{\joule\per\mol}
\end{align*}
Für diese beiden Proben wird der Fehler mit 
\begin{equation*}
  \Delta C_\su{k}=\sqrt{M^2\sigma_\su{c_k}^2}
\end{equation*}
berechnet
