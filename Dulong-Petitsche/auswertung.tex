\subsection{Spezifische Wärmekapazität des Kalorimeters}
Zunächst soll die spezifische Wärmekapazität des Kalorimeters nach Formel
\eqref{eq:kalo} bestimmt werden. Hierfür werden die Messwerte der Tabelle
\ref{tab:kalo} benötigt.
\begin{table}
  \centering
  \begin{tabular}{c c c c c c}
    \toprule
    $m_\su{y}\,/\gr$ & $T_\su{y}\,/\Kel$ & $m_\su{x}\,/\gr$ & $T_\su{x}\,/\Kel$
    & $T_\su{m}$ & $c_\su{g}m_\su{g}\,/\si{\joule\per\gram\kelvin}$ \\
    \midrule
    268.89 & 373.15 & 266.88 & 294.64 & 329.03 & 326.41\\
    220.95 & 373.15 & 300.66 & 294.64 & 323.92 & 296.10\\
    231.61 & 373.15 & 284.78 & 294.64 & 326.60 & 219.71\\
    \bottomrule
  \end{tabular}
  \caption{Messwerte zur Bestimmung der Wärmekapazität}
  \label{tab:kalo}
\end{table}
Der gemittelte Wert für die spezifische Wärmekapazität beträgt dann
\begin{equation*}
  c_\su{g}m_\su{g} = (281\pm45)\,\si{\joule\per\kelvin}
\end{equation*}
Der dazugehörige Fehler berechnet sich über die Standardabweichung.
\newpage
\subsection{Spezifische Wärmekapazität der Proben}
Die spezifische Wärmekapazität wird mittels Formel \eqref{eq:probe} berechnet.
Die gemessenen Werte für Blei sind in der untenstehenden Tabelle \ref{tab:pb}
zu sehen.
Die zu messenden Proben haben dabei die Massen
\begin{align*}
m_\su{Pb} &= 385.33\gr \\
m_\su{Alu}&= 114.37\gr \\
m_\su{Gra}&= 107.98\gr
\end{align*}
\begin{table}
  \centering
  \begin{tabular}{c c c c c}
    \toprule
    $T_\su{Pb}\,/\Kel$ & $m_\su{w}\,/\gr$ & $T_\su{w}\,/\Kel
    $ & $T_\su{m}\,/\Kel$ & $c_\su{Pb}$\\
    \midrule
    373.15 & 521.42 & 294.64 & 297.87 & 255.51 \\
    373.15 & 514.09 & 294.15 & 296.38 & 185.06 \\
    373.15 & 510.64 & 293.90 & 297.13 & 269.10 \\
    \bottomrule
  \end{tabular}
  \caption{Messwerte Blei}
  \label{tab:pb}
\end{table}
Aus diesen Werten ergibt sich eine mittlere Wärmekapazität von
\begin{equation*}
  c_\su{Pb}=(242\pm1)\,\si{\joule\per\kelvin}
\end{equation*}
Der Fehler ergibt sich auch hier aus dem Mittelwert der Fehlerfortpflanzung
welche mit
\begin{equation}
  \Delta c_\su{k} = \sqrt{\left(\frac{(T_\su{m}-T_\su{w})}{m_\su{k}
  (T_\su{k}-T_\su{w})}\right)^2\sigma_\su{Kalorimeter}^2}
  \label{eq:gauss}
\end{equation}
berechnet wird.
Die gemessenen Werte für Aluminium sind in Tabelle \ref{tab:alu} zu sehen.
\begin{table}
  \centering
  \begin{tabular}{c c c c c}
    \toprule
    $T_\su{Alu}\,/\Kel$ & $m_\su{w}\,/\gr$ & $T_\su{w}\,/\Kel
    $ & $T_\su{m}\,/\Kel$ & $c_\su{Alu}$\\
    \midrule
    373.15 & 514.82 & 294.64 & 298.12 & 996.79\\
    373.15 & 512.44 & 294.15 & 297.87 & 1057.70\\
    373.15 & 514.44 & 293.15 & 296.63 & 976.75\\
    \bottomrule
  \end{tabular}
  \caption{Messwerte Aluminium}
  \label{tab:alu}
\end{table}
Im Mittel ergibt sich für die Wärmekapazität ein Wert von
\begin{equation*}
  c_\su{Alu}=(1000.12\pm18.53)\,\si{\joule\per\kelvin}
\end{equation*}
Auch hier wird der Fehler aus dem Mittelwert der Fehler nach Formel \eqref{eq:gauss}
berechnet.
Die aufgenommenen Werte für Graphit sind in Tabelle \ref{tab:alug}
zu finden.
\begin{table}
  \centering
  \begin{tabular}{c c}
    \toprule
    \hrulefill & Graphit \\
    \midrule
    % $m_\su{Probe}\,/\gr$ &
    $T_\su{Graphit}\,/\Kel$& 373.15 \\
    $m_\su{w}\,/\gr$  & 513.79 \\
    $T_\su{w}\,/\Kel$ & 293.40 \\
    $T_\su{m}\,/\Kel$    & 297.87 \\
    \bottomrule
  \end{tabular}
  \caption{Messwerte für Graphit}
  \label{tab:alug}
\end{table}
Daraus ergibt sich nach Formel \eqref{eq:probe} die folgende Wärmekapazität:
\begin{equation}
  c_\su{Graphit}= (1335.51\pm24.75)\,\si{\joule\per\kelvin}
\end{equation}
Die Formel für den Fehler ergibt sich
aus der Gauß'schen Fehlerfortpflanzung.
\subsection{Molwärme der Proben}
Die Molwärme der Proben wird mit Formel \eqref{eq:mol} und der Umstellung
\begin{align*}
  V&=\frac{M}{\rho}\\
  C_\su{P} &= c_\su{k}M
\end{align*}
berechnet. Werte für $M$, $\alpha$, $\rho$ und $\kappa$ sind aus der Anleitung
\cite{201} entnommen und in Tabelle \ref{tab:mol} aufgeführt.
\begin{table}
  \centering
  \begin{tabular}{c | c c c c c}
    \toprule
    Probe & $m\,/\,\si{\gram\per\mol}$&$m\,/\,\si{\gram}$ &$\alpha\,/\,10^{-6}\si{1\per\kelvin}$
    &$\kappa\,/\,\si{\newton\per\meter}$&$\rho\,/\,\si{\kilo\gram\per\cubic
    \meter}$ \\
    \midrule
    Blei      & 207.2 & 385.33 & 29.0 & 42 & 11.35 \\
    Aluminium &  27.0 & 114.37 & 23.5 & 75 &  2.70 \\
    Graphit   &  12.0 & 107.98 &  8.0 & 33 &  2.25 \\
    \bottomrule
  \end{tabular}
  \caption{Werte der Proben}
  \label{tab:mol}
\end{table}
Mit der Mischtemperaturen aus den Tabellen \ref{tab:pb} und \ref{tab:alug}
ergibt sich für Graphit eine mittlere Molwärme von
\begin{equation*}
  C_\su{Gra}= (16\pm0.3)\,\si{\joule\per\mol}
\end{equation*}
und für Aluminium und Blei
\begin{align*}
  C_\su{Alu} &= (27.0\pm0.5)\,\si{\joule\per\mol} \\
  C_\su{Pb} &= (48.31\pm1.0)\,\si{\joule\per\mol}
\end{align*}
Hier wird die Mischtemperatur gemittelt, sodass:
\begin{align*}
  T_\su{m, Alu} &= (297.5\pm0.7)\Kel \\
  T_\su{m, Pb}  &= (297.1\pm0.6)\Kel
\end{align*}
Für Graphit wird der Fehler mit
\begin{equation*}
  \Delta C_\su{k}=\sqrt{m^2\sigma_\su{c_k}^2}
\end{equation*}
berechnet und der Fehler für Blei und Aluminium berechnet sich mit:
\begin{equation*}
  \Delta C_\su{k}=\sqrt{M^2\sigma_\su{c_k}^2+(9\alpha^2\kappa\frac{M}{\rho})^2
  \sigma_\su{T_m}^2}
\end{equation*}
