\begin{table}
  \centering
  \begin{tabular}{c c c c}
    \toprule
    Körper & $c_\su{ermittelt}\,/\,\si{\joule\per\gram\per\kelvin}$ &
    $c_\su{Literatur}\,/\,\si{\joule\per\gram\per\kelvin}$\cite{bleialu} & $\text{Abweichung in}\,\%$ \\
    \midrule
    Alu & 1.010 & 0.896 & 22.9 \\
    Blei & 0.242 & 0.129 & 87.6 \\
    Graphit & 1.349 & 0.715 & 88.7 \\
    \bottomrule
  \end{tabular}
  \caption{Abweichung der spezifischen Wärmekapazität}
\end{table}
Die Wärmekapazität von Aluminium\cite{bleilu} stimmt ungefähr mit den Literaturwerten überein,
jedoch weichen Blei\cite{bleialu} und Graphit \cite{graphit} sehr stark ab.
Dies liegt unter anderem daran,
dass wir angenommen haben, dass wir das Material auf $100\,\si{\celsius}$ erhitzen,
wir haben jedoch nicht nachgemessen, ob der Körper diese Temperatur erreicht hat.
Möglicherweise stimmte diese Näherung nicht exakt, da wir den Körper vielleicht
nicht lange genug erhitzt haben.

\begin{table}
  \centering
  \begin{tabular}{c c c c}
    \toprule
    Körper & $C_\su{ermittelt}\,/\,\si{\joule\per\mol\per\kelvin}$ &
    $C_\su{3R}\,/\,\si{\joule\per\mol\per\kelvin}$ & $\text{Abweichung in} \,\%$ \\
    \midrule
    Alu & 27.2 & 24.94 & 9.1 \\
    Blei & 50.1 & 24.94 & 100.1 \\
    Graphit & 16.2 & 24.94 & 35.0 \\
    \bottomrule
  \end{tabular}
  \caption{Abweichung der Molwärme}
\end{table}
Das Dulong-Petitsche-Gesetz besagt, dass die Molwärme $ C = 3 R = 24.94 \,\si{\joule\per\mol\per\kelvin}$
beträgt\,\cite{chemie}. Bei Aluminium stimmt diese ungefähr mit den experimentell
ermittelten Werten überein, bei Graphit ist die Abweichung im Rahmen der Messungenauigkeiten
auch noch in Ordnung, jedoch weicht sie bei Blei deutlich zu stark ab. Auffällig ist hierbei,
dass die Abweichungen eigentlich genau andersherum verteilt sein müssten, da dass Gesetz
eher für große Atommassen zutrifft, als für kleine. Somit müssten wir normalerweise für Blei
die geringste Abweichung und für Graphit die größte Abweichung erhalten.
