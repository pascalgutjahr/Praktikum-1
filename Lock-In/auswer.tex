\section{Auswertung Teil 1}

In Aufgabenteil 1 wird das Signal, genauer: die Spannung, des Funktionsgenerator untersucht.
Eine konstante Spannung von $U_\text{osc} = 4.6\si{\volt}$ liefert der Oscillator Ausgang. Dabei ist die Frequenz variabel und
es wurde peak-to-peak (U$_\text{SS}$) gemessen, also 2 mal die Amplitude. Eine variable Spannungsamplitude liefert
der Reference Ausgang.

Auffällig und nicht typisch ist außerdem, dass sich beim Drehen des Phasenschiebers (Phase Shifter) die sich Amplitude des Oscillator
Signals ändert. Diese sollte nicht phasenabhängig sein.

\begin{table}[!h]
  \centering
\begin{subtable}{0.3\textwidth}
  \begin{tabular}{SS}
    \toprule
    {$\Delta\varphi$} &
    {$U_\text{SS} \,/\, \si{\volt}$} \\
    \midrule
    0\si{\degree} & 2.84 \\
    45\si{\degree} & 2.28 \\
    90\si{\degree} & 2.60 \\
    135\si{\degree} & 2.14 \\
    180\si{\degree} & 2.80 \\
    \bottomrule
  \end{tabular}
  \caption{ohne Tiefpass}
\end{subtable}
\quad
\begin{subtable}{0.3\textwidth}
  \begin{tabular}{SSS}
    \toprule
    {$\Delta\varphi$} &
    {$U_\text{max} \,/\, \si{\volt}$} &
    {$U_\text{out} \,/\, \si{\volt}$} \\
    \midrule
    0\si{\degree} & 1.68  & 1.07  \\
    15\si{\degree} & 1.64  & 1.01  \\
    30\si{\degree} & 1.40  & 0.77  \\
    45\si{\degree} & 1.09  & 0.49  \\
    60\si{\degree} & 0.50  & 0.16  \\
    75\si{\degree} & 0.08  & 0.01  \\
    90\si{\degree} & -0.12 & 0     \\
    105\si{\degree} & -0.35 & 0.06  \\
    120\si{\degree} & -0.78 & 0.25  \\
    135\si{\degree} & -1.25 & 0.56  \\
    \bottomrule
  \end{tabular}
  \caption{mit Tiefpass}
  \label{tab:oRmT}
\end{subtable}
\caption{unverrauschte Messwerte, Tabelle \ref{tab:oRmT} mit Tiefpassfilter.}
\quad
\hfill
\end{table}


\begin{table}[!h]
  \centering
\begin{subtable}{0.3\textwidth}
  \begin{tabular}{SS}
    \toprule
    {$\Delta\varphi$} &
    {$U_\text{SS} \,/\, \si{\volt}$} \\
    \midrule
    0\si{\degree} & 2.84  \\
    45\si{\degree} & 2.30  \\
    90\si{\degree} & 2.58  \\
    135\si{\degree} & 2.14  \\
    180\si{\degree} & 2.84  \\
    \bottomrule
  \end{tabular}
  \caption{ohne Tiefpass}
\end{subtable}
\quad
\begin{subtable}{0.3\textwidth}
  \begin{tabular}{SSS}
    \toprule
    {$\Delta\varphi$} &
    {$U_\text{max} \,/\, \si{\volt}$} &
    {$U_\text{out} \,/\, \si{\volt}$} \\
    \midrule
    0 \si{\degree} & -1.60 & -1.02   \\
    15 \si{\degree} & -1.63 & -1.00  \\
    30 \si{\degree} & -1.40 & -0.77  \\
    45 \si{\degree} & -1.09 & -0.49  \\
    60 \si{\degree} & -0.53 & -0.17  \\
    75 \si{\degree} & -0.02 & 0      \\
    90 \si{\degree} & 0.13  & 0      \\
    105 \si{\degree} & 0.41  & -0.07 \\
    120 \si{\degree} & 0.78  & -0.25 \\
    135 \si{\degree} & 1.20  & -0.54 \\
    \bottomrule
  \end{tabular}
  \caption{mit Tiefpass}
  \label{tab:mRmT}
\end{subtable}
\caption{verrauschte Messwerte, Tabelle \ref{tab:mRmT} mit Tiefpassfilter, Nebenbedingungen: Signal Attenuator = 1,
        Noise Amplitude = 1$\times 10^{-3}$.}
\quad
\hfill
\end{table}

Die Werte für $U_\text{out}$ werden mit Gleichung \eqref{eqn:Uout} berechnet, $\Delta\varphi$ beschreibt dabei die Phasendifferenz zwischen
Reference- und Oscillatorsignal. $U_\text{SS}$ und $U_\text{max}$ werden am \emph{Low-Pass-Filter Output} abgelesen und
verstärkungsbereinigt.
\begin{figure}[!h]
\begin{minipage}[t]{0.3\textwidth}
\includegraphics[width=\textwidth]{MoRoT/Messung_phi=180.JPG}
\label{fig:1}
\caption{$\Delta\varphi = 0\si{\degree}$}
\end{minipage}
\hspace{10pt}
\vspace{5pt}
\begin{minipage}[t]{0.3\textwidth}
\includegraphics[width=\textwidth]{MoRoT/Messung_phi=225.JPG}
\label{fig:2}
\caption{$\Delta\varphi = 45\si{\degree}$}
\end{minipage}
\hspace{10pt}
\vspace{5pt}
\begin{minipage}[t]{0.3\textwidth}
\includegraphics[width=\textwidth]{MoRoT/Messung_phi=270.JPG}
\label{fig:3}
\caption{$\Delta\varphi = 90\si{\degree}$}
\end{minipage}
\hspace{10pt}
\vspace{5pt}
\begin{minipage}[t]{0.3\textwidth}
\includegraphics[width=\textwidth]{MoRoT/Messung_phi=315.JPG}
\label{fig:4}
\caption{$\Delta\varphi = 135\si{\degree}$}
\end{minipage}
\hspace{12pt}
\vspace{5pt}
\begin{minipage}[t]{0.3\textwidth}
\includegraphics[width=\textwidth]{MoRoT/Messung_phi=360.JPG}
\label{fig:5}
\caption{$\Delta\varphi = 180\si{\degree}$}
\end{minipage}
\hspace{12pt}
\vspace{5pt}
\end{figure}

\begin{itemize}
  \item Abbildung \ref{fig:1}: Annäherung des Signals durch eine Fourierreihe.
  \item Abbildung \ref{fig:2}: Übergang durch Phasenänderung um 45\si{\degree}.
  \item Abbildung \ref{fig:3}: Weitere Phasenänderung um 45\si{\degree}, symmetrisch zur x-Achse.
  \item Abbildung \ref{fig:4}: Fast vollständige Umkehrung, ingesamt um 135\si{\degree} gedreht.
  \item Abbildung \ref{fig:5}: Komplette Drehung um 180\si{\degree}, gespiegelt an der $x$-Achse zum Ausgangssignal.
\end{itemize}

Oben zu sehen sind die Aufzeichnungen des Oszilloskopes ohne den Tiefpassfilter.
\newpage

\begin{figure}[!h]
\begin{minipage}[t]{0.3\textwidth}
\includegraphics[width=\textwidth]{MmRoT/Messung_phi=180.JPG}
\label{fig:6}
\caption*{$\Delta\varphi = 0\si{\degree}$}
\end{minipage}
\hspace{10pt}
\vspace{5pt}
\begin{minipage}[t]{0.3\textwidth}
\includegraphics[width=\textwidth]{MmRoT/Messung_phi=225.JPG}
\label{fig:7}
\caption*{$\Delta\varphi = 45\si{\degree}$}
\end{minipage}
\hspace{10pt}
\vspace{5pt}
\begin{minipage}[t]{0.3\textwidth}
\includegraphics[width=\textwidth]{MmRoT/Messung_phi=270.JPG}
\label{fig:8}
\caption*{$\Delta\varphi = 90\si{\degree}$}
\end{minipage}
\hspace{10pt}
\vspace{5pt}
\begin{minipage}[t]{0.3\textwidth}
\includegraphics[width=\textwidth]{MmRoT/Messung_phi=315.JPG}
\label{fig:9}
\caption*{$\Delta\varphi = 135\si{\degree}$}
\end{minipage}
\hspace{12pt}
\vspace{5pt}
\begin{minipage}[t]{0.3\textwidth}
\includegraphics[width=\textwidth]{MmRoT/Messung_phi=360.JPG}
\label{fig:10}
\caption*{$\Delta\varphi = 180\si{\degree}$}
\end{minipage}
\hspace{12pt}
\vspace{5pt}
\end{figure}

Für die Messwerte mit Rauschen und ohne Tiefpassfilter folgt die analoge Beschreibung der Grafiken. Anzumerken ist hier,
dass sich die Graphen gespiegelt zur $x$-Achse verhalten. \\

Im nächsten Auswertungsteil geht es um die Abhängigkeit der Ausgangsspannung zur Phasenverschiebung.
Dazu wird zunächst das nicht verrauschte Signal verwendet. Die Messdaten zu der Versuchsreihe sind Tabelle \ref{tab:oRmT}
zu entnehmen. Die Phasendifferenz wird für eine bessere Darstellung gegen die Ausgangsspannung U$_\text{out}$ geplottet.
\begin{figure}[!h]
  \centering
  \includegraphics[width=1.0\textwidth]{Plots/Plot_Messung_oR.pdf}
  \caption{Ausgangsspannung U$_\text{out}$ in Abhängigkeit von der Phasendifferenz $\Delta\varphi$.}
  \label{fig:Uout}
\end{figure}
