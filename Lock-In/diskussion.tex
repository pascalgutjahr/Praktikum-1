Zu Beginn ergibt sich die Schwierigkeit  $U = 10 \mV$ einzustellen, da die Skala zum
einen nicht für so kleine Spannungen ausgelegt ist und zum anderen schwankt die
Spannung ab einem Wert von ca. $12 \mV$ so stark, dass genaues Ablesen nicht mehr möglich ist.

Im ersten Versuchsteil ist auffällig, dass die Spannungsverläufe leicht variieren,
wenn die Phase verändert wird. Dies darf jedoch nicht sein, da die Amplitude nicht von
der Phasendifferenz zwischen Signal- und Referenzspannung abhängen darf. Möglicherweise
wurde das Bauteil falsch verkabelt. Unter Berücksichtigung der Messungenauigkeiten des Oszilloskops
bleibt die Frequenz gleich.

Im zweiten Teil der Auswertung ist zu erkennen, dass beide Graphen die zu erwartende
Ausgleichsgerade ergeben, wenn $U_\text{max}$ gegen $\cos{(\Delta\varphi)}$ geplottet wird.
Aus dem Vergleich der beiden Graphen lässt sich schließen, dass der Rauschgenerator
dafür sorgt, dass die gemessenen $U_\text{max}$ Werte geringer ausfallen.

Bei der Lichtmessung mit der Photodiode ergibt sich, wie laut Theorie bereits erwartet,
dass die Spannung $U$ proportional zum Abstand $1/r^2$ abfällt. Hier muss jedoch
berücksichtigt werden, dass die Photodiode nicht nur Lichtimpulse von der LED
empfängt, sondern auch von der Umgebung, da der Versuch nicht in einem
abgedunkelten Raum durchgeführt wird, sondern direkt am Fenster.  Dies ruft somit
ein Rauschen hervor.
