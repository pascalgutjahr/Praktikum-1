Im ersten Versuchsteil ist auffällig, dass die Spannungsverläufe leicht variieren,
wenn die Phase verändert wird. Unter Berücksichtigung der Messungenauigkeiten des Oszillators
bleibt die Frquenz gleich, jedoch ändern sich die peak-to-peak gemessenen Spannungsamplituden.
Die Messwerte ohne Rausch-Generator variieren dabei weniger stark, als die Werte
mit Rausch-Generator.

Im zweiten Teil des Versuchs lassen die Spannungskurven einen cosinusförmigen Verlauf erkennen.
Die Graphen mit und ohne Rauschgenerator ähneln sich hierbei sehr stark, was vermuten lässt,
dass die Messung nicht exakt durchgeführt wurde, da der Rausch-Generator ansonsten keine Veränderung an
den Spannungsverläufen vornehmen würde.

Bei der Lichtmessung mit der Photodiode ergibt sich, dass die Spannung $U$
proportional zum Abstand $1/r^2$ abfällt. Hier muss jedoch berücksichtigt werden, dass
die Photodiode nicht nur Lichtimpulse von der LED empfängt, sondern auch von der
Umgebung, da der Versuch nicht in einem abgedunkelten Raum durchgeführt wird. 
