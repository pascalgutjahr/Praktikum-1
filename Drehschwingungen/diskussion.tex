In diesem Versuch müssen einige Fehlerquellen berücksichtigt werden. Zum einen war es nicht
möglich, dass sich die Kugel nur um die eigene Achse dreht, da sie zusätzlich immer
ein wenig in Schwingung versetzt war. Zudem gab es manchmal Aussetzer bei der Zeitmessung,
da der Lichtimpuls nicht immer im richtigen Winkel reflektiert wurde.
Des Weiteren ist uns aufgefallen, dass es entscheident ist, wie rum die Kugel an den Faden
gehängt wird. Zeigt die Schraube nach rechts, wenn man sich vor dem Versuchsaufbau befindet,
so verkürzen sich die Schwingungsdauern. Zeigt die Schraube jedoch nach links, so
erhöhen sich die gemessenen Zeiten.
