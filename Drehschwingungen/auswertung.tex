\subsection{Bestimmung des Schubmoduls}
Tabelle \ref{tab:data1} zeigt die Periodendauern der Schwingungen für den ersten
Teil des Versuchs.
\begin{table}
  \centering
  \begin{tabular}{c}
    \toprule
    $\su{T}$\,/\,s \\
    \midrule
    20.008 \\
    20.000 \\
    20.009 \\
    20.009 \\
    20.005 \\
    19.993 \\
    19.989 \\
    20.004 \\
    20.011 \\
    20.009 \\
    \bottomrule
  \end{tabular}
  \caption{Messwerte der Periodendauer ohne Helmholtzspulen}
  \label{tab:data1}
\end{table}
Aus diesen Werten wird der Mittlwert bestimmt, mit dem die weiteren Rechnungen
durchgeführt werden. Die gemittelte Periodendauer hat somit einen Wert von
\begin{equation*}
  T = (20.004 \pm 0.007)\sek
\end{equation*}
Der Draht besitzt die Abmessungen:
\begin{align*}
  L = 66.5 \cm
  r = 0.0965 \mm
\end{align*}
Die übrigen benötigten Messgrößen sind
\begin{align*}
  m_\su{k} &= (0.512 \pm 2.1\cdot10^{-4})\kg \\
  R_\su{K} &= (25.4 \pm 0.002)\cdot10^{-3}\mt\\
  \Theta   &=  2.25\cdot10^{-6}\,\si{\kilo\gram\square\meter}\\
  N_\su{Spule} &= 390\\
  R_\su{Spule} &= 78\cdot10^{-3}\mt\\
  I_\su{max,Spule} &= 1.4\Amp
\end{align*}
Diese lassen sich am Versuchsaufbau ablesen %Formulierung
Das Schubmodul $G$ lässt sich nun nach Formel \eqref{eq:schub} berechnen. Der
zugehörige Fehler ergibt sich nach Gauß'scher Fehlerfortpflanzung durch:
\begin{equation*}
  \Delta G = \sqrt{\left(\frac{-32\pi m_\su{k} R_\su{k}^2 L}{5T^3 r^4}\right)^2
  \cdot\sigma_\su{T} + \left(\frac{16\pi R_\su{k}^2L}{5T^2r^4}\right)^2\cdot
  \sigma_\su{m_k}^2+\left(\frac{32\pi m_\su{k}R_\su{k}L}{5T^2R^4}\right)^2\cdot
  \sigma_\su{R_k}^2}
\end{equation*}
Somit hat das Schubmodul einn Wert von
\begin{equation*}
  G = (6.364 \pm 0.005)\cdot10^{10}\,\si{\newton\per\square\meter}
\end{equation*}
\subsection{Bestimmung des magnetischen Moments}
Tabelle \ref{tab:helm} zeigt die Periodendauern in einem homogenen B-Feld. Das
B-Feld wird durch den anliegenden Strom reguliert.
\begin{table}[H]
  \centering
  \begin{tabular}{c| c c c c |c}
    \toprule
    $I\,/\,\si{\ampere}$ & $T_1\,/\,\si{\second}$ & $T_2\,/\,\si{\second}$ & $T_3\,/\,\si{\second}$ & $T_4\,/\,\si{\second}$
    & $T_\su{mittel}\,/\,\si{\second}$\\
    \midrule
    0.1 & 14.247 & 14.172 & 14.105 & 14.085 & 14.15 ± 0.06 \\
    0.2 & 11.123 & 11.076 & 11.007 & 11.124 & 11.08 ± 0.05\\
    0.3 &  9.653 & 9.645 & 9.650 & 9.644 &    9.648 ± 0.004  \\
    0.4 &  8.622 & 8.622 & 8.621 & 8.612 &    8.619 ± 0.004  \\
    0.5 &  7.873 & 7.871 & 7.866 & 7.846 &    7.86 ± 0.01  \\
    0.6 &  7.191 & 7.199 & 7.231 & 7.227 &    7.21 ± 0.02  \\
    0.7 &  6.902 & 6.851 & 6.852 & 6.849 &    6.86 ± 0.02  \\
    0.8 &  6.504 & 6.411 & 6.533 & 6.549 &    6.50 ± 0.05  \\
    0.9 &  6.184 & 6.194 & 6.070 & 5.997 &    6.11 ± 0.08  \\
    1.0 &  5.803 & 4.987 & 5.875 & 5.789 &    5.6  ± 0.4  \\
    \bottomrule
  \end{tabular}
  \caption{Schwingungsdauern mit Helmholtzspulen}
  \label{tab:helm}
\end{table}

Auch hier wird mit der gemittelten Periodendauer weiter gerechnet.
mithilfe der Formel
\begin{equation*}
  B=\mu_0\frac{8\cdot I\cdot N}{\sqrt{125}R_\su{Spule}}
\end{equation*}
lässt sich das Magnetfeld zu verschiedenen Strömen berechnen \cite{bfeld}.
Diese sind in der untenstehenden Tabelle \ref{tab:bfeld} zu sehen.
\begin{table}
  \centering
  \begin{tabular}{c c}
    \toprule
    I\,/\,A & B\,/\,$\si{\milli\tesla}$\\
    \midrule
    0.1 &  0.5 \\
    0.2 &  0.9 \\
    0.3 &  1.4 \\
    0.4 &  1.8 \\
    0.5 &  2.3 \\
    0.6 &  2.7 \\
    0.7 &  3.2 \\
    0.8 &  3.6 \\
    0.9 &  4.1 \\
    1.0 &  4.5 \\
    \bottomrule
  \end{tabular}
  \caption{Magnetfelder für verschiedene Ströme.}
  \label{tab:bfeld}
\end{table}
Die Magnetfelder werden wie in Abbildung \ref{fig:Bplot} gegen $\frac{1}{T^2}$
aufgetragen.
\begin{figure}
  \includegraphics{'bilder/Bplot.pdf'}
  \caption{}
  \label{fig:Bplot}
\end{figure}
Die Parameter der linearen Regression werden mit Python berechnet.
\begin{align*}
  a &= (0.159   \pm 0.009 )\,\si{\tesla\square\second}
  b &= (-0.0003 \pm 0.0002)\,\si{\tesla}
\end{align*}
