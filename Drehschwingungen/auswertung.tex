\subsection{Bestimmung des Schubmoduls}
Tabelle \ref{tab:data1} zeigt die Periodendauern der Schwingungen für den ersten
Teil des Versuchs.
\begin{table}
  \begin{tabular}{c}
    \toprule
    $\su{T}$\,/\,s \\
    \midrule
    20.008 \\
    20.000 \\
    20.009 \\
    20.009 \\
    20.005 \\
    19.993 \\
    19.989 \\
    20.004 \\
    20.011 \\
    20.009 \\
    \bottomrule
  \end{tabular}
  \caption{Messwerte der Periodendauer ohne Helmholtzspulen}
  \label{tab:data1}
\end{table}
Aus diesen Werten wird der Mittlwert bestimmt, mit dem die weiteren Rechnungen
durchgeführt werden. Die gemittelte Periodendauer hat somit einen Wert von
\begin{equation*}
  T = (20.004 \pm 0.007)\sek
\end{equation*}
Der Draht besitzt die Abmessungen:
\begin{align*}
  L = 66.5 \cm
  r = 0.0965 \mm
\end{align*}
Die übrigen benötigten Messgrößen sind
\begin{align*}
  m_\su{k} &= (0.512 \pm 2.1\cdot10^{-4})\kg \\
  R_\su{K} &= (25.4 \pm 0.002)\cdot10^{-3}\mt\\
  \Theta   &=  2.25\cdot10^{-6}\,\si{\kilo\gram\square\meter}\\
  N_\su{Spule} &= 390\\
  R_\su{Spule} &= 78\cdot10^{-3}\mt\\
  I_\su{max,Spule} &= 1.4\Amp
\end{align*}
Diese lassen sich am Versuchsaufbau ablesen %Formulierung
Das Schubmodul $G$ lässt sich nun nach Formel \eqref{eq:schub} berechnen. Der
zugehörige Fehler ergibt sich nach Gauß'scher Fehlerfortpflanzung durch:
\begin{equation*}
  \Delta G = \sqrt{\left(\frac{-32\pi m_\su{k} R_\su{k}^2 L}{5T^3 r^4}\right)^2
  \cdot\sigma_\su{T} + \left(\frac{16\pi R_\su{k}^2L}{5T^2r^4}\right)^2\cdot
  \sigma_\su{m_k}^2+\left(\frac{32\pi m_\su{k}R_\su{k}L}{5T^2R^4}\right)^2\cdot
  \sigma_\su{R_k}^2}
\end{equation*}
Somit hat das Schubmodul einn Wert von
\begin{equation*}
  G = (6.364 \pm 0.005)\cdot10^{10}\,\si{\newton\per\square\meter}
\end{equation*}
