\subsection{Bestimmung des Schubmoduls}
Tabelle \ref{tab:data1} zeigt die Periodendauern der Schwingungen für den ersten
Teil des Versuchs.
\begin{table}[H!]
  \begin{tabular}{c}
    \toprule
    \su{T}\,/\,s \\
    \midrule
    20.008 \\
    20.000 \\
    20.009 \\
    20.009 \\
    20.005 \\
    19.993 \\
    19.989 \\
    20.004 \\
    20.011 \\
    20.009
    \bottomrule
  \end{tabular}
  \caption{Messwerte der Periodendauer ohne Helmholtzspulen}
  \label{tab:data1}
\end{table}
Aus diesen Werten wird der Mittlwert bestimmt, mit dem die weiteren Rechnungen
durchgeführt werden. Die gemittelte Periodendauer hat somit einen Wert von
\begin{equation*}
  T = (20.004 \pm 0.007)\sek
\end{equation*}
Der Draht besitzt die Abmessungen:
\begin{align*}
  L = 66.5 \cm
  r = 0.0965 \mm
\end{align*}
Die übrigen benötigten Messgrößen sind
\begin{align*}
  m_\su{k} &=  \\
  R_\su{K} &=  \\
  \Theta   &=  \\
  N_\su{Spule} &= \\
  R_\su{Spule} &= \\
  I_\su{max,Spule} &= 
\end{align*}
Diese lassen sich am Versuchsaufbau ablesen %Formulierung
Das Schubmodul $G$ lässt sich nun nach Formel \eqref{eq:schub} berechnen. Der
zugehörige Fehler ergibt sich nach Gauß'scher Fehlerfortpflanzung durch:
\begin{equation*}
  \Delta G = \sqrt{\frac{m_\su{k} R_\su{k}^2 L}{T^3 r^4}}
\end{equation*}
Somit hat das Schubmodul einn Wert von
\begin{equation*}
  G = ( \pm )\,\si{\kilo\gram\per\square\second\meter}
\end{equation*}
