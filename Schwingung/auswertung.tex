Die gegebenen Werte der einzelnen Elemente lauten:
\begin{align*}
  L &= (16,8 \pm 0,1) \,\si{\milli\henry} \\
  C &= (2,07 \pm 0,01) \,\si{\nano\farad} \\
  R_1 &= (67,2 \pm 0,2) \Ohm \\
  R_2 &= (682 \pm 1) \Ohm
\end{align*}
Tabelle \ref{tab:amp} zeigt die gemessene Spannungsamplitude in Abhängigkeit der Zeit bei
einer gedämpften Schwingung.
\begin{table}
  \begin{tabular}{c c c c}
    \toprule
    \Delta t/\symup{\si{\micro\second}} & \Delta V/\Volt &
    \Delta t/\symup{\si{\micro\second}} & \Delta V/\Volt \\
    \midrule
     10 & 6,2 & 140 & 2,4  \\
     20 & 5,9 & 150 & 2,0  \\
     30 & 5,4 & 160 & 2,0  \\
     40 & 5,0 & 170 & 1,8  \\
     50 & 4,6 & 180 & 1,7  \\
     60 & 4,3 & 190 & 1,5  \\
     70 & 3,9 & 200 & 1,4  \\
     80 & 3,7 & 210 & 1,3  \\
     90 & 3,3 & 220 & 1,2  \\
    100 & 3,2 & 230 & 1,1  \\
    110 & 2,8 & 240 & 1,0  \\
    120 & 2,8 & 250 & 0,8  \\
    130 & 2,4 & \hrulefill & \hrulefill \\
    \bottomrule
  \end{tabular}
  \caption{Zeitabhäängigkeit der Amplitude}
  \label{tab:amp}
\end{table}

Hieraus ergibt sich dann das Bild, welches in Abbildung \ref{fig:AMP} zu sehen ist.
\begin{figure}[h]
  \centering
  \includegraphics[width=0.8\textwidth]{Bilder/amp.pdf}
  \caption{Einhüllende der gedämpften Schwingung}
  \label{fig:AMP}
\end{figure}
\\
Da $U(t)\propto I(t)$ gilt, ist die Form der Einhüllenden nach Formel
(\ref{eqn:einh}) durch
\begin{equation}
  A = A_0\su{e}^{-2\pi\mu t}
\end{equation}
gegeben.
Mit der Ausgleichsrechnung ergeben sich die Werte:
\begin{align*}
  A_0 &= (6,83 \pm 0,05)\Volt\\
  \mu &= (1251,3 \pm 13,6)\,\frac{1}{\sek}.
\end{align*}
Die Werte für $T_\su{ex}$ und $R_\su{eff}$ ergeben sich nach Formel \eqref{eqn:reff}
und \eqref{eqn:tex} und lauten:
\begin{align*}
  R_\su{eff} &= (264 \pm 2)\Ohm \\ % Wert erneut überprüfen
  T_\su{ex}  &= (1,27 \pm 0,01)\,\cdot 10^{-4}\sek
\end{align*}
%Platzhalter für Einhüllende

Für die Weiteren Versuchsteile werden andere Elemente mit den Werten
\begin{align*}
  L &= (3,53 \pm 0,03) \,\si{\milli\henry} \\
  C &= (5,015 \pm 0,015) \,\si{\nano\farad} \\
  R_1 &= (30,3 \pm 0,1) \Ohm \\
  R_2 &= (271,6 \pm 0,3) \Ohm
\end{align*}
verwendet.
Der gemessene Wert für den Widerstand $R_\su{ap}$, bei dem der aperiodische Fall
eintritt, beträgt $2,3\,\si{\kilo\ohm}$. Der nach Formel \eqref{eqn:r_ap}
berechnete Wert liegt jedoch bei $(1,678 \pm 0,007)\,\si{\kilo\ohm}$.
Die Daten zur Bestimmung der Resonanzüberhöhung $q$ und die Breite der
Resonanzkurve $\nu_+ - \nu_-$ finden sich in Tabelle \ref{tab:Ucon} wieder.
\begin{table}
  \centering
  \begin{tabular}{c c}
    \toprule
    $\nu/\Hz$ & $U_{ss}$/\Volt \\
    \midrule
     15000 &  8,0  \\
     20000 &  8,1  \\
     25000 & 11,0  \\
     30000 & 14,2  \\
     31000 & 15,0  \\
     32000 & 15,7  \\
     33000 & 16,4  \\
     34000 & 17,0  \\
     35000 & 17,2  \\
     36000 & 17,4  \\
     37000 & 17,1  \\
     38000 & 16,5  \\
     39000 & 15,8  \\
     40000 & 14,9  \\
     45000 & 10,1  \\
     50000 &  7,0  \\
    \bottomrule
  \end{tabular}
  \caption{Kondensatorspannung in Abhängigkeit der Frequenz}
  \label{tab:Ucon}
\end{table}

\\
Die Erregerspannung $U$ beträgt dabei $6,72 \Volt$.
Das Verhältnis $\frac{U_\su{C}}{U}$ wird gegen $\nu$ in einem halblogarithmischen
Diagramm aufgetragen.
\begin{figure}[h]
  \centering
  \includegraphics[width=0.8\textwidth]{Bilder/UcUv.pdf}
  \caption{Darstellung der normierten Kondensatorspannung}
  \label{fig:UcUv}
\end{figure}
Der Maximalwert $q_\su{exp}$ wird aus dem Graph \ref{fig:AMP} als Güte abgelesen. Der
theoretische Wert für die Güte $q_\su{theo}$ wird nach Formel \eqref{eqn:guete} berechnet.
Somit erhält man:
\begin{align*}
  q_\su{exp} &= 2,6 \\
  q_\su{theo} &= 3,09 \pm 0,01
\end{align*}
Um die Breite der Resonanzfrequenz bestimmen zu können, wird der Frequenzbereich
um das Maximum nun linear, wie in Abbildung \ref{fig:UcUv} im unteren Graphen
zu sehen, dargestellt. Die Breite der Resonanzkurve wird abgelesen und mit dem
nach \eqref{eqn:breite} berechneten Theoriewert verglichen.
\begin{align*}
  (\nu_+ - \nu_-)_\su{exp} &= 10\kHz \\
  (\nu_+ - \nu_-)_\su{theo}&= (12,2 \pm 0,1)\kHz
\end{align*}
Um die Werte der Resonanzfrequenz $\nu_\su{res}$ berechnen zu können, werden
die Daten aus Tabelle \ref{tab:phase} benötigt.
\begin{table}
  \centering
  \begin{tabular}{c c}
    \toprule
    $\nu\,/\kHz$ & $\Delta t\,/\,\su{\si{\micro\second}}$ \\
    \midrule
    15    &    2,4   \\
    20    &    2,4   \\
    25    &    2,4   \\
    30    &    3,2   \\
    31    &    3,8   \\
    32    &    4,2   \\
    33    &    4,8   \\
    34    &    5,0   \\
    35    &    5,6   \\
    36    &    6,2   \\
    37    &    6,6   \\
    38    &    7,2   \\
    39    &    7,8   \\
    40    &    8,0   \\
    45    &    8,8   \\
    50    &    8,4   \\
    55    &    8,0   \\
    \bottomrule
  \end{tabular}
  \caption{Phase in Abhängigkeit der Phase}
  \label{tab:phase}
\end{table}

Diese Daten werden außerdem dazu verwendet, um die Frequenzen $\nu_1$ und
$\nu_2$ zu berechnen, an denen die Phase $\frac{\pi}{4}$ bzw.
$\frac{3\pi}{4}$ beträgt. Die Phase $\varphi$ wird durch
\begin{equation}
  \varphi_\su{rad} = 2\pi \cdot \nu \cdot \Delta t
\end{equation}
bestimmt, und wie in Abbildung \ref{fig:philog}
zu sehen, gegen die Frequenz $\nu$ aufgetragen.
\newpage
\begin{figure}[!h]
  \centering
  \includegraphics[width=0.7\textwidth]{Bilder/phasehalblog.pdf}
  \caption{Halblogarithmische Darstellung der Phasenverschiebung zw. Erreger-
  und Kondensatorspannung}
  \label{fig:philog}
\end{figure}
Dazu wird der Bereich in Abbildung \ref{fig:phse} um die Resonanzfrequenz genauer
betrachtet und linear dargestellt.

\begin{figure}[!h]
  \centering
  \includegraphics[width=0.7\textwidth]{Bilder/phaselinear.pdf}
  \caption{Lineare Darstellung der Phasenverschiebung zwischen Erreger- und
  Kondensatorspannung}
  \label{fig:phse}
\end{figure}
Die Frequenz $\nu_\su{res}$ wird nach Formel \eqref{eqn:wres} berechnet,
während $\nu_1$ und $\nu_2$ nach Formel \eqref{eqn:w12} berechnet werden.
Die berechneten Werte werden erneut mit den experimentell beobachteten Werten
Verglichen:
\begin{align*}
  \nu_\su{res,exp} &= 36\kHz \\
  \nu_\su{res,theo}&= 36,8 \kHz \\
  \nu_\su{1,exp} &= 29,9\kHz \\
  \nu_\su{1,theo}&=32,2\kHz \\
  \nu_\su{2,exp} &= 42,1 \kHz \\
  \nu_\su{2,theo}&= 44,4\kHz
\end{align*}

Die Werte für $\nu_\su{1,exp}$ und $\nu_\su{2,exp}$ ergeben sich durch
\begin{equation*}
  \nu_\su{1, 2, exp} = \nu_\su{res,exp} \mp \frac{1}{2}(\nu_+ - \nu_-)_\su{theo}
\end{equation*}
