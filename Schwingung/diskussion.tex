Anhand der vorliegenden Werte lässt sich feststellen, dass die gemessenen
Werte gut mit den berechneten Werten übereinstimmen. Die Einzige Ausnahme hierbei
betrifft den Effektivwiderstand $R_\su{eff}$. Der berechnete Widerstand weicht
hierbei um $195,8\Ohm$ ab. Dies lässt sich zum einen durch den Generatorwiderstand
erklären, aber auch durch die Widerstände der anderen Bauteile erklären, welche
in der Theorie nicht beachtet wurden. Dies erklärt auch die Diskrepanz bei dem
Widerstand, bei dem der aperiodische Fall eintritt. Hier liegt die Diskrepanz
jedoch nur bei $0,6\,10^3\Ohm$.
Tabelle \ref{tab:fehler} zeigt zusammengefasst die relativen Abweichungen der
gemessenen Werten.
\begin{table}
  \centering
  \begin{tabular}{c c c c}
    \toprule
    Messung  & Theoretischer Wert & experimentellerwert & relative Abweichung \\
     \midrule
     Güte & 3,09 & 2,6 & $15\,\si{\percent}$ \\
     Breite & $12,2\,\cdot 10^3\Hz$ & $10\,\cdot 10^3\Hz$ & $18\,\si{\percent}$ \\
     $\nu_\su{res}$ & $36,82\kHz$ & $36\kHz$ & $2\,\si{\percent}$ \\
     $\nu_1$ & $32,19\kHz$ & $29,9\kHz$ & $7\,\si{\percent}$ \\
     $\nu_2$ & $44,4\kHz$ & $42,1\kHz$ & $5\,\si{\percent}$ \\
     \bottomrule
  \end{tabular}
  \caption{Relative Abweichungen}
  \label{tab:fehler}
\end{table}
Anhand dieser Werte lässt sich feststellen, dass diese Werte relativ genau gemessen
wurden. Die bestehenden Abweichungen bestehen zum einen in den weiterhin nicht
beachteten Widerständen des Kondensators und der Spule. Weitere Ungenauigkeiten
kommen durch fehlerhaftes Ablesen und das ungenaue Einstellen des Cursors
am Oszilloskop.
Ein Fehler für $\nu_1, \nu_2 \text{und} \nu_\su{res}$ konnte nicht berechnet werden,
Da der Wert unter der Wurzel ein negatives Vorzeichen erhalten hat.
