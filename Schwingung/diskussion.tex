Anhand der vorliegenden Werte lässt sich feststellen, dass die gemessenen
Werte gut mit den berechneten Werten übereinstimmen. Die einzige Ausnahme hierbei
betrifft den Effektivwiderstand $R_\su{eff}$. Der berechnete Widerstand weicht
hierbei um $195,8\Ohm$ ab. Dies lässt sich zum einen durch den Generatorwiderstand
erklären, aber auch durch die Widerstände der anderen Bauteile, welche
in der Theorie nicht beachtet wurden. Dies erklärt auch die Diskrepanz bei dem
Widerstand, bei dem der aperiodische Fall eintritt. Hier liegt die Diskrepanz
jedoch nur bei $0,6\kOhm$.
Tabelle \ref{tab:fehler} zeigt zusammengefasst die relativen Abweichungen der
gemessenen Werten - mit der Formel
\begin{equation}
  \Delta f = \frac{x_\su{theo}-x_\su{exp}}{x_\su{theo}}.
\end{equation}
\begin{table}[!h]
  \centering
  \begin{tabular}{c c c c}
    \toprule
    Messung  & Theoretischer Wert & experimenteller Wert & relative Abweichung \\
     \midrule
     Güte & 3,09 & 2,6 & $15\,\si{\percent}$ \\
     Breite & $12,2\kHz$ & $10\kHz$ & $18\,\si{\percent}$ \\
     $\nu_\su{res}$ & $36,82\kHz$ & $36\kHz$ & $2\,\si{\percent}$ \\
     $\nu_\su{1}$ & $32,19\kHz$ & $29,9\kHz$ & $7\,\si{\percent}$ \\
     $\nu_\su{2}$ & $44,4\kHz$ & $42,1\kHz$ & $5\,\si{\percent}$ \\
     \bottomrule
  \end{tabular}
  \caption{Relative Abweichungen}
  \label{tab:fehler}
\end{table}

\noindent Die bestehenden Abweichungen bestehen zum Teil aus den weiterhin nicht
beachteten Widerständen des Kondensators und der Spule. Weitere Ungenauigkeiten
kommen durch fehlerhaftes Ablesen und das ungenaue Einstellen des Cursors
am Oszilloskop zustande.
Ein Fehler für $\nu_1$, $\nu_2$ und $\nu_\su{res}$ konnte nicht berechnet werden,
da der Wert unter der Wurzel ein negatives Vorzeichen erhalten hat.
