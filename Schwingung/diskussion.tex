Die Differenz des berechneten $R_\su{ap}$ weicht um etwa $0,6\,\cdot10^3 \Ohm$
vom gemessenen Widerstand ab. Diese Differenz kommt unter anderem durch die Widerstände
der restlichen Bauteile zustande, welche in der Theorie nicht beachtet wurden.
Eine weitere Erklärung für diese Diskrepanz liegt in der Ableseungenauigkeit
des eingestellten Grenzwiderstandes.

Für die Güte wurde relativ genau gemessen, da hier die relative Abweichung zum
theoretischen Wert nur um $15\,\si{\percent}$ abweicht.
Die Breite der Resonanzkurve wurde ebenfalls ziemlich genau gemessen, da hier
nur eine Abweichung von $18\si{\percent}$ berechnet wird.
Diese kleineren Fehler entstehen hauptächlich durch ungenaues Ablesen und das
ungenaue Einstellen des Cursors am Oszilloskop.


Der verbaute Widerstand $R_\su{1}$ weicht hier um $195,8\Ohm$ ab.
(---> muss noch an die richtige Stelle)
