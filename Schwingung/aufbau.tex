Für die erste Messung wird die Schaltung aus Abbildung (\ref{fig:schlt1})
aufgebaut.
\begin{figure}[h]
  \centering
  \includegraphics{Bilder/dämpfschalt.JPG}
  \caption{Schaltung für gedämpfte Schwingung}
  \label{fig:schlt1}
\end{figure} \\
Mit Hilfe dieses Aufbaus soll der Abfall der Amplitude eines gedämpften
Schwingkreises untersucht werden. Der Nadelimpulsgenerator wird dabei so
eingestellt, dass die Amplitude der Kondensatorspannung $U_\su{C}$ etwa um
den Faktor 3 bis 8 abnimmt. Die Spannung wird am Oszilloskop gegen die Zeit
aufgetragen.
Abbildung (\ref{fog:dämpfung}) zeigt das ausgegebene Bild des Oszilloskops,
\begin{figure} % Bild kann gerne auch erst in der Auswertung verwendet werden
  \centering
  \includegraphics{Bilder/Dämpfung.JPG}
  \caption{Gedämpfte Schwingung am Oszilloskop}
  \label{fig:dämpfung}
\end{figure}
an welchem die Spannungsamplituden mit der zugehörigen Zeitdifferenz $t$
abgelesen werden. Diese Messung ist relativ genau, da der dämpfende Einfluss
des Eingangswiderstands des Oszilloskops mit Hilfe des hochohmigen Tastkopfes
($R_\su{i}=10\,\si{\mega\ohm}$) vernachlässigbar klein gehalten werden.
