In der ersten Messung wurde die Skala des Voltmeters geändert, wodurch sich leichte
Fehler ergeben, da sich die Innenwiderstände dabei verändern. Zudem konnten sowohl bei
dieser, als auch bei den anderen Messungen die Werte auf der Skala nicht exakt
abgelesen werden, da sich der Zeiger häufig zwischen zwei Messstrichen befand.
Weitere Ungenauigkeiten ergeben sich aus dem Aufbau, da die Bauteile nicht idealisiert sind.
Da die mit Python berechneten Abweichungen in allen Messungen jedoch nur sehr gering ausfallen,
lässt sich somit sagen, dass die Messungen gut gelungen sind.

Auffällig ist die Abweichung im letzten Auswertungsteil, bei dem die Messdaten der
Monozelle mit der Theoriekurve der Leistung verglichen werden. Da wir einen nur sehr geringen
systematsichen Fehler berechnet haben, dürfte die Abweichung auch hier nur sehr
gering ausfallen, dies ist jedoch nicht der Fall. Die Abweichung ist deutlich zu hoch.
Dies spricht somit für einen Fehler anderer Herkunft. Wir haben versucht herauszufinden,
ob dieser Fehler bei der Messung oder bei der Auswertung aufgetaucht ist, jedoch
haben wir ihn leider nicht finden können.
