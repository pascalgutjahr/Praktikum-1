Die Leerlaufspannung $U_{\text{0}}$ bezeichnet die Spannung, die ohne
fließenden Strom $I$ an der Spannungsquelle gemessen werden kann.
Sobald ein endlicher Strom fließt, sinkt die Spannung $U_{\text{K}}$ auf einen
Wert unterhalb $U_{\text{0}}$ ab. $U_{\text{K}}$ bezeichnet hierbei die
Klemmenspannung und fällt über der Spannungsquelle ab.
Der Spannungsabfall lässt sich durch den Innenwiderstand der Spannungsquelle
erklären.
Gemäß dem Zweiten Kirchhoffschen Gesetz
\begin{equation}
  \sum_\symup{n} \symup{U}_{0_\symup{n}} = \symup{\sum_m R_m I_m}
  \label{eqn:kirch}
\end{equation}
