Die Leerlaufspannung $U_\symup{0}$ der Monozelle und der Innenwiderstand
$R_\symup{V}$ des Voltmeters betragen:
\begin{align*}
  U_\symup{0} &= 1,6 \,\symup{\si{\volt}} \\
  R_\symup{V} &\approx 10\,\symup{\si{\mega\ohm}}
\end{align*}
Tabelle \ref{tab:U_k} zeigt die aufgenommenen Werte für die Messung von
$U_\symup{k}$ nach Abbildung \ref{fig:schlt1}. Der Wert für den Widerstand wird
hierbei in Schritten von $2,5\,\symup{\si\ohm}$ hochreguliert. Der Innenwiderstand
des Amperemeters liegt bei $0,1\,\symup{\si{\mega\ohm}}$.
\begin{table}[H]
  \centering
  \begin{tabular}{c c c c}
    \toprule
    $U_{\symup{k}}/\symup{\si{\milli\volt}}$ & $I/\symup{\si{\milli\ampere}}$  &
    $U_{\symup{k}}/\symup{\si{\milli\volt}}$ & $I/\symup{\si{\milli\ampere}}$  \\
    \midrule
      70  &  74  &   990  &  18  \\
     100  &  72  &  1020  &  16  \\
     390  &  56  &  1050  &  15  \\
     510  &  48  &  1050  &  14  \\
     640  &  42  &  1080  &  12  \\
     740  &  36  &  1110  &  11  \\
     790  &  33  &  1125  &   9  \\
     860  &  29  &  1140  &   8  \\
     910  &  26  &  1170  &   7  \\
     960  &  22  &  1200  &   7  \\
    1000  &  21  &  \hrulefill  & \hrulefill  \\
    \bottomrule
  \end{tabular}
  \caption{Klemmspannung der Monozelle in Abhängigkeit von $I$}
  \label{tab:U_k}
\end{table}
Mit diesen Werten lässt sich der Graph aus Abbildung \ref{fig:mono} zeichnen.
\begin{figure}[h]
  \centering
  \includegraphics[width=0.8\textwidth]{bilder/mono.pdf}
  \caption{Messwerte der Monozelle}
  \label{fig:mono}
\end{figure}
\\
Die Werte für $a$ und $b$ lauten dann:
\begin{align*}
  -a &= R_\symup{i} = (16,3 \pm 0,3)\, \symup{\si{\ohm}} \\
   b &= U_\symup{0} = (1,30 \pm 0,01)\, \symup{\si{\volt}}
\end{align*}
\newpage
Die Messwerte für das Schaltbild aus Abbildung \ref{fig:schlt2} werden in
Tabelle \ref{tab:ggn} gezeigt.
\begin{table}[H]
  \centering
  \begin{tabular}{c c c c}
    \toprule
    $U_{\symup{k}}/\symup{\si{\volt}}$ & $I/\symup{\si{\milli\ampere}}$  &
    $U_{\symup{k}}/\symup{\si{\volt}}$ & $I/\symup{\si{\milli\ampere}}$  \\
    \midrule
    3,7  &  180  &  2,4  &  90  \\
    3,7  &  170  &  2,3  &  90  \\
    3,3  &  150  &  2,3  &  90  \\
    3,1  &  140  &  2,3  &  85  \\
    3,0  &  130  &  2,2  &  80  \\
    2,8  &  120  &  2,2  &  80  \\
    2,7  &  110  &  2,2  &  80  \\
    2,6  &  110  &  2,2  &  80  \\
    2,6  &  100  &  2,1  &  80  \\
    2,5  &  100  &  2,1  &  75  \\
    2,5  &   95  &  \hrulefill  &  \hrulefill \\
    \bottomrule
  \end{tabular}
  \caption{Messwerte mit Gegenspannung}
  \label{tab:ggn}
\end{table}
Aus diesen Werten folgt der Graph in Abbildung \ref{fig:ggn}.
\begin{figure}
  \centering
  \includegraphics[width=0.8\textwidth]{bilder/gegen.pdf}
  \caption{Messwerte der Monozolle mit Gegenspannung}
  \label{fig:ggn}
\end{figure}
\\
Die von Python berechneten Werte lauten dann:
\begin{align*}
   a &= R_\symup{i} = (15,8 \pm 0,4)\, \symup{\si{\ohm}} \\ %EInheit richtig?
   b &= U_\symup{0} = (0,93 \pm 0,04)\,\symup{\si{\volt}}
\end{align*}
Da sich durch die angelegte Gegenspannung die Stromrichtung umkehrt, ändert sich
das Vorzeichen der Steigung.



\begin{table}[H]
  \centering
  \begin{tabular}{c c c c}
    \toprule
    $U_{\symup{k}}/\symup{\si{\milli\volt}}$ & $I/\symup{\si{\milli\ampere}}$  &
    $U_{\symup{k}}/\symup{\si{\milli\volt}}$ & $I/\symup{\si{\milli\ampere}}$  \\
    \midrule
     96   &  2.61  &  177  &  1.20  \\
     96   &  2.61  &  183  &  1.14  \\
    114   &  2.31  &  186  &  1.08  \\
    129   &  2.04  &  189  &  1.02  \\
    138   &  1.89  &  192  &  0.99  \\
    147   &  1.71  &  192  &  0.93  \\
    153   &  1.59  &  195  &  0.90  \\
    159   &  1.50  &  198  &  0.87  \\
    165   &  1.41  &  198  &  0.84  \\
    168   &  1.32  &  198  &  0.84  \\
    174   &  1.26  &  \hrulefill  & \hrulefill  \\
    \bottomrule
  \end{tabular}
  \caption{Rechteckspannung}
  \label{tab:recht}
\end{table}
Die Frequenz $f$ beträgt $50\,\si{\hertz}$, die Range $1 \,\si{\volt}$,
der Innenwiderstand $100\,\si{\ohm}$ und die Amplitude steht auf $0,5 \, max$.
Mit den Messwerten aus Tabelle \ref{tab:recht} ergibt sich:
\begin{figure}[h]
  \centering
  \includegraphics[width=0.8\textwidth]{bilder/recht.pdf}
  \caption{Messwerte der Rechteckspannung}
  \label{fig:re}
\end{figure}
\\
Die von Python berechneten Werte lauten:
\begin{align*}
   -a &= R_\symup{i} = (58,1 \pm 0,5)\, \symup{\si{\ohm}} \\ %EInheit richtig?
   b &= U_\symup{0} = (0,2473 \pm 0,0008)\,\symup{\si{\volt}}
\end{align*}



\begin{table}[H]
  \centering
  \begin{tabular}{c c c c}
    \toprule
    $U_{\symup{k}}/\symup{\si{\volt}}$ & $I/\symup{\si{\milli\ampere}}$  &
    $U_{\symup{k}}/\symup{\si{\volt}}$ & $I/\symup{\si{\milli\ampere}}$  \\
    \midrule
    0.54   &  1.74  &  1.62  &  0.30  \\
    0.60   &  1.65  &  1.62  &  0.27  \\
    0.93   &  1.20  &  1.65  &  0.24  \\
    1.11   &  0.93  &  1.65  &  0.21  \\
    1.23   &  0.81  &  1.65  &  0.21  \\
    1.32   &  0.66  &  1.68  &  0.21  \\
    1.38   &  0.57  &  1.68  &  0.18  \\
    1.47   &  0.48  &  1.68  &  0.18  \\
    1.53   &  0.42  &  1.68  &  0.18  \\
    1.56   &  0.36  &  1.71  &  0.15  \\
    1.59   &  0.33  &  \hrulefill  & \hrulefill  \\
    \bottomrule
  \end{tabular}
  \caption{Sinusspannung}
  \label{tab:sin}
\end{table}
Die Frequenz $f$ beträgt $1\,\si{\kilo\hertz}$, die Range $1\,\si{\volt}$, der
Innenwiderstand $100\,\si{\ohm}$ und die Amplitude ist maximal.
Mit den Werten aus Tabelle \ref{tab:sin} folgt:
\begin{figure}[h]
  \centering
  \includegraphics[width=0.8\textwidth]{bilder/sinus.pdf}
  \caption{Messwerte der Sinusspannung}
  \label{fig:si}
\end{figure}
\\
Die von Python berechneten Werte lauten dann:
\begin{align*}
   -a &= R_\symup{i} = (740 \pm 6)\, \symup{\si{\ohm}} \\
   b &= U_\symup{0} = (1,821 \pm 0,004)\,\symup{\si{\volt}}
\end{align*}

Da der Innenwiderstand eines Voltmeters nicht unendlich groß ist, muss der systematische
Fehler bei der direkten Leerlaufspannung betrachtet werden. Dazu wird Gleichung
\eqref{eqn:U0k} umgeformt:
\begin{equation}
  U_\symup{0} = U_\symup{k} + U_\symup{k} \frac{R_\symup{i}}{R_\symup{a}}
\end{equation}
$U_\symup{k}$ ist die Leerlaufspannung der Monozelle, $R_\symup{a}$ ist gleich dem
Innenwiderstand des Voltmeters, also $R_\symup{V}$.
Für den Fehler folgt:
\begin{align*}
  \Delta U_\symup{0} &= U_\symup{0} - U_\symup{k} = 2.6 \cdot 10^{-6} \,\si{\ohm} \\
  \frac{\Delta U_\symup{0}}{U_\symup{k}} &= 1.6 \cdot 10^{-6} \, \symup{\%}
\end{align*}
Dieser Fehler ist vernachlässigbar klein.

Wird das Voltmeter hinter das Amperemeter, also an Punkt $H$, geschaltet, so wird
nicht nur noch der Spannungsabfall der Spannungsquelle gemessen, sondern zusätzlich
noch der Spannungsabfall des Amperemeters, da auch dieses einen geringen Innenwiderstand
trägt.

Im letzten Teil der Auswertung werden die Messdaten aus der ersten Messung für die Monozelle
mit der Leistungskurve überprüft. Dafür wird $U_\text{k} \cdot I$ gegen $R_\text{a} = \frac{U_\text{k}}{I}$
geplottet. Die Theoriekurve lautet:
\begin{equation}
  N = I^2 R_\text{a} = \frac{U_\text{0}^2}{(R_\text{i}+ R_\text{a})^2} R_\text{a}
\end{equation}

\begin{figure}[!h]
  \centering
  \includegraphics[width=0.8\textwidth]{bilder/leistung.pdf}
  \caption{Vergleich Leistungskurve mit Messwerten der Monozelle}
  \label{fig:leistung}
\end{figure}
