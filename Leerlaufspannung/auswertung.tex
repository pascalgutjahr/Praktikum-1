Die Leerlaufspannung $U_\symup{0}$ der Monozelle und der Innenwiderstand
$R_\symup{V}$ des Voltmeters lauten:
\begin{align*}
  U_\symup{0} &= 1,6 \,\symup{\si{\volt}} \\
  R_\symup{V} &\approx 10\,\symup{\si{\mega\ohm}}
\end{align*}
Tabelle \ref{tab:U_k} zeigt die aufgenommenen Werte für die Messung von
$U_\symup{k}$
\begin{table}[H]
  \centering
  \begin{tabular}{c c c c}
    \toprule
    $U_\symup{k/\si{\milli\volt}}$ & $I/\symup{\si{\milli\ampere}}$  &
    $U_\symup{k/\si{\milli\volt}}$ & $I/\symup{\si{\milli\ampere}}$  \\
    \midrule
      70  &  74  &   990  &  18  \\
     100  &  72  &  1020  &  16  \\
     390  &  56  &  1050  &  15  \\
     510  &  48  &  1050  &  14  \\
     640  &  42  &  1080  &  12  \\
     740  &  36  &  1110  &  11  \\
     790  &  33  &  1125  &   9  \\
     860  &  29  &  1140  &   8  \\
     910  &  26  &  1170  &   7  \\
     960  &  22  &  1200  &   7  \\
    1000  &  21  &  \hrulefill  & \hrulefill  \\
    \bottomrule
  \end{tabular}
  \caption{Klemmspannung der Monozelle in Abhängigkeit von $I$}
  \label{tab:U_k}
\end{table}
mit diesen Werten lässt sich der Plot \ref{fig:mono} zeichnen
\begin{figure}[h]
  \centering
  \includegraphics{bilder/mono.pdf}
  \caption{Messwerte der Monozelle}
  \label{fig:mono}
\end{figure}
