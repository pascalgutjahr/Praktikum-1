Die Leerlaufspannung $U_\symup{0}$ der Monozelle und der Innenwiderstand
$R_\symup{V}$ des Voltmeters betragen:
\begin{align*}
  U_\symup{0} &= 1,6 \,\symup{\si{\volt}} \\
  R_\symup{V} &\approx 10\,\symup{\si{\mega\ohm}}
\end{align*}
Tabelle \ref{tab:U_k} zeigt die aufgenommenen Werte für die Messung von
$U_\symup{k}$ nach Abbildung \ref{fig:schlt1}. Der Wert für den Widerstand wird
hierbei in Schritten von $2,5\symup{\si\ohm}$ hochreguliert. Der Innenwiderstand
des Amperemeters liegt bei $0,1 \symup{\si{\mega\ohm}}$.
\begin{table}[H]
  \centering
  \begin{tabular}{c c c c}
    \toprule
    $U_\symup{k/\si{\milli\volt}}$ & $I/\symup{\si{\milli\ampere}}$  &
    $U_\symup{k/\si{\milli\volt}}$ & $I/\symup{\si{\milli\ampere}}$  \\
    \midrule
      70  &  74  &   990  &  18  \\
     100  &  72  &  1020  &  16  \\
     390  &  56  &  1050  &  15  \\
     510  &  48  &  1050  &  14  \\
     640  &  42  &  1080  &  12  \\
     740  &  36  &  1110  &  11  \\
     790  &  33  &  1125  &   9  \\
     860  &  29  &  1140  &   8  \\
     910  &  26  &  1170  &   7  \\
     960  &  22  &  1200  &   7  \\
    1000  &  21  &  \hrulefill  & \hrulefill  \\
    \bottomrule
  \end{tabular}
  \caption{Klemmspannung der Monozelle in Abhängigkeit von $I$}
  \label{tab:U_k}
\end{table}
Mit diesen Werten lässt sich der Graph aus Abbildung \ref{fig:mono} zeichnen.
\begin{figure}[h]
  \centering
  \includegraphics[width=0.8\textwidth]{bilder/mono.pdf}
  \caption{Messwerte der Monozelle}
  \label{fig:mono}
\end{figure}
\\
Die Werte für $a$ und $b$ lauten dann:
\begin{align*}
  -a &= R_\symup{i} = (16,3 \pm 0,3) \symup{\si{\ohm}} \\ %milli richtig?
   b &= U_\symup{0} = (1,30 \pm 0,01) \symup{\si{\milli\volt}}
\end{align*}
\newpage
Die Messwerte für das Schaltbild aus Abbildung \ref{fig:schlt2} werden in
Tabelle \ref{tab:ggn} gezeigt.
\begin{table}[H]
  \centering
  \begin{tabular}{c c c c}
    \toprule
    $U_\symup{k/\si{\volt}}$ & $I/\symup{\si{\milli\ampere}}$  &
    $U_\symup{k/\si{\volt}}$ & $I/\symup{\si{\milli\ampere}}$  \\
    \midrule
    3,7  &  180  &  2,4  &  90  \\
    3,7  &  170  &  2,3  &  90  \\
    3,3  &  150  &  2,3  &  90  \\
    3,1  &  140  &  2,3  &  85  \\
    3,0  &  130  &  2,2  &  80  \\
    2,8  &  120  &  2,2  &  80  \\
    2,7  &  110  &  2,2  &  80  \\
    2,6  &  110  &  2,2  &  80  \\
    2,6  &  100  &  2,1  &  80  \\
    2,5  &  100  &  2,1  &  75  \\
    2,5  &   95  &  \hrulefill  &  \hrulefill \\
    \bottomrule
  \end{tabular}
  \caption{Messwerte mit Gegenspannung}
  \label{tab:ggn}
\end{table}
Aus diesen Werten folgt der Graph in Abbildung \ref{fig:ggn}.
\begin{figure}
  \centering
  \includegraphics[width=0.8\textwidth]{bilder/gegen.pdf}
  \caption{Messwerte der Monozolle mit Gegenspannung}
  \label{fig:ggn}
\end{figure}
\\
Die von Python berechneten Werte lauten dann:
\begin{align*}
   a &= R_\symup{i} = (15,8 \pm 0,4) \symup{\si{\ohm}} \\ %EInheit richtig?
   b &= U_\symup{0} = (0,93 \pm 0,04) \symup{\si{\milli\volt}}
\end{align*}
Da sich durch die angelegte Gegenspannung die Stromrichtung umkehrt ändert sich
das Vorzeichen der Steigung.
