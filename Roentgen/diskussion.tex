


Abweichung der Maximalenergie von 35 kev beträgt 29.7 \% zu 24.61 kev

Abweichungen der einzelnen Abschirmkonstanten $\su{\sigma_k}$
\begin{table}
  \centering
  \begin{tabular}{c c c c }
    \toprule
    $\su{Element}$ & $\su{Literaturwert}$ & $\su{gemessener Wert}$ & $\su{relative Abweichung}$ \\
    \midrule
    Zink       & 3.56 & 4.39 & 23.3\,\% \\
    Germanium  & 3.66 & 3.56 &  3.0\,\% \\
    Brom       & 3.85 & 3.74 &  3.0\,\% \\
    Strontium  & 4.00 & 3.69 &  7.8\,\% \\
    Zirkonium  & 5.04 & 5.70 & 13.1\,\% \\
    Gold
    \bottomrule
  \end{tabular}
  \caption{relative Abweichungen der Abschirmkonstanten}
  \label{tab:abw}
\end{table}
Ein Grund für die teilweise sehr hohen Abweichungen ist der Zustand der einzelnen
Absorber, da die Elementschicht teilweise beschädigt ist. Desweiteren lässt sich
der Winkel nicht exakt ablesen, was zu weiteren Fehlern führt. Auch ist nicht
sicher, ob die Absorber richtig auf dem Zählrohr befestigt wurden.
