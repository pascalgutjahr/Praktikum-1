In Tabelle \ref{tab:info} sind die Literaturwerte zu den hier verwendeten
Elementen einzusehen. Die Glanzwinkel $\Theta$ berechnen sich nach Formel
\eqref{eqn:bragg} mit $E=h\cdot c / \lambda$.
\begin{table}
  \centering
  \begin{tabular}{c c c c c}
    \toprule
    $\su{Metall}$  & $\su{Ordnungszahl} \,\, \su{Z}$ &
    $E^\su{Lit}_\su{K} \,/\, \keV$ & $\theta^\su{Lit}_\su{K}\,/\,\si{\degree}$ & $\sigma^\su{K}$ \\
    \midrule
     Zn & 30 & 9.65 & 18.6 &  \\
     Ge & 32 & 11.11 & 16.1 &  \\
     Br & 35 & 13.47 & 13.2 &  \\
     Sr & 38 & 16.10 & 11.0 &  \\
     Zr & 40 & 17.09 & 9.0 &  \\
     Au & 79 & 13.70 & 13.0 (LII) &  \\
     Au & 79 & 11.92 & 15.0 (LIII) &  \\
    \bottomrule
  \end{tabular}
  \caption{Literaturwerte der Metalle \cite{Elit},\,\cite{KAu}}
  \label{tab:info}
\end{table}
\begin{figure}[!h]
  \centering
  \includegraphics[width=0.5\textwidth, angle=270]{bilder/Bragg3.pdf}
  \caption{Überprüfung der Bragg-Bedingung}
  \label{fig:Bragg3}
\end{figure}
Zu Beginn wird die Bragg-Bedingung überprüft. Der Graph in Abbildung \ref{fig:Bragg3}
lässt einen Glanzwinkel von ca. $2\Theta = 26.8 \,\si{\degree}$ erkennen. Dies stimmt mit
dem eingestellten Kristallwinkel von $\Theta = 14\,\si{\degree}$ ungefähr überein,
wodurch die Bragg-Bedingung bestätigt wird.

Anschließend wird das Emissionssprektrum der Cu-Röntgenröhre untersucht.
Dieses Spektrum ist in Abb. \ref{fig:EmissionCu} zu sehen.
\begin{figure}
  \centering
  \includegraphics[width=0.5\textwidth, angle=270]{bilder/EmissionCu.pdf}
  \caption{Emissionspektrum der Cu-Röntgenröhre}
  \label{fig:EmissionCu}
\end{figure}
Der vordere Peak steht für die $K_\su{\alpha}$ und der hintere für die $K_\su{\beta}$
Linie. Die Halbwertsbreite, bei der die Intensität auf die Hälfte abgesunken ist,
der $\alpha$-Linie läuft von $\Theta = 19.5\,\si{\degree}$ bis $\Theta = 20.1\deg$.
Die der $\beta$-Linie läuft von $\Theta = 21.9\deg$ bis $\Theta = 22.3\deg$.
Die Energien ergeben sich damit zu
\begin{align}
  E_\su{\Theta_1, \alpha} &= 10.23 \keV \\
  E_\su{\Theta_2, \alpha} &= 9.93 \keV \\
  E_\su{\Theta_1, \alpha} &= 9.14 \keV \\
  E_\su{\Theta_2, \alpha} &= 8.00 \keV. \\
\end{align}
Das Auflösungsvermögen berechnet sich durch die Differenz der jeweiligen Energien,
also
\begin{equation}
  \Delta E = E_2 - E_1.
\end{equation}
Damit folgt
\begin{align}
  \Delta E_\su{\alpha} &= 0.30 \keV  \\
  \Delta E_\su{\beta} &= 0.15 \keV. \\
\end{align}





\begin{figure}
  \centering
  \includegraphics[width=0.5\textwidth, angle=270]{bilder/AbsorpZn.pdf}
  \caption{Absorptionsspektrum von Zink}
  \label{fig:Zink}
\end{figure}

\begin{figure}
  \centering
  \includegraphics[width=0.5\textwidth, angle=270]{bilder/AbsorpGe.pdf}
  \caption{Absorptionsspektrum von Germanium}
  \label{fig:Germanium}
\end{figure}

\begin{figure}
  \centering
  \includegraphics[width=0.5\textwidth, angle=270]{bilder/AbsorpBr.pdf}
  \caption{Absorptionsspektrum von Brom}
  \label{fig:Brom}
\end{figure}

\begin{figure}
  \centering
  \includegraphics[width=0.5\textwidth, angle=270]{bilder/AbsorpSr.pdf}
  \caption{Absorptionsspektrum von Strontium}
  \label{fig:Strontium}
\end{figure}

\begin{figure}
  \centering
  \includegraphics[width=0.5\textwidth, angle=270]{bilder/AbsorpZr.pdf}
  \caption{Absorptionsspektrum von Zirkonium}
  \label{fig:Zirkonium}
\end{figure}

\begin{figure}
  \centering
  \includegraphics[width=0.5\textwidth, angle=270]{bilder/AbsorpAu.pdf}
  \caption{Absorptionsspektrum von Aurum}
  \label{fig:Aurum}
\end{figure}
