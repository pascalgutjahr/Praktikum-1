Röntgenstrahlung entsteht, wenn Elektronen aus einer Glühkathode emittieren und
auf eine Anode beschleunigt werden. Bei dem Auftreffen auf die Anode entsteht
Röntgenstrahlung, welche sich aus einem kontinuierlichen Bremsspektrum und
charakteristischen Röntgenstrahlung zusammensetzt.
\subsection{Bremsspektrum}
Wenn ein Elektron im Coulombfeld abgebremst wird, wird ein Photon entsendet,
dessen Energie gleich dem Energieverlust des Elektrons ist.
Da das Elektron sowohl seine gesamte kinetische Energie $E_\su{kin}$, als auch
nur einen Teil abgeben kann, wird das Bremsspektrum auch als kontinuierliches
Spektrum bezeichnet.
Die maximale Energie beziehungsweise minimale Wellenlänge lässt sich mittels
Formel \eqref{Emax} berechnen.
\begin{equation}
  \lambda_\su{min} = \frac{h\cdot c}{e_0U}
  \label{Emax}
\end{equation}
Die minimale Wellenlänge ergibt sich bei vollständiger Abbremsung des Elektrons.
Hierbei wird die gesamte kinetische Energie in Strahlungsenergie $E=h\cdot\nu$
umgewandelt.
\subsection{Charakteristisches Spektrum}
Beim charakteristischen Spektrum wird das Anodenmaterial ionisiert, sodass
eine Leerstelle in der inneren Schale entsteht. Ein Elektron aus einer äußeren
Schale kann nun  unter Aussendung eines Röntgenquants in eine der inneren
Schalen zurückfallen. Die Energie des Quants entspricht dabei der
Energiedifferenz der beiden Energieniveaus.
Die Energiedifferenz wird mit Formel \eqref{Equant} berechnet.
\begin{equation}
  h\nu = E_\su{m}-E_\su{n}
  \label{Equant}
\end{equation}
In diesem Fall besteht das Spektrum aus scharfen Linien, welche mit $K_\su{\alpha}
$, $K_\su{\beta}$ etc. bezeichnet werden. Die Buchstaben stehen hierbei für die
Schale, auf der die Übergänge enden, während die griechischen Buchstaben die
Herkunft des Elektrons bezeichnen.

In einem Mehrelektronenatom wird die Kernladung von den Hüllenelektronen und
der Wechselwirkung der Elektronen abgeschirmt was dazu führt, dass die
Coulomb-Anziehung auf das Elektron verringert wird. Die Bindungsenergie eines
Elektrons auf der n-ten Schale berechnet sich dann mit
\begin{equation}
  E_\su{n}=-R_\su{\infty}z_\su{eff}^2\cdot \frac{1}{n^2}.
  \label{En}
\end{equation}
Der Abschirmeffekt wird durch die effektive Kernladung
\begin{equation*}
  z_\su{eff}=z-\sigma
\end{equation*}
berücksichichtigt. Die Abschirmkonstante $\sigma$ ist für hedes Elektron
verschieden, aber empirisch bestimmbar.
Unter Anwendung von Formel \eqref{En} lässt sich die Energie der $K_\su{\alpha}$-
Linie mit
\begin{equation}
  E_\su{K_\su{\alpha}}=R_\su{\infty}(z-\sigma_1)^2\cdot\frac{1}{1^2}
  -R_\su{\infty}(z-\sigma_2)^2\cdot\frac{1}{2^2}
  \label{EKAlpha}
\end{equation}
berechnen. Jede charakteristische Linie wird in eine Reihe von eng beieinander
liegenden Linien aufgelöst, da die äußeren Elektronen unterschiedliche
Bindungsenergien besitzen.
\subsection{Absorption von Röntgenstrahlung}
Bei Röntgenstrahlung unter $1 \,\si{\mega\electronvolt}$ spielen der Comptoneffekt
und der Photoeffekt eine dominante Rolle.
Mit zunehmender Energie nimmt der Absobtionskoeffizient ab und nimmt
sprunghaft zu, wenn die Photonenenergie gerade größer ist als die Bindungsenergie
eines Elektrons aus der nächsten inneren Schale.
Der Verlauf der Absorption ist beispielhaft in Abbildung \ref{fig:absorb} zu sehen.
\begin{figure}
  \centering
  \includegraphics[width=0.8\textwidth]{bilder/absorb.pdf}
  \caption{Verlauf der Absorption\cite{V602}}
  \label{fig:absorb}
\end{figure}
Die Lage der Absorptionskanten
\begin{equation*}
  h\nu_\su{abs} = E_\su{n} - E_\su{\infty}
\end{equation*}
ist nahezu identisch mit der Bindungsenergie eines Elektrons. Die zugehörigen
Energien werden je nach der Schale aus der das Elektron stammt als jeweilige
Absorptionskante bezeichnet. %verständlich?
Wegen der Feinstruktur lassen sich 3 L-Kanten, aber nur eine K-Kante beobachten.
Die Bindungsenergie unter berücksichtigung der Feinstruktur wird mit der
Sommerfeldschen Feinstrukturformel
\begin{equation}
  EW_\su{n, j}=R_\su{\infty}\left(z_\su{eff,1}^2\cdot\frac{1}{n^2}
  +\alpha^2 z_\su{eff,2}^4\cdot\frac{1}{n^3}\left(\frac{1}{j+\frac{1}{2}}-
  \frac{3}{4n}\right)\right)
  \label{sommerfeld}
\end{equation}
berechnen.
