Röntgenstrahlung entsteht, wenn Elektronen aus einer Glühkathode emittieren und
auf eine Anode beschleunigt werden. Bei dem Auftreffen auf die Anode entsteht
Röntgenstrahlung, welche sich aus einem kontinuierlichen Bremsspektrum und
charakteristischen Röntgenstrahlung zusammensetzt.
\subsection{Bremsspektrum}
Wenn ein Elektron im Coulombfeld abgebremst wird, wird ein Photon entsendet,
dessen Energie gleich dem Energieverlust des Elektrons ist.
Da das Elektron sowohl seine gesamte kinetische Energie $E_\su{kin}$, als auch
nur einen Teil abgeben kann, wird das Bremsspektrum auch als kontinuierliches
Spektrum bezeichnet.
Die maximale Energie beziehungsweise minimale Wellenlänge lässt sich mittels
Formel \eqref{Emax} berechnen.
\begin{equation}
  \lambda_\su{min} = \frac{h\cdot c}{e_0U}
  \label{Emax}
\end{equation}
Die minimale Wellenlänge ergibt sich bei vollständiger Abbremsung des Elektrons.
Hierbei wird die gesamte kinetische Energie in Strahlungsenergie $E=h\cdot\nu$
umgewandelt.
\subsection{Charakteristisches Spektrum}
Beim charakteristischen Spektrum wird das Anodenmaterial ionisiert, sodass
eine Leerstelle in der inneren Schale entsteht. Ein Elektron aus einer äußeren
Schale kann nun  unter Aussendung eines Röntgenquants in eine der inneren
Schalen zurückfallen.  
