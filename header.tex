\documentclass
[
bibliography=totoc, %für Literaturverzeichnis
captions=tableheading
]
{scrartcl}

\usepackage{scrhack} %??

\usepackage[aux]{rerunfilecheck}
\usepackage{polyglossia}
\setmainlanguage{german}
\usepackage{amsmath} %unverzichtbare Mathe-Befehle
\usepackage{amssymb} %viele Mathe-Symbole
\usepackage{fontspec}
\usepackage[style=alphabetic]{biblatex}
%\addbibresource{name der Datei mit .bib} %für Literaturverzeichnis
\usepackage{mathtools} %Erweiterungen für amsmath
\usepackage{unicode-math}
%Noch mehr Pakete
\setmathfont{Latin Modern Math}
\usepackage[math-style=ISO, bold-style=ISO, sans-style=italic, nabla=upright, partial=upright]{unicode-math}
%sobald man die Pakete untereinander schreibt, so verfärbt sich der Befehl blau...
\usepackage{float}
\floatplacement{figure}{htbp}
\floatplacement{table}{htbp}
\usepackage[section, below]{placeins}


\usepackage[unicode]{hyperref}
\usepackage{bookmark}
%Einstellungen hier, z.b. Fonts

\newcommand{\be}{\begin{equation}} %Kurzbefehl für \begin{equation}
\newcommand{\ee}{\end{equation}} %Kurzbefehl für \end{equation}
%Befehle eingefügt um Zeichen zu sparen
