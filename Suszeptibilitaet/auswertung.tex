In Abb. \ref{fig:Durchlass} wird $\sfrac{U_\su{A}}{U_\su{E}}$ gegen die Frequenz
$\nu$ geplottet. $U_\su{A}$ ist die gemessene Ausgangsspannung, $U_\su{E} = 0.41\,\si{\volt}$
ist die Eingangsspannung.
\begin{figure}
  \centering
  \includegraphics[width=0.8\textwidth]{bilder/frequenz.pdf}
  \caption{Durchlassfrequenz des Selektivvertärkers}
  \label{fig:Durchlass}
\end{figure}
Die ideale Durchlassfrequenz liegt bei
\begin{equation}
  \nu_0 = 35.10 \kHz.
\end{equation}
Die untere Grenzfrequenz $\nu_-$ und die obere Grenzfrequenz $\nu_+$ liegen bei:
\begin{align}
  \nu_- = 34.85 \kHz \\
  \nu_+ = 35.35 \kHz.
\end{align}









\begin{table}
  \centering
  \begin{tabular}{c c c}
    \toprule
    $\su{Probe}$ & $\su{R_{ohne}/\mOhm}$ &$\su{R_{Probe}}/\mOhm$ \\
    \midrule
    $\su{Nd_2O_3}$   & 420 & 435 \\
                     & 485 & 510 \\
                     & 615 & 670 \\
                     & 690 & 720 \\
                     & 775 & 875 \\ \hline
    $\su{Gd_2O_3}$   & 810 & 820 \\
                     & 790 & 770 \\
                     & 730 & 860 \\
                     & 585 & 920 \\
                     &1030 &1135 \\ \hline
    $\su{C_6O_{12}Pr_2}$  & 205 & 150 \\
                          & 295 & 325 \\
                          & 330 & 350 \\
                          & 315 & 250 \\
                          & 375 & 370 \\ \hline
    $\su{Dy_2O_3}$   & 370 & 500 \\
                     & 520 & 510 \\
                     & 305 & 200 \\
                     & 490 & 420 \\
                     & 450 & 400 \\
    \bottomrule
  \end{tabular}
  \caption{Widerstandsmessungen}
  \label{tab:mess2}
\end{table}
Mit den obigen Werten lässt sich die mittlere Differenz des Widerstands
$\Delta R$ bestimmen, mit denen hinterher die Suszeptibilität berechnet wird.
\begin{align*}
  \Delta R_\su{Nd3O3}    &= ( 45 \pm  31) \mOhm \\
  \Delta R_\su{Gd2O3}    &= (120 \pm 117) \mOhm \\
  \Delta R_\su{C6O12Pr2} &= ( 35 \pm  22) \mOhm \\
  \Delta R_\su{Dy2O3}    &= ( 73 \pm  42) \mOhm
\end{align*}
Nach Formel \eqref{eqn:suszep} lassen sich die folgenden Werte für die Suszeptibilität
berechnen:
\begin{align*}
  \chi_\su{Nd_2O_3}       &= 1.68 \pm 1.13 \\
  \chi_\su{Gd_2O_3}       &= 2.22 \pm 2.16 \\
  \chi_\su{C_6O_{12}Pr_2} &= 0.25 \pm 0.16 \\
  \chi_\su{Dy_2O_3}       &= 0.79 \pm 0.94 %Gemessen mit Glas
\end{align*}
