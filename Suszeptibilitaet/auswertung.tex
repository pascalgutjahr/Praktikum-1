\subsection{Bestimmung der Durchlassfrequenz}
In Abb. \ref{fig:Durchlass} wird $\sfrac{U_\su{A}}{U_\su{E}}$ gegen die Frequenz
$\nu$ geplottet. $U_\su{A}$ ist die gemessene Ausgangsspannung, $U_\su{E} = 0.41\,\si{\volt}$
ist die Eingangsspannung.
Die ideale Durchlassfrequenz liegt bei
\begin{equation}
  \nu_0 = 35.10 \kHz.
\end{equation}
Die untere Grenzfrequenz $\nu_-$ und die obere Grenzfrequenz $\nu_+$ liegen bei:
\begin{align}
  \nu_- = 34.85 \kHz \\
  \nu_+ = 35.35 \kHz.
\end{align}
\begin{figure}
  \centering
  \includegraphics[width=0.8\textwidth]{bilder/frequenz.pdf}
  \caption{Durchlassfrequenz des Selektivvertärkers}
  \label{fig:Durchlass}
\end{figure}
Die Werte sind in Tablle \ref{tab:werte} zu finden.
\newpage
\begin{table}
  \centering
  \begin{tabular}{cc|cc}
    \toprule
    \multicolumn{1}{c}{Frequenz} & \multicolumn{1}{c|}{Ausgangsspannung} &
    \multicolumn{1}{c}{Frequenz} & \multicolumn{1}{c}{Ausgangsspannung} \\
    {$\nu\,/\kHz$} & {$U_\su{A}\,/\mV$}& {$\nu\,/\kHz$} & {$U_\su{A}\,/\mV$} \\
    \midrule
       33.0 &   45.0 & 34.6  &   165.0 \\
       33.1 &   46.0 & 34.7  &   234.0 \\
       33.2 &   49.0 & 34.8  &   234.0 \\
       33.3 &   53.0 & 34.9  &   320.0 \\
       33.4 &   54.0 & 35.0  &   405.0 \\
       33.5 &   58.0 & 35.1  &   475.0 \\
       33.6 &   62.0 & 35.2  &   415.0 \\
       33.7 &   66.0 & 35.3  &   370.0 \\
       33.8 &   73.0 & 35.4  &   245.0 \\
       33.9 &   79.0 & 35.5  &   235.0 \\
       34.0 &   84.0 & 35.6  &   180.0 \\
       34.1 &   92.0 & 35.7  &   138.0 \\
       34.2 &   96.0 & 35.8  &   132.0 \\
       34.3 &  117.0 & 35.9  &   114.0 \\
       34.4 &  120.0 & 36.0  &   105.0 \\
       34.5 &  141.0 & 38.0  &    33.0 \\
  \bottomrule
  \end{tabular}
  \caption{Messwerte}
  \label{tab:werte}
\end{table}
\newpage
\subsection{Experimentelle Bestimmung der paramagnetischne Suzeptibilität}
\begin{table}
  \centering
  \begin{tabular}{c c c}
    \toprule
    Probe & $\su{R_{ohne}/\mOhm}$ &$\su{R_{Probe}}/\mOhm$ \\
    \midrule
    $\su{Nd_2O_3}$   & 420 & 435 \\
                     & 485 & 510 \\
                     & 615 & 670 \\
                     & 690 & 720 \\
                     & 775 & 875 \\ \hline
    $\su{Gd_2O_3}$   & 810 & 820 \\
                     & 790 & 770 \\
                     & 730 & 860 \\
                     & 585 & 920 \\
                     &1030 &1135 \\ \hline
    $\su{C_6O_{12}Pr_2}$  & 205 & 150 \\
                          & 295 & 325 \\
                          & 330 & 350 \\
                          & 315 & 250 \\
                          & 375 & 370 \\ \hline
    $\su{Dy_2O_3}$   & 370 & 500 \\
                     & 520 & 510 \\
                     & 305 & 200 \\
                     & 490 & 420 \\
                     & 450 & 400 \\
    \bottomrule
  \end{tabular}
  \caption{Widerstandsmessungen}
  \label{tab:mess2}
\end{table}
Mit den obigen Werten lässt sich die mittlere Differenz des Widerstands
$\Delta R$ bestimmen, mit denen hinterher die Suszeptibilität berechnet wird.
\begin{align*}
  \Delta R_\su{Nd_3O_3}    &= ( 45 \pm  31) \mOhm \\
  \Delta R_\su{Gd_2O_3}    &= (120 \pm 117) \mOhm \\
  \Delta R_\su{C_6O_{12}Pr_2} &= ( 35 \pm  22) \mOhm \\
  \Delta R_\su{Dy_2O_3}    &= ( 73 \pm  42) \mOhm
\end{align*}
In Tabelle \ref{tab:Daten} sind die allgemeinen Daten zu den verwendeten Substanzen
zu finden. Die Dichte für $C_6O_{12}Pr_2$ wurde dabei seperat berechnet. Dabei wurde
der Behälter als zylinderförmig mit dem Radius $r=5\mm$ angenommen. Die anderen Dichten
stammen aus der Versuchsanleitung. $L$ beschreibt die Länge der Probe und $M$ die
Masse der Probe.
\begin{table}
  \centering
  \begin{tabular}{c c c c}
    \toprule
    $\text{Substanz}$ & $\su{L}\,/\,\si{\centi\meter}$ & $M\,/\gr$ & $\rho \,/\Dichte$ \\
    \midrule
    Gd_2O_3 & 16 & 14.08 & 7.400 \\
    Nd_2O_3 & 16 & 9.00  & 7.240 \\
    Dy_2O_3 & 16 & 23.80 & 7.800 \\
    C_6O_{12}Pr_2 & 16 & 7.87 & 0.626 \\
    \bottomrule
  \end{tabular}
\end{table}

Nach Formel \eqref{eqn:suszep} lassen sich die folgenden Werte für die Suszeptibilität
berechnen:
\begin{align*}
  \chi_\su{Nd_2O_3}       &= 1.68 \pm 1.13 \\
  \chi_\su{Gd_2O_3}       &= 2.22 \pm 2.16 \\
  \chi_\su{C_6O_{12}Pr_2} &= 0.25 \pm 0.16 \\
  \chi_\su{Dy_2O_3}       &= 0.79 \pm 0.94 %Gemessen mit Glas
\end{align*}
Der Fehler der Suszeptibilität wird mit der Gaußschen Fehlerfortpflanzung
\begin{equation*}
  \Delta\chi = \sqrt{\left(\frac{2FL\rho}{R_3M}\right)^2\cdot\sigma_\su{\Delta R}^2
  + \left(\frac{2\Delta R FL\rho}{R_3^2M}\right)^2\cdot\sigma_\su{R_3}}
\end{equation*}
berechnet.

\subsection{Theoretische Bestimmung der paramagnetischen Suszeptibilität}
Zur theoretischen Bestimmung der Suszeptibilität wird Formel \eqref{eqn:chitheo}
verwendet. Die Temperatur wird dabei als Raumtemperatur $T = 293.15 \,\si{\kelvin}$
angenommen. Der Landé-Faktor wird mit Formel \eqref{eqn:lande} berechnet und ist
in Tabelle \ref{tab:zustände} zu finden. Das $N$ aus Formel \eqref{eqn:chitheo}
wird mit Formel \eqref{eqn:momente} bestimmt. Die nötigen Werte befinden sich in
Tabelle \ref{tab:Daten}.
\begin{table}
  \centering
  \begin{tabular}{c c c c c c}
    \toprule
    Substanz & $L$ & $S$ & $J$ & $g_\su{J}$ & $\chi_\su{theo} \cdot 10^{-4} $ \\
    \midrule
    \ce{Nd2O3}  & 6  & 1.5 & 4.5  & 0.73 & 15.30 \\
    \ce{Gd2O3}   & 0  & 3.5  & -3.5  & 2.4 & 99.30 \\
    \ce{C6O12Pr2} & 6 & 1.5 & 4.5 & 0.73 & 8.81 \\
    \ce{Dy2O3} & 5 & 2.5 & 2.5  & 0.29 & 0.83 \\
    \bottomrule
  \end{tabular}
  \caption{Zustände der Substanzen}
  \label{tab:zustände}
\end{table}
