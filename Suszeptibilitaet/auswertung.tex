\begin{table}
  \centering
  \begin{tabular}{c c c}
    \toprule
    $\su{Probe}$ & $\su{R_{ohne}/\mOhm}$ &$\su{R_{Probe}}/\mOhm$ \\
    \midrule
    $\su{Nd2O3}$   & 420 & 435 \\
                   & 485 & 510 \\
                   & 615 & 670 \\
                   & 690 & 720 \\
                   & 775 & 875 \\ \hline
    $\su{Gd2O3}$   & 810 & 820 \\
                   & 790 & 770 \\
                   & 730 & 860 \\
                   & 585 & 920 \\
                   &1030 &1135 \\ \hline
    $\su{C6O12Pr2}$& 205 & 150 \\
                   & 295 & 325 \\
                   & 330 & 350 \\
                   & 315 & 250 \\
                   & 375 & 370 \\ \hline
    $\su{Dy2O3}$   & 370 & 500 \\
                   & 520 & 510 \\
                   & 305 & 200 \\
                   & 490 & 420 \\
                   & 450 & 400 \\
    \bottomrule
  \end{tabular}
  \caption{Widerstandsmessungen}
  \label{tab:mess2}
\end{table}
Mit den obigen Werten lässt sich so die mittlere Differenz des Widerstands
$\Delta R$ bestimmen, mit denen hinterher die Suszeptibilität berechnet wird.
\begin{align*}
  \Delta R_{Nd3O3}    &= ( 45 \pm  30.5) \mOhm \\
  \Delta R_{Gd2O3}    &= (120 \pm 117.2) \mOhm \\
  \Delta R_{C6O12Pr2} &= ( 35 \pm  22.1) \mOhm \\
  \Delta R_{Dy2O3}    &= ( 73 \pm  41.9) \mOhm
\end{align*}
