Tabelle \ref{tab:rel} zeigt die Abweichungen der theoretischen Suszeptibilität und
der tatsächlich bestimmten. Es fällt auf, dass diese stark voneinander
abweichen.
\begin{table}
  \centering
  \begin{tabular}{c c c c}
    \toprule
     Probe & theoretischer Wert & experimenteller Wert & relative Abweichung \\
     \midrule
     $\su{Nd_2O_3}$       &   & 1.68 \\
     $\su{Gd_2O_3}$       &   & 2.22 \\
     $\su{C_6o_{12}Pr_2}$ &   & 0.25 \\
     $\su{Dy_2O_3}$       &   & 0.79 \\
     \bottomrule
  \end{tabular}
  \caption{Relative Abweichung der Suszeptibilität}
  \label{tab:rel}
\end{table}
Die großen Abweichungen lassen sich durch den Aufbau der Apparatur erklären, da
beim messen der Widerstände die Ausgangsspannung nicht bei gleichem Widerstand
erreicht wurde. Auch ein höherer Widerstand führt somit zur Ausgangsspannung
von $0.08\mV$.
Für die Probe $\su{Dy_2O_3}$ war zudem keine Masse angegeben. Da unklar ist,
wie groß die Masse vom Glas ist, wurde die Masse höher angenommen, als sie
tatsächlich ist.
Die Dichte von $\su{C_6O_{12}Pr_2}$ wurde selbstständig berechnet und unterliegt
somit auch einem systematischen Fehler.
