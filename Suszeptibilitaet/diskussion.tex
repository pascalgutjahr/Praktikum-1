Tabelle \ref{tab:rel} zeigt die Abweichungen der theoretischen Suszeptibilität und
der tatsächlich bestimmten. Es fällt auf, dass diese stark voneinander
abweichen.
\begin{table}
  \centering
  \begin{tabular}{c c c c}
    \toprule
     Probe & theoretischer Wert & experimenteller Wert & relative Abweichung \\
     \midrule
     $\su{Nd_2O_3}$       &  $ 15.30 \cdot 10^{-4}$ & 1.68 & $109 \cdot 10^3$\\
     $\su{Gd_2O_3}$       &  $ 99.30 \cdot 10^{-4}$ & 2.22 & $ 22 \cdot 10^3$\\
     $\su{C_6O_{12}Pr_2}$ &  $ 8.81 \cdot 10^{-4}$ & 0.25  & $ 28 \cdot 10^3$\\
     $\su{Dy_2O_3}$       &  $ 0.83 \cdot 10^{-4}$ & 0.79  & $952 \cdot 10^3$\\
     \bottomrule
  \end{tabular}
  \caption{Relative Abweichung der Suszeptibilität}
  \label{tab:rel}
\end{table}
Die großen Abweichungen lassen sich durch den Aufbau der Apparatur erklären, da
beim messen der Widerstände die Ausgangsspannung nicht bei gleichem Widerstand
erneut erreicht wurde. Auch ein höherer Widerstand führt somit zur Ausgangsspannung
von $U_\su{A}= 0.08\mV$. Es war nahezu völlig egal, welchen Widerstand man eingestellt hat,
die Ausgangsspannung hat sich komplett willkürlich verhalten und landete am Ende
immer exakt bei dem Wert $U_\su{A} = 0.08\mV$.

Für die Probe $\su{Dy_2O_3}$ war zudem keine Masse angegeben. Wir haben die gesamte
Masse der Probe inklusive des Glases gemessen. Da unklar ist,
wie groß die Masse des Glases ist, haben wir sie einfach vernachlässigt und angenommen,
dass $m_\su{Glas} = 0$ gilt. Dies ist natürlich definitiv nicht der Fall.
Zudem wurde die Dichte von $\su{C_6O_{12}Pr_2}$ selbstständig berechnet und unterliegt
somit auch einem systematischen Fehler, da wir die Probe als zylinderförmig angenommen
haben und die Abmessungen der Probe nur mit einem Lineal bestimmt haben.

Trotzdem lässt sich der riesen Fehler nur durch einen Defekt in der Messapparatur
erklären.
